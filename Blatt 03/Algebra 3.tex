\documentclass[a4paper,10pt]{article}
%\documentclass[a4paper,10pt]{scrartcl}

\usepackage{xltxtra}
\usepackage[ngerman]{babel}
\usepackage{amsmath}
\usepackage{amssymb}
\usepackage{amsthm}
\usepackage{mathtools}
\usepackage{nicefrac}
\usepackage{enumerate}
\usepackage{leftidx}

\theoremstyle{definition}
\newtheorem{beh}{Behauptung}
\newtheorem{bem}[beh]{Bemerkung}
\newtheorem{lem}[beh]{Lemma}
\newtheorem*{anm}{Anmerkung}
\newtheorem*{ia}{Induktionsanfang}
\newtheorem*{is}{Induktionsschritt}

\renewcommand{\thesection}{Aufgabe 3.\arabic{section}.}
\renewcommand{\thesubsection}{(\roman{subsection})}
\renewcommand{\thesubsubsection}{}

\newcommand{\N}{\mathbb{N}}
\newcommand{\Z}{\mathbb{Z}}
\newcommand{\Q}{\mathbb{Q}}
\newcommand{\R}{\mathbb{R}}
\newcommand{\C}{\mathbb{C}}
\newcommand{\Sn}{\mathfrak{S}}
\newcommand{\dt}{\,\text{d}t}
\newcommand{\F}[1]{\mathbb{F}_{#1}}
\newcommand{\id}{\operatorname{id}}
\newcommand{\ord}{\operatorname{ord}}
\newcommand{\inn}{\operatorname{inn}}
\newcommand{\sgn}{\operatorname{sgn}}
\newcommand{\kchar}{\operatorname{char}}
\newcommand{\Id}{\operatorname{Id}}
\newcommand{\GL}[2]{\operatorname{GL}(#1,#2)}
\newcommand{\Img}{\operatorname{Im}}
\newcommand{\Ker}{\operatorname{Ker}}
\newcommand{\Hom}{\operatorname{Hom}}
\newcommand{\End}{\operatorname{End}}
\newcommand{\vect}[1]{\begin{pmatrix}#1\end{pmatrix}}
\newcommand{\gen}[1]{\left\langle#1\right\rangle}

\newenvironment{lgs}[1][c]{\left\{\setlength{\arraycolsep}{1pt}\begin{array}{#1}}{\end{array}\right.}

\makeatletter
\renewcommand*\env@matrix[1][*\c@MaxMatrixCols c]{%
  \hskip -\arraycolsep
  \let\@ifnextchar\new@ifnextchar
  \array{#1}}
\makeatother

\setromanfont[Mapping=tex-text]{Linux Libertine O}
% \setsansfont[Mapping=tex-text]{DejaVu Sans}
% \setmonofont[Mapping=tex-text]{DejaVu Sans Mono}
\parindent0pt

\title{\textsc{Einführung in die Algebra \\ \Large Blatt 3}}
\author{Jendrik Stelzner}
\date{\today}

\begin{document}
\maketitle





\section{}
Für $n \in \{1,2\}$ ist $\Sn_n$ kommutativ, also $Z(\Sn_1) = \Sn_1$ und $Z(\Sn_2) = \Sn_2$. Für $n \geq 3$ ist $Z(\Sn_n) = \{1\}$ die triviale Gruppe:\\
Sei $\pi \in Z(\Sn_n)$ und $\sigma := \vect{1&2&\ldots&n-1&n} \in \Sn_n$ die Rotation mit $\sigma(1)=2$. Es gibt dann $s \in \{0,\ldots,n-1\}$ mit $\pi(1) = \sigma^s(1)$. Da $\pi$ mit allen Elementen in $\Sn_n$ kommutiert, ist damit für alle $m \in \{1,\ldots,n\}$
\[
 \pi(m)
 = \pi(\sigma^m(1))
 = \sigma^m(\pi(1))
 = \sigma^m(\sigma^s(1))
 \underset{(*)}{=} \sigma^s(\sigma^m(1))
 = \sigma^s(m),
\]
also $\sigma^s = \pi$, wobei bei $(*)$ die Kommutativität von $\gen{\sigma}$ genutzt wird. Da wegen der Kommutativität von $\sigma^s = \pi$
\[
 \tau_{12} = \sigma^s\ \tau_{12}\ (\sigma^s)^{-1} = \tau_{(1+s)(2+s)},
\]
wobei $\tau_{kl}$ die Transposition von $k \bmod n$ und $l \bmod n$ bezeichnet, muss $s = 0$, also $\pi = \sigma^s = \id$. Dass $\id \in Z(\Sn_n)$ ist klar, da $Z(\Sn_n) \subseteq \Sn_n$ eine Untergruppe ist.





\section{}
\begin{lem}\label{lem:normalprodukt}
 Sei $G$ eine endliche Gruppe, und seien $N_1, N_2 \subseteq G$ Normalteiler in $G$ mit $\ord G = \ord N_1 \cdot \ord N_2$ und $N_1 \cap N_2 = \{1\}$. Dann ist $G \cong N_1 \times N_2$.
\end{lem}
\begin{proof}[Beweis des Lemmas:]
 Für $n_1 \in N_1$ und $n_2 \in N_2$ ist $n_1 n_2 = n_2 n_1$, denn $N_1$ und $N_2$ sind als Normalteiler invariant bezüglich Konjugation, weshalb
 \[
  n_1 \cdot \underbrace{n_2 n_1^{-1} n_2^{-1}}_{\in N_1} \in N_1 \text{ und }
  \underbrace{n_1 n_2 n_1^{-1}}_{\in N_2} \cdot n_2^{-1} \in N_2.
 \]
 Aus $N_1 \cap N_2 = \{1\}$ folgt $n_1 n_2 n_1^{-1} n_2^{-1} = 1$, also $n_1 n_2 = n_2 n_1$. Es sei nun
 \[
  \varphi : N_1 \times N_2 \rightarrow G, (n_1, n_2) \mapsto n_1 n_2.
 \]
 Aus der eben gezeigten Kommutativität folgt, dass $\varphi$ ein Gruppenhomomorphismus ist, da für alle $(n_1,n_2), (n'_1, n'_2) \in N_1 \times N_2$
 \begin{align*}
  \varphi( (n_1,n_2) (n'_1, n'_2) )
  = \varphi( (n_1 n'_1, n_2 n'_2) )
  &= n_1 n'_1 n_2 n'_2 \\
  &= n_1 n_2 n'_1 n'_2
  = \varphi(n_1, n_2) \varphi(n'_1, n'_2).
 \end{align*}
 $\varphi$ ist injektiv, da für $(n_1, n_2) \in \Ker \varphi$
 \[
  1 = \varphi(n_1, n_2) = n_1 n_2, \text{ also } N_2 \ni n_2 = n_1^{-1} \in N_1,
 \]
und daher $n_1 = n_2 = 1$. Da
 \[
  \ord N_1 \times N_2 = \ord N_1 \cdot \ord N_2 = \ord G < \infty
 \]
 folgt aus der Injektivität von $\varphi$ direkt die Bijektivität. Da $\varphi$ ein Gruppenisomorphismus ist, ist $G \cong N_1 \times N_2$.
\end{proof}

\begin{bem}\label{bem:primschnitt}
 Sind $p$ und $q$ verschiedene Primzahlen und sind $G$ und $H$ Untergruppen einer Gruppe, so dass $G$ eine $p$-Gruppe und $H$ eine $q$-Gruppe ist, so ist $G \cap H = \{1\}$.
\end{bem}
\begin{proof}
 Nach dem Satz von Lagrange ist $\ord G\ \cap\ H$ ein Teiler von $\ord G$ und $\ord H$, aus der Teilerfremdheit von $p$ und $q$ folgt daher, dass $\ord G\ \cap\ H = 1$, also $G\ \cap\ H = \{1\}$.
\end{proof}


Es sei $G$ eine Gruppe der Ordnung $\ord G = 5929 = 11^2 \cdot 7^2$. Es sei $s$ die Anzahl der $11$-Sylowgruppen in $G$. Nach den Sylowsätzen ist
\[
 s \mid \ord G = 11^2 \cdot 7^2, \qquad s \equiv 1 \mod 11.
\]
Da $s \equiv 1 \bmod 11$ ist $s$ kein Vielfaches von $11$. Zusammen mit $s \mid 11^2 \cdot 7^2$ ergibt sich daraus, dass $s \in \{1,7,49\}$. Da jedoch
\[
 7 \not\equiv 1 \not\equiv 49 \mod 11
\]
muss $s = 1$. Insbesondere ist die eindeutige $11$-Sylowgruppe $S_{11} \subseteq G$, die nach den Sylowsätzen existiert, daher ein Normalteiler in $G$.
Analoges ergibt sich für die Anzahl $r$ der $7$-Sylowgruppen in $G$: Es muss
\[
 r \mid \ord G = 11^2 \cdot 7^2, \qquad r \equiv 1 \mod 7,
\]
also $r \in \{1,11,121\}$, und wegen
\[
 11 \not\equiv 1 \not\equiv 121 \mod 7
\]
daher $r = 1$. Also ist auch die eindeutige, nach den Sylowsätzen existierende $7$-Sylowgruppe $S_7 \subseteq G$ ein Normalteiler in $G$.

Da $\ord S_{11} = 11^2$ und $\ord S_7 = 7^2$ ist $\ord G = \ord S_{11} \cdot \ord S_7$. Nach Bemerkung \ref{bem:primschnitt} ist auch $S_{11} \cap S_7 = \{1\}$. Nach Lemma \ref{lem:normalprodukt} ist ist also $G \cong S_{11} \times S_7$. Da $\ord S_{11}$ und $\ord S_7$ Quadrate von Primzahlen sind, sind $S_{11}$ und $S_7$, wie aus der Vorlesung bekannt, abelsch. Also ist $G$ als Produkt abelscher Gruppen ebenfalls abelsch.





\section{}


\subsection{}\label{ssec:fak}
Durch
\[
 G \times G/H \rightarrow G/H, (g,aH) \mapsto gaH
\]
wird eine Aktion von $G$ auf der Menge der Linksnebenklassen $G/H$ definiert (die Wohldefiniertheit ist klar). Diese Aktion entspricht dem Gruppenhomomorphismus
\[
 \varphi : G \rightarrow S(G/H), g \mapsto (aH \mapsto gaH).
\]
Es ist daher
\[
 \ord G
 = \ord \Ker \varphi \cdot \ord \Img \varphi.
\]
Da $\ord \Img \varphi$ nach dem Satz von Lagrange ein Teiler von $
\ord S(G/H) = (G:H)!$ ist, $\ord G$ jedoch kein Teiler von $(G : H)!$, muss $\ord \Ker \varphi \neq 1$, also $\Ker \varphi$ nichttrivial sein. $\Ker \varphi$ ist als Kern eines Gruppenhomomorphismus normal in $G$. Auch ist $\Ker \varphi \subseteq H$, denn für alle $n \in \Ker \varphi$ ist $nH = H$, da $H \in G/H$, also $n \in H$. Damit ist $\Ker \varphi \subseteq H$ ein nichttrivialer Normalteiler von $G$.


\subsection{}
Es gilt zu bemerken, dass die Aussage nur unter der zusätzlichen Bedingung $k > 0$ gilt: Ansonsten ist die triviale Gruppe mit $p=2$, $k=0$ und $m=1$ ein Gegenbeispiel. Es wird daher die Aussage unter der zusätzlichen Annahme $k > 0$ gezeigt:

Nach den Sylowsätzen gibt es eine $p$-Sylowgruppe $S \subseteq G$. Da $p \nmid m$ ist $\ord S = p^k$, also $(G : S) = m$. Wegen den Annahmen $k > 0$ und $p > m$ ist daher
\[
\ord G = p^k m \nmid m! = (G : S)!.
\]
Nach Aufgabenteil \textbf{\ref{ssec:fak}} gibt es daher einen nicht trivialen Normalteiler $N \subseteq S \subseteq G$ von $G$ in $S$.





\section{}


\subsection{}\label{ssc:Sylowschnitt}
Da $S \subseteq H$ ein $p$-Sylowgruppe in $H$ ist, ist $S \subseteq G$ eine $p$-Gruppe in $G$. Nach den Sylowsätzen gibt es daher eine $p$-Sylowgruppe $T \subseteq G$ mit $S \subseteq T$. Da $S \subseteq H$ und $S \subseteq T$ ist $S \subseteq T \cap H$. Da $T \cap H \subseteq T$ eine $p$-Gruppe in $H$ ist, und $S$ als Sylowgruppe in $H$ bereits eine in $H$ maximale $p$-Gruppe ist, muss bereits $S = T \cap H$.


\subsection{}
Sei $S \subseteq G$ eine normale $p$-Sylowgruppe in $G$. Sei $T := S \cap H$. Als Untergruppe $T \subseteq S$ ist $T$ eine $p$-Gruppe. Wie aus der Vorlesung bekannt ist $T$ normal in $H$. Nach den Sylowsätzen gibt es eine Sylowgruppe $T' \subseteq H$ mit $T \subseteq T'$. Nach Aufgabenteil \textbf{\ref{ssc:Sylowschnitt}} gibt es eine Sylowgruppe $S' \subseteq G$ mit $T' = S' \cap H$. Da $S$ normal ist, ist $S$, wie aus der Vorlesung bekannt, die einzige $p$-Sylowgruppe in $G$. Also muss $S = S'$, und damit $T = T'$. Also ist $T$ eine normale $p$-Sylowgruppen $H$.


\subsection{}
Sei $S \subseteq H$ eine $p$-Sylowgruppe; eine solche existiert nach den Sylowsätzen. Nach Aufgabenteil \textbf{\ref{ssc:Sylowschnitt}} gibt es eine $p$-Sylowgruppe $T' \subseteq G$ mit $S = T' \cap H$. Da $T$ und $T'$ $p$-Sylowgruppen in $G$ sind, sind sie konjugiert zueinander, d.h. es gibt ein $g \in G$ mit $T' = g\, T g^{-1}$. Da $H$ normal in $G$ ist, ist $gHg^{-1} = H$, sowie
\[
 g^{-1}Sg \subseteq g^{-1}Hg = H.
\]
Insbesondere ist $g^{-1}Sg$ wieder eine Untergruppe von $H$ mit $\ord g^{-1}Sg = \ord S$ (da $\inn_{g^{-1}}$ ein Automorphismus ist), also eine $p$-Sylowgruppe in $H$. Da nun
\begin{align*}
 T \cap H
 &= g^{-1}g(T \cap H)g^{-1}g
 = g^{-1} \left( g(T \cap H)g^{-1} \right) g \\
 &= g^{-1} \left( \left(g\,Tg^{-1}\right) \cap \left(gHg^{-1}\right) \right) g
 = g^{-1} (T' \cap H) g
 = g^{-1} S g
\end{align*}
ist diese $p$-Sylowgruppe gerade $T \cap H$.





\section{}
Es ist $\ord G = 132 = 2^2 \cdot 3 \cdot 11$.  Für $p \in \{2,3,11\}$ sei $s_p$ die Anzahl der $p$-Sylowgruppen in $G$. Nach den Sylowsätzen ist für alle $p \in \{2,3,11\}$
\[
 s_p \mid \ord G = 2^2 \cdot 3 \cdot 11, \qquad s_p \equiv 1 \mod p.
\]
Aus diesen Bedingungen ergibt sich direkt, dass $s_2 \in \{1,3,11,33\}$, $s_3 \in \{1,4,22\}$ und $s_{11} \in \{1,12\}$ (vgl. \textbf{Aufgabe 3.2.}). Dabei besitzt $G$ für $p \in \{2,3,11\}$ genau dann eine normale $p$-Sylowgruppe, wenn $s_p = 1$.

Angenommen, $G$ besitzt keine normale $p$-Sylowgruppe. Dann muss $s_2 \geq 3$, $s_3 \geq 4$ und $s_{11} = 12$. Es ergeben sich dann die folgenden Beobachtungen:
\begin{itemize}
 \item Die $2$-Sylowgruppen haben Ordung $2^2=4$, die $3$-Sylowgruppen Ordnung $3$ und die $11$-Sylowgruppen Ordnung $11$.
 \item Sylowgruppen zu verschiedenen Primzahlen haben nach Bemerkung \ref{bem:primschnitt} triviale Schnitte.
 \item $p$-Sylowgruppen gleicher Primzahlen haben für $p = 11$ oder $p = 3$ ebenfalls nur trivale Schnitte, da für zwei verschiedene solche $p$-Sylowgruppen $S_p$ und $S'_p$ für den Schnitt $S_p \cap S'_p$ die Ordnung $\ord S_p \cap S'_p$ ein Teiler von $\ord S_p = \ord S'_p = p$ ist, also im Falle $\ord S_p \cap S'_p \neq 1$ bereits $\ord S_p \cap S'_p = p$ und damit $S_p = S_p \cap S'_p = S'_p$.
 \item Der Schnitt zweier verschiedener $2$-Sylowgruppen ist entweder trivial oder von Ordnung $2$. Drei solche Gruppen verfügen zusammen über mindestens $5$ verschiedene Elemente.
\end{itemize}
Zusammen ergibt sich damit, dass $G$ mindestens
\[
 12 \cdot 11 - 12 + 4 \cdot 3 - 4 + 5 = 133
\]
verschiedene Elemente hat. Da aber $\ord = 132$ kann dies nicht sein. Also ist $s_p = 1$ für ein $p \in \{2,3,11\}$, weshalb $G$ ein normale $p$-Sylowgruppe besitzt.


 









\end{document}
