\documentclass[a4paper,10pt]{article}
%\documentclass[a4paper,10pt]{scrartcl}

\usepackage{xltxtra}
\usepackage[ngerman]{babel}
\usepackage{amsmath}
\usepackage{amssymb}
\usepackage{amsthm}
\usepackage{mathtools}
\usepackage{nicefrac}
\usepackage{enumerate}
\usepackage{leftidx}

\theoremstyle{definition}
\newtheorem*{beh}{Behauptung}
\newtheorem*{bem}{Bemerkung}
\newtheorem*{lem}{Lemma}
\newtheorem*{anm}{Anmerkung}
\newtheorem*{ia}{Induktionsanfang}
\newtheorem*{is}{Induktionsschritt}

\renewcommand{\thesection}{Aufgabe 3.\arabic{section}.}
\renewcommand{\thesubsection}{(\roman{subsection})}
\renewcommand{\thesubsubsection}{}

\newcommand{\N}{\mathbb{N}}
\newcommand{\Z}{\mathbb{Z}}
\newcommand{\Q}{\mathbb{Q}}
\newcommand{\R}{\mathbb{R}}
\newcommand{\C}{\mathbb{C}}
\newcommand{\Sn}{\mathfrak{S}}
\newcommand{\dt}{\,\text{d}t}
\newcommand{\F}[1]{\mathbb{F}_{#1}}
\newcommand{\id}{\operatorname{id}}
\newcommand{\ord}{\operatorname{ord}}
\newcommand{\inn}{\operatorname{inn}}
\newcommand{\sgn}{\operatorname{sgn}}
\newcommand{\kchar}{\operatorname{char}}
\newcommand{\Id}{\operatorname{Id}}
\newcommand{\GL}[2]{\operatorname{GL}(#1,#2)}
\newcommand{\Img}{\operatorname{Im}}
\newcommand{\Ker}{\operatorname{Ker}}
\newcommand{\Hom}{\operatorname{Hom}}
\newcommand{\End}{\operatorname{End}}
\newcommand{\vect}[1]{\begin{pmatrix}#1\end{pmatrix}}
\newcommand{\gen}[1]{\left\langle#1\right\rangle}

\newenvironment{lgs}[1][c]{\left\{\setlength{\arraycolsep}{1pt}\begin{array}{#1}}{\end{array}\right.}

\makeatletter
\renewcommand*\env@matrix[1][*\c@MaxMatrixCols c]{%
  \hskip -\arraycolsep
  \let\@ifnextchar\new@ifnextchar
  \array{#1}}
\makeatother

\setromanfont[Mapping=tex-text]{Linux Libertine O}
% \setsansfont[Mapping=tex-text]{DejaVu Sans}
% \setmonofont[Mapping=tex-text]{DejaVu Sans Mono}
\parindent0pt

\title{\textsc{Einführung in die Algebra \\ \Large Blatt 3}}
\author{Jendrik Stelzner}
\date{\today}

\begin{document}
\maketitle





\section{}
Für $n = \{1,2\}$ ist $\Sn_n$ kommutativ, also $Z(\Sn_1) = \Sn_1$ und $Z(\Sn_2) = \Sn_2$. Für $n \geq 3$ ist $Z(\Sn_n) = \{1\}$ die triviale Untergruppe:\\
Sei $\pi \in Z(\Sn_n)$ und $\sigma := \vect{1&2&\ldots&n-1&n} \in \Sn_n$ die Rotation mit $\sigma(1)=2$. Es gibt dann $s \in \{0,\ldots,n-1\}$ mit $\pi(1) = \sigma^s(1)$. Da $\pi$ mit allen Elementen in $\Sn_n$ kommutiert, ist damit für alle $m \in \{1,\ldots,n\}$
\[
 \pi(m)
 = \pi(\sigma^m(1))
 = \sigma^m(\pi(1))
 = \sigma^m(\sigma^s(1))
 \underset{(*)}{=} \sigma^s(\sigma^m(1))
 = \sigma^s(m),
\]
also $\sigma^s = \pi$, wobei bei $(*)$ die Kommutativität von $\gen{\sigma}$ genutzt wird. Da wegen der Kommutativität von $\sigma^s = \pi$
\[
 \tau_{12} = \sigma^s\ \tau_{12}\ (\sigma^s)^{-1} = \tau_{(1+s)(2+s)},
\]
wobei $\tau_{kl}$ die Transposition von $k \bmod n$ und $l \bmod n$ bezeichnet, muss $s = 0$, also $\pi = \sigma^s = \id$. Dass $\id \in Z(\Sn_n)$ ist allerdings klar, da $Z(\Sn_n) \subseteq \Sn_n$ eine Untergruppe ist.








\end{document}
