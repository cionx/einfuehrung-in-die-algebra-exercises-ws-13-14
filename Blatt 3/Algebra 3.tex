\documentclass[a4paper,10pt]{article}
%\documentclass[a4paper,10pt]{scrartcl}

\usepackage{xltxtra}
\usepackage[ngerman]{babel}
\usepackage{amsmath}
\usepackage{amssymb}
\usepackage{amsthm}
\usepackage{mathtools}
\usepackage{nicefrac}
\usepackage{enumerate}
\usepackage{leftidx}

\theoremstyle{definition}
\newtheorem*{beh}{Behauptung}
\newtheorem*{bem}{Bemerkung}
\newtheorem*{lem}{Lemma}
\newtheorem*{anm}{Anmerkung}
\newtheorem*{ia}{Induktionsanfang}
\newtheorem*{is}{Induktionsschritt}

\renewcommand{\thesection}{Aufgabe 3.\arabic{section}.}
\renewcommand{\thesubsection}{(\roman{subsection})}
\renewcommand{\thesubsubsection}{}

\newcommand{\N}{\mathbb{N}}
\newcommand{\Z}{\mathbb{Z}}
\newcommand{\Q}{\mathbb{Q}}
\newcommand{\R}{\mathbb{R}}
\newcommand{\C}{\mathbb{C}}
\newcommand{\Sn}{\mathfrak{S}}
\newcommand{\dt}{\,\text{d}t}
\newcommand{\F}[1]{\mathbb{F}_{#1}}
\newcommand{\id}{\operatorname{id}}
\newcommand{\ord}{\operatorname{ord}}
\newcommand{\inn}{\operatorname{inn}}
\newcommand{\sgn}{\operatorname{sgn}}
\newcommand{\kchar}{\operatorname{char}}
\newcommand{\Id}{\operatorname{Id}}
\newcommand{\GL}[2]{\operatorname{GL}(#1,#2)}
\newcommand{\Img}{\operatorname{Im}}
\newcommand{\Ker}{\operatorname{Ker}}
\newcommand{\Hom}{\operatorname{Hom}}
\newcommand{\End}{\operatorname{End}}
\newcommand{\vect}[1]{\begin{pmatrix}#1\end{pmatrix}}
\newcommand{\gen}[1]{\left\langle#1\right\rangle}

\newenvironment{lgs}[1][c]{\left\{\setlength{\arraycolsep}{1pt}\begin{array}{#1}}{\end{array}\right.}

\makeatletter
\renewcommand*\env@matrix[1][*\c@MaxMatrixCols c]{%
  \hskip -\arraycolsep
  \let\@ifnextchar\new@ifnextchar
  \array{#1}}
\makeatother

\setromanfont[Mapping=tex-text]{Linux Libertine O}
% \setsansfont[Mapping=tex-text]{DejaVu Sans}
% \setmonofont[Mapping=tex-text]{DejaVu Sans Mono}
\parindent0pt

\title{\textsc{Einführung in die Algebra \\ \Large Blatt 3}}
\author{Jendrik Stelzner}
\date{\today}

\begin{document}
\maketitle





\section{}
Für $n = \{1,2\}$ ist $\Sn_n$ kommutativ, also $Z(\Sn_1) = \Sn_1$ und $Z(\Sn_2) = \Sn_2$. Für $n \geq 3$ ist $Z(\Sn_n) = \{1\}$ die triviale Untergruppe:\\
Sei $\pi \in Z(\Sn_n)$ und $\sigma := \vect{1&2&\ldots&n-1&n} \in \Sn_n$ die Rotation mit $\sigma(1)=2$. Es gibt dann $s \in \{0,\ldots,n-1\}$ mit $\pi(1) = \sigma^s(1)$. Da $\pi$ mit allen Elementen in $\Sn_n$ kommutiert, ist damit für alle $m \in \{1,\ldots,n\}$
\[
 \pi(m)
 = \pi(\sigma^m(1))
 = \sigma^m(\pi(1))
 = \sigma^m(\sigma^s(1))
 \underset{(*)}{=} \sigma^s(\sigma^m(1))
 = \sigma^s(m),
\]
also $\sigma^s = \pi$, wobei bei $(*)$ die Kommutativität von $\gen{\sigma}$ genutzt wird. Da wegen der Kommutativität von $\sigma^s = \pi$
\[
 \tau_{12} = \sigma^s\ \tau_{12}\ (\sigma^s)^{-1} = \tau_{(1+s)(2+s)},
\]
wobei $\tau_{kl}$ die Transposition von $k \bmod n$ und $l \bmod n$ bezeichnet, muss $s = 0$, also $\pi = \sigma^s = \id$. Dass $\id \in Z(\Sn_n)$ ist allerdings klar, da $Z(\Sn_n) \subseteq \Sn_n$ eine Untergruppe ist.





\section{}





\section{}


\subsection{}\label{ssec:fak}
Durch
\[
 G \times G/H \rightarrow G/H, (g,aH) \mapsto gaH
\]
wird eine Aktion von $G$ auf der Menge der Linksnebenklassen $G/H$ definiert. Diese Aktion entspricht dem Gruppenhomomorphismus
\[
 \varphi : G \rightarrow S(G/H), g \mapsto (aH \mapsto gaH).
\]
Es ist daher
\[
 \ord G
 = \ord \Ker \varphi \cdot \ord \Img \varphi.
\]
Da $\ord \Img \varphi$ ein Teiler von $\ord S(G/H) = (G:H)!$ ist, $\ord G$ jedoch kein Teiler von $(G : H)!$, muss $\ord \Ker \varphi \neq 1$, also $\Ker \varphi$ nichttrivial sein. $\Ker \varphi$ ist als Kern eines Gruppenhomomorphismus normal in $G$. Es ist $\Ker \varphi \subseteq H$, denn für alle $n \in \Ker \varphi$ ist $nH = H$, da $H$ eine Linksnebenklasse in $G/H$ ist, also $n \in H$. Damit ist $\Ker \varphi \subseteq H$ ein nichttrivialer Normalteiler von $G$.


\subsection{}
Es gilt zu bemerken, dass die Aussage nur unter der zusätzlichen Bedingung $k > 0$ gilt: Ansonsten ist die triviale Gruppe mit $p=2$, $k=0$ und $m=1$ ein Gegenbeispiel. Es wird daher die Aussage unter der zusätzlichen Annahme $k > 0$ gezeigt:

Nach den Sylowsätzen gibt es eine $p$-Sylowgruppe $S \subseteq G$. Da $S$ eine maximale $p$-Untergruppe ist, ist $\ord S = p^k$, also $(G : S) = m$. Wegen den Annahmen $k > 0$ und $p > m$ ist daher
\[
\ord G = p^k m \nmid m! = (G : S)!.
\]
Nach Aufgabenteil \textbf{\ref{ssec:fak}} gibt es daher einen nicht trivialen Normalteiler $N \subseteq S \subseteq G$ von $G$ in $S$.





\section{}


\subsection{}\label{ssc:Sylowschnitt}
Da $S \subseteq H$ ein $p$-Sylowgruppe in $H$ ist, ist $S \subseteq G$ eine $p$-Gruppe in $G$. Nach den Sylowsätzen gibt es daher eine $p$-Sylowgruppe $T \subseteq G$ mit $S \subseteq T$. Da $S \subseteq H$ und $S \subseteq T$ ist $S \subseteq T \cap H$. Da $T \cap H \subseteq T$ eine $p$-Gruppe in $H$ ist, und $S$ als Sylowgruppe in $H$ bereits eine in $H$ maximale $p$-Gruppe ist, muss bereits $S = T \cap H$.


\subsection{}
Sei $S \trianglelefteq G$ eine normale $p$-Sylowgruppe in $G$. Sei $T := S \cap H$. Als Untergruppe ist $T \subseteq S$ eine $p$-Gruppe. Wie aus der Vorlesunge bekannt ist $T$ normal in $H$. Nach den Sylowsätzen gibt es eine Sylowgruppe $T' \subseteq H$ mit $T \subseteq T'$. Nach Aufgabenteil \textbf{\ref{ssc:Sylowschnitt}} gibt es daher eine Sylowgruppe $S' \subseteq G$ mit $T' = S' \cap H$. Da $S$ normal ist, ist $S$, wie aus der Vorlesung bekannt, die einzige $p$-Sylowgruppe in $G$. Also muss $S = S'$, und damit $T = T'$. Also ist $T$ eine normale $p$-Sylowgruppen $H$.


\subsection{}
Sei $S \subseteq H$ eine $p$-Sylowgruppe; eine solche existiert nach den Sylowsätzen. Nach Aufgabenteil \textbf{\ref{ssc:Sylowschnitt}} gibt es eine $p$-Sylowgruppe $T' \subseteq G$ mit $S = T' \cap H$. Da $T$ und $T'$ $p$-Sylowgruppen in $G$ sind, sind sie konjugiert zueinander, d.h. es gibt ein $g \in G$ mit $T' = g\, T g^{-1}$. Da $H$ normal in $G$ ist, ist auch $gHg^{-1} = H$. Es ist daher
\[
 S = T' \cap H = g\, T g^{-1} \cap g H g^{-1} = g(T \cap H)g^{-1},
\]
wobei genutzt wird, dass $\inn_g$ als Automorphismus bijektiv ist. Da $S$ und $T \cap H$ konjugiert zueinder sind, und $S$ eine $p$-Sylowgruppe in $H$ ist, ist auch $T \cap H$ eine $p$-Sylowgruppe in $H$.














\end{document}
