\documentclass[a4paper,10pt]{article}
%\documentclass[a4paper,10pt]{scrartcl}

\usepackage{xltxtra}
\usepackage[ngerman]{babel}
\usepackage{amsmath}
\usepackage{amssymb}
\usepackage{amsthm}
\usepackage{mathtools}
\usepackage{nicefrac}
\usepackage{enumerate}
\usepackage{leftidx}

\theoremstyle{definition}
\newtheorem*{beh}{Behauptung}
\newtheorem*{bem}{Bemerkung}
\newtheorem*{lem}{Lemma}
\newtheorem*{anm}{Anmerkung}
\newtheorem*{ia}{Induktionsanfang}
\newtheorem*{is}{Induktionsschritt}

\renewcommand{\thesection}{Aufgabe 2.\arabic{section}.}
\renewcommand{\thesubsection}{(\roman{subsection})}
\renewcommand{\thesubsubsection}{}

\newcommand{\N}{\mathbb{N}}
\newcommand{\Z}{\mathbb{Z}}
\newcommand{\Q}{\mathbb{Q}}
\newcommand{\R}{\mathbb{R}}
\newcommand{\C}{\mathbb{C}}
\newcommand{\dt}{\,\text{d}t}
\newcommand{\F}[1]{\mathbb{F}_{#1}}
\newcommand{\id}{\operatorname{id}}
\newcommand{\ord}{\operatorname{ord}}
\newcommand{\inn}{\operatorname{inn}}
\newcommand{\sgn}{\operatorname{sgn}}
\newcommand{\kchar}{\operatorname{char}}
\newcommand{\Id}{\operatorname{Id}}
\newcommand{\GL}[2]{\operatorname{GL}(#1,#2)}
\newcommand{\Img}{\operatorname{Im}}
\newcommand{\Ker}{\operatorname{Ker}}
\newcommand{\Hom}{\operatorname{Hom}}
\newcommand{\End}{\operatorname{End}}
\newcommand{\vect}[1]{\begin{pmatrix}#1\end{pmatrix}}
\newcommand{\gen}[1]{\left\langle#1\right\rangle}

\newenvironment{lgs}[1][c]{\left\{\setlength{\arraycolsep}{1pt}\begin{array}{#1}}{\end{array}\right.}

\makeatletter
\renewcommand*\env@matrix[1][*\c@MaxMatrixCols c]{%
  \hskip -\arraycolsep
  \let\@ifnextchar\new@ifnextchar
  \array{#1}}
\makeatother

\setromanfont[Mapping=tex-text]{Linux Libertine O}
% \setsansfont[Mapping=tex-text]{DejaVu Sans}
% \setmonofont[Mapping=tex-text]{DejaVu Sans Mono}
\parindent0pt

\title{\textsc{Einführung in die Algebra \\ \Large Blatt 2}}
\author{Jendrik Stelzner}
\date{\today}

\begin{document}
\maketitle





\section{}
Auf $X'$ sei eine $G$-Aktion definiert als
\[
 G \times X' \rightarrow X', (g,H(\tilde{g},x)) \mapsto g * H(\tilde{g},x) := H(g \tilde{g}, x),
\]
wobei $H(\tilde{g},x) \in X'$ die Bahn von $(\tilde{g},x) \in G \times X$ bezeichnet. [\emph{Korrektur: Es fehlt der Nachweis der Wohldefiniertheit, d.h. die Unabhängigkeit vom Repräsentanten der Bahn.}] Es handelt sich bei $*$ um ein $G$-Aktion, da für alle $H(g,x) \in X'$
\[
 1 * H(g,x) = H(1 \cdot g, x) = H(g,x)
\]
und für $g_1, g_2 \in G$ und $H(\tilde{g},x) \in X'$
\[
 g_1 * (g_2 * H(\tilde{g},x))
 = g_1 * H(g_2 \tilde{g}, x)
 = H(g_1 g_2 \tilde{g}, x)
 = (g_1 g_2) * H(\tilde{g},x).
\]
Weiter sei die $H$-Abbildung $\Phi$ definiert als
\[
 \Phi : X \mapsto X', x \mapsto H(1,x).
\]
$\Phi$ ist eine $H$-Abbildung, da für alle $h \in H$ und $x \in X$
\[
 \Phi(hx) = H(1,hx) = H(h,x) = h * H(1,x) = h * \Phi(x).
\]
Dabei gilt des zu bemerken, dass $H(1,hx) = H(h,x)$, da
\[
 H(1,hx) = (H h^{-1}) (1,hx) = H (h^{-1}(1,hx)) = H(h,x),
\]
wobei die Multiplikation die in der Aufgabe gegebene $H$-Aktion auf $G \times X$ ist.

Es gilt nun zu zeigen, dass $f \mapsto f \circ \Phi$ eine Bijektion zwischen der Menge der $G$-Abbildungen von $X'$ nach $Y$ und der Menge der $H$-Abbildungen von $X$ nach $Y$ definiert.

Es gilt zunächst die Wohldefiniertheit der induzierten Abbildung zu überprüfen, d.h. dass für eine $G$-Abbildung $f : X' \rightarrow Y$ die Komposition $f \circ \Phi$ eine $H$-Abbildung von $X$ nach $Y$ ist. Dies ist aber der Fall, da offenbar $f \circ \Phi : X \rightarrow Y$, und für alle $h \in H$ und $x \in X$,
\begin{align*}
 (f \circ \Phi)(hx)
 &= f( H(1,hx) )
 = f( H(h,x) ) \\
 &= f(h * H(1,x) )
 = h f( H(1,x) )
 = h (f \circ \Phi)(x).
\end{align*}

Die Injektivität ergibt sich daraus, dass für $G$-Abbildungen $f, f' : X' \rightarrow Y$ mit $f \circ \Phi = f' \circ \Phi$ für alle $H(g,x) \in X'$ gilt
\begin{align*}
 f( H(g,x) )
 &= f( g * H(1,x) )
 = g * f( H(1,x) )
 = g * (f \circ \Phi)(x) \\
 &= g * (f' \circ \Phi)(x)
 = g * f'( H(1,x) )
 = f'( g * H(1,x) )
 = f'( H(g,x) ),
\end{align*}
also $f = f'$.

Die Surjektivität der Abbildung ergibt sich daraus, dass sich für jede $H$-Abbildung $\psi : X \rightarrow Y$ eine $G$-Abbildung $f : X' \rightarrow Y$ konstruieren lässt, so dass $\psi = f \circ \Phi$: Für $H(g,x) \in X'$ sei $f(H(g,x)) :=  g \psi(x)$. Dieser Ausdruck ist wohldefiniert, denn ist $H(g,x) = H(g',x')$, so ist $(g',x') = h(g,x) = (gh^{-1},hx)$ für ein $h \in H$ und somit
\[
 g \psi(x) = g h^{-1} h \psi(x) = g h^{-1} \psi(hx) = g' \psi(x').
\]
$f$ ist eine $G$-Abbildung, da für alle $g \in G$ und $H(\tilde{g},x) \in X'$
\[
 f( g * H(\tilde{g},x) )
 = f( H(g\tilde{g}, x) )
 = g \tilde{g} \psi(x)
 = g f( H(\tilde{g},x) ).
\]





\section{}
Angenommen, es ist $\ord Z = 1$, also $Z = \{1\}$, aber die Anzahl der Konjugationsklassen in $G$ ist echt größer als $\frac{\ord G}{p}$. Sei dann $x_1, \ldots, x_n$ ein Repräsentantensystem der Konjugationsklassen von $G-\{1\}$; nach Annahme ist dabei $n \geq \frac{\ord G}{p}$. Es ist nun $(G : Z_{x_i}) \neq 1$ für $i=1,\ldots,n$, da sonst $Z_{x_i} = G$, also $1 \neq x_i \in Z$, im Widerspruch zur Annahme dass $Z = \{1\}$. Da $(G : Z_{x_i})$ ein Teiler von $\ord G$ ist, muss daher $(G : Z_{x_i}) \geq p$ für $i=1,\ldots,n$. Es ist daher
\[
 \ord G = \ord Z + \sum_{i=1}^n (G : Z_{x_i}) \geq 1 + \frac{\ord G}{p} p = 1 + \ord G,
\]
also $0 \geq 1$, was offensichtlich falsch ist. Also muss, wenn die Anzahl den Konjugationsklassen in $G$ echt größer als $\frac{\ord G}{p}$ ist, $\ord Z \geq 2$.





\section{}
Zunächst gilt es zu bemerken, dass $x \in X$ genau dann ein Fixpunkt von $G$ ist, falls $gx = x$ für alle $g \in G$, also $G_x = G$, und somit $(G : G_x) = 1$. Da $Gx = \{x\}$ ist $x$ auch der einzige Repräsentant der Bahn von $x$.

Sei $x_1, \ldots, x_n$ ein Respräsentantensystem der Bahnen auf $X$. Dann ist nach der Bahnengleichung
\[
 17 = \ord X = \sum_{i=1}^n (G : G_{x_i}),
\]
wobei für je $i=1, \ldots, n$ der Index $(G : G_{x_i})$ ein Teiler von $\ord G$ ist, und daher $(G : G_{x_i}) \in \{1,7,11,77\}$. Betrachtet man nun alle Möglichkeiten, $17$ als Summe aus diesen Zahlen darzustellen, so ergibt sich, dass
\[
 17
 = 11 + 6 \cdot 1
 = 2 \cdot 7 + 3 \cdot 1
 = 7 + 10 \cdot 1
 = 17 \cdot 1.
\]
Es gibt also in jedem möglichen Fall paarweise verschiedene $i_1, i_2, i_3 \in \{1, \ldots, n\}$ mit $(G : G_{x_{i_j}}) = 1$. Also sind $x_{i_1}, x_{i_2}$ und $x_{i_3}$ drei (paarweise verschiedene) Fixpunkte von $G$.





\section{}

\subsection{}
Es sei $x_1, \ldots, x_n$ ein Repräsentantensystem der Bahnen auf $X$. Für $i=1,\ldots,n$ gilt: Da $(G : G_{x_i})$ ein Teiler von $\ord G = p^k$ ist, ist $(G : G_{x_i}) = p^{l_i}$ für ein $l_i \in \{0, 1, \ldots, k\}$. Durch passende Durchnummerierung der $x_i$ kann davon ausgegangen werden, dass es ein $j \in \{0,\ldots,n\}$ gibt, so dass $(G : G_{x_i}) = 1$, also $l_i = 0$, für $i=1,\ldots,j$, und $(G : G_{x_i}) \geq p$, also $l_i \geq 1$ für $i > j$. Insbesondere sind $x_1, \ldots, x_j$ die Fixpunkte von $G$ (vergleiche anfänglich Bemerkung in \textbf{Aufgabe 2.3}).

Nach der Bahnengleichung ist nun
\[
 |X|
 = \sum_{i=1}^n (G : G_{x_i})
 = j + \sum_{i=j+1}^n p^{l_i}
 = j + p \sum_{i=j+1}^n \underbrace{p^{l_i - 1}}_{\in \N},
\]
also insbesondere $\ord X \equiv j\ (\textrm{mod}\ p)$. Da $j$ gerade die Anzahl der Fixpunkte von $G$ ist, ist dies die zu zeigende Aussage.


\subsection{}
Zunächst gilt es zu bemerken, dass die zu zeigende Aussage für $N = \{1\}$ nicht gilt. Im Folgenden wird sie daher unter der zusätzlichen Annahme $N \neq \{1\}$ gezeigt.

Da $N$ ein Normalteiler ist, ist $gNg^{-1} = N$ für alle $g \in G$. Also ist $h \in Gh \subseteq N$ für alle $h \in N$, wobei $Gh = \{ghg^{-1} : g \in G\}$ die Bahn von $h$ unter der Konjugationsaktion ist. Es folgt, dass $N$ die disjunkte Vereinigung von Bahnen ist. Angenommen $N \cap Z = \{1\}$. Da $Z = \{g \in G : Gg = \{g\}\}$ ist dann $1$ das einzige Element $h \in N$ mit $Gh = \{h\}$. Nach der Klassengleichung ist daher
\[
 \ord N = 1 + \sum_{i=1}^n \underbrace{(G : Z_{x_i})}_{\neq 1},
\]
wobei $x_1, \ldots, x_n$ ein Repräsentantensystem der Bahnen in $N-\{1\}$ ist. Da jedes $(G : Z_{x_i})$ als Teiler von $\ord G = p^k$ eine $p$-Potenz ist, und $(G : Z_{x_i}) \neq 1$ für alle $i=1,\ldots, n$, folgt aus der obigen Gleichung, dass
\[
 \ord N \equiv 1 \mod p.
\]
Da jedoch $\ord N$ ebenfalls ein von $1$ verschiedener Teiler von $\ord G = p^k$ ist, muss $\ord N \equiv 0 \bmod p$. Wegen dieses Widerspruches muss $N \cap Z \supsetneq \{1\}$.






\section{}
Da die Ordnung des Zentrums $Z$ von $G$ ein Teiler von $\ord G = p^3$ ist, muss $\ord Z \in \{1, p, p^2, p^3\}$. Es ergibt sich jedoch, dass $\ord Z = p$ sein muss:

Da $Z$ abelsch ist, $G$ jedoch nicht, muss $Z \neq G$ und somit $\ord Z \neq p^3$.

Wäre $\ord Z = p^2$, so wäre $\ord G/Z = \frac{\ord G}{\ord Z} = p$, also $G/Z$ zyklisch, und $G$ daher, wie aus der Vorlesung bekannt, abelsch.

Auch ist aus der Vorlesung bekannt, dass $Z \neq \{1\}$, da $G$ eine nichttriviale $p$-Gruppe ist.

Es muss also $\ord Z = p$.









\end{document}
