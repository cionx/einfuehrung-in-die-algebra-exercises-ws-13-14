\documentclass[a4paper,10pt]{article}
%\documentclass[a4paper,10pt]{scrartcl}

\usepackage{xltxtra}
\usepackage[ngerman]{babel}
\usepackage{amsmath}
\usepackage{amssymb}
\usepackage{amsthm}
\usepackage{mathtools}
\usepackage{nicefrac}
\usepackage{enumerate}
\usepackage{leftidx}

\theoremstyle{definition}
\newtheorem{beh}{Behauptung}
\newtheorem{bem}[beh]{Bemerkung}
\newtheorem{lem}[beh]{Lemma}
\newtheorem*{anm}{Anmerkung}
\newtheorem*{ia}{Induktionsanfang}
\newtheorem*{is}{Induktionsschritt}

\renewcommand{\thesection}{Aufgabe 4.\arabic{section}.}
\renewcommand{\thesubsection}{(\roman{subsection})}
\renewcommand{\thesubsubsection}{}

\newcommand{\N}{\mathbb{N}}
\newcommand{\Z}{\mathbb{Z}}
\newcommand{\Q}{\mathbb{Q}}
\newcommand{\R}{\mathbb{R}}
\newcommand{\C}{\mathbb{C}}
\newcommand{\Sn}{\mathfrak{S}}
\newcommand{\dt}{\,\text{d}t}
\newcommand{\F}[1]{\mathbb{F}_{#1}}
\newcommand{\id}{\operatorname{id}}
\newcommand{\ord}{\operatorname{ord}}
\newcommand{\inn}{\operatorname{inn}}
\newcommand{\sgn}{\operatorname{sgn}}
\newcommand{\kchar}{\operatorname{char}}
\newcommand{\kgV}{\operatorname{kgV}}
\newcommand{\Id}{\operatorname{Id}}
\newcommand{\GL}[2]{\operatorname{GL}(#1,#2)}
\newcommand{\Img}{\operatorname{Im}}
\newcommand{\Ker}{\operatorname{Ker}}
\newcommand{\Hom}{\operatorname{Hom}}
\newcommand{\End}{\operatorname{End}}
\newcommand{\Aut}{\operatorname{Aut}}
\newcommand{\Inn}{\operatorname{Inn}}
\newcommand{\vect}[1]{\begin{pmatrix}#1\end{pmatrix}}
\newcommand{\gen}[1]{\left\langle#1\right\rangle}

\newenvironment{lgs}[1][c]{\left\{\setlength{\arraycolsep}{1pt}\begin{array}{#1}}{\end{array}\right.}

\makeatletter
\renewcommand*\env@matrix[1][*\c@MaxMatrixCols c]{%
  \hskip -\arraycolsep
  \let\@ifnextchar\new@ifnextchar
  \array{#1}}
\makeatother

\setromanfont[Mapping=tex-text]{Linux Libertine O}
% \setsansfont[Mapping=tex-text]{DejaVu Sans}
% \setmonofont[Mapping=tex-text]{DejaVu Sans Mono}
\parindent0pt

\title{\textsc{Einführung in die Algebra \\ \Large Blatt 4}}
\author{Jendrik Stelzner}
\date{\today}

\begin{document}
\maketitle





\section{}
Sei $G$ eine Gruppe der Ordnung $8$. Gibt es ein $g \in G$ mit $\ord g = 8$, so ist $\gen{g} = G$, also $G$ zyklisch, und daher $G \cong \Z/8\Z$. Es wird im Folgenden davon ausgegangen, dass kein solches Element in $G$ existiert.

Zunächst wird der Fall untersucht, dass $G$ abelsch ist ($G$ wird hierfür additiv geschrieben): Durch eine Reihe von Fallunterscheidungen finden wir eine Untergruppe $H \subseteq G$ der Ordnung $4$ und ein Element $g \in G-H$ der Ordnung 2:

Da $G$ eine $2$-Gruppe der Ordnung $8$ ist, gibt es, wie aus der Vorlesung bekannt, eine Untergruppe $H' \subseteq G$ mit $\ord H' = 4$. Sei $g' \in G-H$. Ist $\ord g' = 2$, so sei $H := H'$ und $g := g'$.
Ist $\ord g' = 4$, so wird zwischen zwei Fällen unterschieden: Ist $2g' \not\in H'$, so sei $H := H'$ und $g := 2g'$. Ist $\ord g = 4$, so wird erneut zwischen zwei Fällen unterschieden: Bekanntermaßen ist entweder $H' \cong \Z/4\Z$ oder $H' \cong (\Z/2\Z)^2$. Ist $H' \cong \Z/4\Z$, so ist $H' = \gen{a}$ für ein $a \in H'$. Da $2g' \in H$ mit $\ord 2g' = 2$ muss $2g' = 2a$. Es sei in diesem Fall $H := H'$ und $g := a+g'$. Ist $H' \cong (\Z/2\Z)^2$, so gibt es neben $2g'$ noch ein weiteres $a \in H'$ mit $\ord a = 2$. Es sei dann $H := \gen{g'}$ und $g := a$.

Es ist nun $\ord G = \ord H \cdot \ord \gen{g}$ sowie $H \cap \gen{g} = 1$, wobei $1$ die triviale Gruppe bezeichnet. Da $G$ kommutativ ist, sind $H$ und $\gen{g}$ beide normal in $G$. Es ist daher (wie bereits letzte Woche gezeigt)
\[
 G \cong H \times \gen{g} \cong H \times \Z/2\Z.
\]
Da $\ord H = 4$ ist $H \cong \Z/4\Z$ oder $H \cong (\Z/2\Z)^2$, und daher
\[
 G \cong \Z/4\Z \times \Z/2\Z \text{ oder } G \cong (\Z/2\Z)^3.
\]

Es wird nun der Fall untersucht, dass $G$ nicht abelsch ist: Da $G$ nicht abelsch ist, gibt es ein $a \in G$ mit $a^2 \neq 1$. Da $\ord a$ ein Teiler von $\ord G = 8$ ist, muss dabei $\ord a = 4$. Es ist also $\gen{a} = \{1, a, a^2, a^3\}$. Da $(G : \gen{a}) = 2$ ist $\gen{a}$ normal in $G$. Sei $b \in G-\gen{a}$. Da $\gen{a} \subsetneq \gen{a,b} \subseteq G$ ist $\ord \gen{a,b} > \ord \gen{a} = 4$, also $\ord \gen{a,b} = 8$, und deswegen $G = \gen{a,b}$. Es ist also
\[
 G = \{1, b, a, ab, a^2, a^2 b, a^3, a^3 b\}.
\]
Insbesondere kommutieren $a$ und $b$ nicht miteinander, da $G$ sonst abelsch wäre. Da $\gen{a}$ normal in G ist, ist $bab^{-1} \in \gen{a}$. Da $\ord bab^{-1}  = \ord a$ muss $bab^{-1} = a$ oder $bab^{-1} = a^3$. Da aber $bab^{-1} = a \Leftrightarrow ba = ab$ muss $bab^{-1} = a^3$, also $ba = a^3 b$. Für $b+\gen{a} \in G/\!\gen{a}$ ist $b+\gen{a} \neq 1+\gen{a}$, da $b \not\in \gen{a}$, wegen $\ord G/\!\gen{a} = 2$ jedoch $b^2 + \gen{a} = (b+\gen{a})^2 = 1+\gen{a}$, also $b^2 \in \gen{a}$. Da $\ord b = 2$ oder $\ord b = 4$ ist $\ord b^2 = 1$ oder $\ord b^2 = 2$. Also muss $b^2 = 1$ oder $b^2 = a^2$. Es wird nun zwischen diesen beiden Fällen unterschieden:

Ist $b^2 = 1$, so ist die $G$ durch die Eigenschaften
\begin{equation}\label{eq: isomorphie1}
 \ord{a} = 4, \quad \ord{b} = 2, \quad G = \gen{a,b}, \quad ba = a^3 b
\end{equation}
bereits eindeutig charakterisiert: Aus diesen Bedingungen ergibt sich für $G$ durch direktes Ausrechnen die Verknüpfungstabelle
\begin{equation*}\renewcommand*{\arraystretch}{1.5}
 \begin{matrix}[|c|c|c|c|c|c|c|c|c|]\hline
      \cdot &     1 &     b & a     &   a b & a^2   & a^2 b & a^3   & a^3 b \\\hline
          1 &     1 &     b & a     &   a b & a^2   & a^2 b & a^3   & a^3 b \\\hline
          b &     b &     1 & a^3 b & a^3   & a^2 b & a^2   & a   b & a     \\\hline
      a     & a     & a   b & a^2   & a^2 b & a^3   & a^3 b &     1 &     b \\\hline
      a   b & a   b & a     &     b &     1 & a^3 b & a^3   & a^2 b & a^2   \\\hline
      a^2   & a^2   & a^2 b & a^3   & a^3 b &     1 &     b & a     & a   b \\\hline
      a^2 b & a^2 b & a^2   & a   b & a     &     b &     1 & a^3 b & a^3   \\\hline
      a^3   & a^3   & a^3 b &     1 &     b & a     & a   b & a^2   & a^2 b \\\hline
      a^3 b & a^3 b & a^3   & a^2 b & a^2   & a   b & a     &     b &     1 \\\hline
 \end{matrix}
\end{equation*}
Insbesondere ist $G$ durch diese Bedingungen bis auf Isomorphie eindeutig bestimmt, d.h. jede Gruppe $H$, in der es Element $g,h$ gibt, die \eqref{eq: isomorphie1} erfüllen, ist zu $G$ isomorph. Daraus ergibt sich, dass $G \cong D_4$: Für $\sigma, \tau \in D_4$ mit
\[
 \sigma = \vect{1 & 2 & 3 & 4} \text{ und } \tau = \vect{1 & 2}\vect{3 & 4}
\]
ist $\sigma^4 = \tau^2 = \id$, $D_4 = \gen{\sigma,\tau}$ sowie $\tau \sigma = \sigma^3 \tau$. Also ist $G \cong  D_4$.

Im Fall $b^2 = a^2$ geht man analog vor: Durch die Bedingungen
\begin{equation}\label{eq: isomorphie2}
 \ord a = 4,\quad b^2 = a^2,\quad G = \gen{a,b},\quad ba = a^3 b
\end{equation}
ergibt sich die Verknüpfungstabelle
\begin{equation*}\renewcommand*{\arraystretch}{1.5}
 \begin{matrix}[|c|c|c|c|c|c|c|c|c|]\hline
      \cdot &     1 &     b & a     &   a b & a^2   & a^2 b & a^3   & a^3 b \\\hline
          1 &     1 &     b & a     &   a b & a^2   & a^2 b & a^3   & a^3 b \\\hline
          b &     b & a^2   & a^3 b & a     & a^2 b &     1 & a   b & a^3   \\\hline
      a     & a     & a   b & a^2   & a^2 b & a^3   & a^3 b &     1 &     b \\\hline
      a   b & a   b & a^3   &     b & a^2   & a^3 b & a     & a^2 b &     1 \\\hline
      a^2   & a^2   & a^2 b & a^3   & a^3 b &     1 &     b & a     & a   b \\\hline
      a^2 b & a^2 b &     1 & a   b & a^3   &     b & a^2   & a^3 b & a     \\\hline
      a^3   & a^3   & a^3 b &     1 &     b & a     & a   b & a^2   & a^2 b \\\hline
      a^3 b & a^3 b & a     & a^2 b &     1 & a   b & a^3   &     b & a^2   \\\hline
 \end{matrix}
\end{equation*}
In diesem Fall ergibt sich, dass $G \cong Q_8$, wobei $Q_8$ die Quaternionengruppe bezeichnet (in der Form in der sie auf dem ersten Übungszettel angegeben war), da $I,J \in Q_8$ die Bedingungen \eqref{eq: isomorphie2} erfüllen: Es ist $\ord I = 4$, $I^2 = -E = J^2$ sowie $JI = -IJ = I^3J$. 





\section{}


\subsection{}\label{ssc: zykelordnung}
Es ist nach Definition
\begin{equation}\label{eq: orddef}
 \ord \pi = \min\{n \in \N, n \geq 1 : \pi^n = \id \}.
\end{equation}
Die $x_i$ sind paarweise verschieden, und es ist $\pi(x_i) = x_{i+1}$ für $i=1,\ldots,r-1$ und $\pi(x_r) = x_1$. Daher ist für $n=1,\ldots,r-1$
\[
 \pi^n(x_1) = x_{1+n} \neq x_1,
\]
also $\pi^n \neq \id$. Da allerdings für $i=1,\ldots,n$
\[
 \pi^r(x_i) = x_i 
\]
ist $\ord \pi = r$ nach \eqref{eq: orddef}. Analog ergibt sich, dass $\ord \tau = s$.

Da $\pi$ und $\tau$ fremd sind, kommutieren sie miteinander (aus der Vorlesung bekannt). Es kommutieren daher $\pi^n$ und $\tau^m$ ist daher für alle $n,m \in \N$, da
\begin{align*}
 \pi^n \tau
 = \prod_{i=1}^n \pi \cdot \tau
 = \prod_{i=1}^{n-1} \pi \cdot \tau \pi
 = \prod_{i=1}^{n-2} \pi \cdot \tau \pi^2
 = \ldots
 = \pi \tau \prod_{i=1}^{n-1} \pi
 = \tau \pi^n
\end{align*}
und deshalb
\begin{align*}
 \pi^n \tau^m
 &= \prod_{i=1}^n \pi \cdot \prod_{i=1}^m \tau
 = \tau \cdot \prod_{i=1}^n \pi \cdot \prod_{i=1}^{m-1} \tau
 = \tau^2 \cdot \prod_{i=1}^n \pi \cdot \prod_{i=1}^{m-2} \tau \\
 &= \ldots
 = \prod_{i=1}^{m-1} \tau \cdot \prod_{i=1}^n \pi \cdot \tau
 = \prod_{i=1}^m \tau \cdot \prod_{i=1}^n \pi
 = \tau^m \pi^n.
\end{align*}
Auch folgt aus der Fremdheit von $\pi$ und $\tau$, dass $\gen{\pi} \cap \gen{\tau} = 1$: Für $\sigma \in \gen{\pi} \cap \gen{\tau}$ ist $\pi^n = \sigma = \tau^m$ für passende $n,m \in \N$ mit $0 \leq n < r$ und $0 \leq m < s$. Es ist dann für $i=1,\ldots,r$
\[
 x_i = \pi^{r}(x_i) = \pi^{r-n}(\pi^n(x_i)) = \pi^{r-n}(\tau^m(x_i)) = \pi^{r-n}(x_i),
\]
weshalb $r-n$ ein Vielfaches von $\ord \pi = r$ sein muss; wegen $r-n \leq r$ muss also $r-n = r$ und daher $n=0$ und $\sigma = \pi^n = \id$.

Für alle $t \in \N, t \geq 1$ mit $(\pi \tau)^t = \id$ ist
\[
 \pi^t \tau^t = (\pi \tau)^t = \id,
\]
also $\pi^t = (\tau^t)^{-1} = \tau^{s-t} \in \gen{\tau}$. Wie oben bemerkt ist daher $\pi^t = \id$, also $t$ ein Vielfaches von $\ord \pi = r$. Analog ergibt sich, dass $t$ auch ein Vielfaches von $\ord \tau = s$ ist. Also ist $t \geq \kgV(r,s)$. Andererseits ist
\[
 (\pi \tau)^{\kgV(r,s)} = \pi^{\kgV(r,s)} \tau^{\kgV(r,s)} = \id^2 = \id.
\]
Also ist $\ord \pi\tau = \kgV(r,s)$.


\subsection{}
Es ist
\begin{align*}
 \sigma :&=
 \begin{pmatrix}[ccccccccccc]
  1 & 2 &  3 &  4 & 5 & 6 & 7 & 8 & 9 & 10 & 11\\
  4 & 1 & 10 & 11 & 8 & 9 & 7 & 2 & 3 &  6 & 5
 \end{pmatrix} \\
 &= \underbrace{\vect{1 & 4 & 11 & 5 & 8 & 2}}_{=: \pi} \underbrace{\vect{3 & 10 & 6 & 9}}_{=: \tau}.
\end{align*}
Nach Aufgabenteil \ref{ssc: zykelordnung} ist $\ord \pi = 6$ und $\ord \tau = 4$.
Da $\pi$ und $\tau$ fremde Zykeln sind, ist daher
\begin{align*}
 \sigma^{2013}
 &= (\pi \tau)^{2013}
 = \pi^{2013}\ \tau^{2013}
 = \pi^3 \tau \\
 &= \vect{1 & 4 & 11 & 5 & 8 & 2}^3 \vect{3 & 10 & 6 & 9} \\
 &= \vect{1 & 5} \vect{2 & 11} \vect{4 & 8} \vect{3 & 10 & 6 & 9} \\
 &=
 \begin{pmatrix}[ccccccccccc]
  1 &  2 &  3 & 4 & 5 & 6 & 7 & 8 & 9 & 10 & 11\\
  5 & 11 & 10 & 8 & 1 & 9 & 7 & 4 & 3 &  6 &  2
 \end{pmatrix}.
\end{align*}





\section{}
Da es in $\Sn_1$ keine Transpositionen gibt, wird im Folgenden der Fall $n \geq 2$ betrachtet.

Wie aus der Vorlesung bekannt bilden die Transpositionen ein Erzeugendensystem von $\Sn_n$. Es sind hierfür jedoch schon die Transpositionen $\tau_m$ der Form $\tau_m := \vect{1 & m}$ mit \mbox{$m \in \{2, \ldots, n\}$} ausreichend, da sich jede Transposition $(a,b) \in \Sn_n$ als
\[
 \vect{a & b} = \tau_a \tau_b \tau_a
\]
schreiben lässt. $\varphi$ ist also durch die Bilder dieser $n-1$ Transpositionen bereits eindeutig bestimmt.

Die $\tau_m$ kommutieren nicht miteinander, da für $m, m' \in \{2,\ldots,n\}$ mit $m \neq m'$
\[
 \tau_{m'} \tau_m = \vect{m & m' & 1} \neq \vect{m' & m & 1} = \tau_m \tau_{m'}.
\]
Daraus folgt, dass auch die Transpositionen der Form $\varphi(\tau_m)$ nicht miteinander kommutieren, also insbesondere nicht fremd zueinander sind: Gibt es $m, m' \in \{2, \ldots, n\}$ mit
\[
 \varphi(\tau_m)\varphi(\tau_{m'}) = \varphi(\tau_{m'})\varphi(\tau_m),
\]
so ist
\[
 \varphi(\tau_m \tau_{m'})
 = \varphi(\tau_m)\varphi(\tau_{m'})
 = \varphi(\tau_{m'})\varphi(\tau_m)
 = \varphi(\tau_{m'} \tau_m),
\]
wegen der Injektivität von $\varphi$ also $\tau_m \tau_{m'} = \tau_{m'} \tau_m$ und daher $m = m'$.

Für $m \in \N$ seien $a_m, b_m \in \{1,\ldots,n\}$ so dass $\varphi(\tau_m) = \vect{a_m & b_m}$. Da die $\tau_m$ paarweise verschieden sowie nicht fremd sind, gibt es wegen der Injektivität von $\varphi$ für alle $m, m' \in \{2, \ldots, m\}$ genau ein $i \in \{a_m, b_m\}$ und genau ein $j \in \{a_{m'}, b_{m'}\}$ mit $i = j$.

\begin{beh}\label{beh: transpositionen gemeinsames element}
 Es ist $\bigcap_{m=2}^n \{a_m, b_m\} \neq \emptyset$.
\end{beh}
\begin{proof}
 Für $n \in \{2,3\}$ ist nichts zu zeigen. Sei daher $n \geq 4$. Angenommen, die Behauptung gilt nicht. Sei $m_0 := \min \left\{k \in \{2,\ldots,n\right\} : \bigcap_{m=2}^k \{a_m,b_m\}\}$. Da $\tau_2$ und $\tau_3$ nicht fremd sind, ist $m_0 \geq 4$. Wegen der Minimalität von $m_0$ gibt es ein \mbox{$a \in \{1,\ldots,n\}$}, so dass $a \in \{a_m,b_m\}$ für alle $m \in \{2,\ldots,m_0-1\}$. Da die $\tau_m$ paarweise verschieden sind gibt es auch $c_2, \ldots, c_{m_0-1} \in \{1,\ldots,n\}$ mit $\{a_m, b_m\} = \{a, c_m\}$ für alle $m \in \{2, \ldots, m_0 -1\}$. Nach Definition von $m_0$ muss $a \not \in \{a_{m_0}, b_{m_0}\}$. Da jedoch $\tau_{m_0}$ nicht fremd zu $\tau_2$ und $\tau_3$ ist, muss $c_2, c_3 \in \{a_{m_0},b_{m_0}\}$, da $c_2$ und $c_3$ verschieden sind, ist also $\{a_{m_0}, b_{m_0}\} = \{c_2, c_3\}$.
 Es ist daher
 \begin{align*}
  \varphi(\tau_{m_0})
  &= \vect{a_{m_0} &  b_{m_0}} = \vect{c_2 & c_3} = \vect{a & c_2} \vect{a & c_3} \vect{a & c_2} \\
  &= \varphi(\tau_2)\varphi(\tau_3)\varphi(\tau_2) = \varphi(\tau_2 \tau_3 \tau_2),
 \end{align*}
 wegen der Injektivität von $\varphi$ also
 \[
  \vect{1 & m_0} = \tau_{m_0} = \tau_2 \tau_3 \tau_2 = \vect{1 & 2} \vect{1 & 3} \vect{1 & 2} = \vect{2 & 3}.
 \]
 Dies ist offenbar ein Widerspruch.
\end{proof}

Seien $c_1, \ldots, c_n \in \{1,\ldots,n\}$ paarweise verschieden, so dass $\{a_m, b_m\} = \{c_1, c_m\}$ für alle $m \in \{2,\ldots,m\}$; die Existenz entsprechender Elemente folgt aus der Behauptung \ref{beh: transpositionen gemeinsames element} und der Fremdheit der $\tau_m$. $\pi \in \Sn_n$ sei definiert als
\[
 \pi(c_1) := 1 \text{ und } \pi(c_m) := m \text{ für alle } m \in \{2,\ldots,n\}.
\]

Durch direktes Nachrechen ergibt sich nun, dass $\varphi = \inn_\pi$, also $\varphi(x) = \pi^{-1} \cdot x \cdot \pi$ für alle $x \in \Sn_n$: Wie zu Beginn bemerkt genügt es dies für die $\tau_m$ zu zeigen. Da für alle $m \in \{2, \ldots, n\}$
\[
 \varphi(\vect{1 & m}) = \vect{c_1 & c_m} = \pi^{-1} \vect{1 & m} \pi
\]
ist dies der Fall. Es ist also $\varphi = \inn_\pi$.





\section{}
Sei $n \in \N$ beliebig aber fest. Da $[G,G]$ eine Untergruppe von $G$ ist, ist $1 \in [G,G]$, also $1 \in G_n$, da $1^n = 1 \in [G,G]$. Für alle $g \in G_n$ ist wegen $g^n \in [G,G]$ auch $(g^{-1})^n = (g^n)^{-1} \in [G,G]$, also $g^{-1} \in G_n$. Dass für $g,h \in G_n$ auch $gh \in G_n$ ergibt sich mithilfe der folgenden Bemerkung.

\begin{bem}
 Sei $G$ eine Gruppe und seien $g,h \in G$. Dann gibt es für alle $n \in \N$ ein $c \in [G,G]$ mit
 \[
  (gh)^n = g^n h^n c.
 \]
\end{bem}
\begin{proof}
 Der Beweis verläuft per Induktion über $n$.
 \begin{ia}
  Sei $n=0$. Dann ist
  \[
   (gh)^n = (gh)^0 = 1 = 1 \cdot 1 \cdot 1 = g^0 h^0 \cdot 1 = g^n h^n \cdot 1.
  \]
 \end{ia}
 \begin{is}
  Sei $n \geq 1$ und gelte die Aussage für $n-1$. Nach Induktionsvoraussetzung gibt es ein $c \in [G,G]$ mit $(gh)^{n-1} = g^{n-1} h^{n-1} c$. Es ist daher
  \begin{align*}
   (gh)^n
   &= gh (gh)^{n-1}
   = gh g^{n-1} h^{n-1} c \\
   &= g^n h \left[h^{-1},g^{1-n}\right] h^{n-1} c
   = g^n h^n \left[h^{-1},g^{1-n}\right] \left[\left[h^{-1},g^{1-n}\right]^{-1}, h^{1-n}\right] c.
  \end{align*}
  Da $[G,G]$ eine Untergruppe von $G$ ist, ist
  \[
   \left[h^{-1},g^{1-n}\right] \left[\left[h^{-1},g^{1-n}\right]^{-1}, h^{1-n}\right] c \in [G,G]. \qedhere
  \]
 \end{is}
\end{proof}

Da $[G,G]$ ein Untergruppe von $G$ ist, und $g^n, h^n \in [G,G]$, ist für $c \in [G,G]$ mit $(gh)^n = g^n h^n c$ auch $(gh)^n = g^n h^n c \in [G,G]$.

Es gilt noch zu zeigen, dass $G_n$ normal in $G$ ist, dass also für $g \in G_n$ und $h \in G$ auch $hgh^{-1} \in G_n$. Da $g^n \in [G,G]$ und $[G,G]$ normal in $G$ ist, gilt
\[
 (hgh^{-1})^n = h(g^n)h^{-1} \in [G,G],
\]
also auch $hgh^{-1} \in G_n$.





\section{}
Da $\ord\,[G,G] = 2$ ist $[G,G] = \{1, \sigma\}$ für ein selbstinverses Element $\sigma \in G$. $G$ ist nicht abelsch, denn sonst wäre $[G,G] = 1$. $G$ ist insbesondere nichttrivial.

Für alle $g \in G$ ist $g^2 \in Z$, wobei $Z$ das Zentrum von $G$ bezeichnet: Es ist für alle $h \in G$
\begin{equation}\label{eq: g und h kommutieren}
 \begin{aligned}
  g^2 h
  &= g h g \left[g^{-1},h^{-1}\right]
  = h g \left[g^{-1},h^{-1}\right] g \left[g^{-1},h^{-1}\right] \\
  &= h g \left[g^{-1},h^{-1}\right] \left[h^{-1},g\right] g,
 \end{aligned}
\end{equation}
da
\[
 g \left[g^{-1},h^{-1}\right]
 = g g^{-1} h^{-1} g h = h^{-1} g h = h^{-1} g h g^{-1} g = \left[h^{-1},g\right] g.
\]
Es ist nun
\[
 \left[g^{-1},h^{-1}\right] = 1
 \Leftrightarrow g^{-1}  \in Z_{\{h^{-1}\}}
 \Leftrightarrow g \in Z_{\{h^{-1}\}}
 \Leftrightarrow \left[h^{-1},g\right] = 1,
\]
da $Z_{\{h^{-1}\}}$ eine Untergruppe von $G$ ist. Da $\ord\,[G,G] = 2$ und $\left[g^{-1},h^{-1}\right], \left[h^{-1},g\right] \in [G,G]$ folgt daraus, dass $\left[g^{-1},h^{-1}\right] = \left[h^{-1},g\right]$, und da jedes Element in $[G,G]$ selbstinvers ist, daher auch $\left[g^{-1},h^{-1}\right] \left[h^{-1},g\right] = 1$. Aus \eqref{eq: g und h kommutieren} folgt daher, dass $g^2h = hg^2$. Aus der Beliebigkeit von $h \in G$ folgt damit $g^2 \in Z$.

Da $Z = \Ker \inn$ bedeutet dies, dass $\inn_g^2 = \inn_{g^2} = \id$ für alle $g \in G$, dass also alle $\varphi \in \Inn(G)$ selbstinvers sind. Dies hat zwei wichtige Konsequenzen: Zum einen folgt aus der folgenden Bemerkung, dass $\ord \Inn(G)$ gerade ist.

\begin{bem}
 Sei $G$ eine nichttriviale Gruppe, so dass alle $g \in G$ selbstinvers sind. Dann ist $\ord G$ gerade.
\end{bem}
\begin{proof}
 Da $G$ nichttrivial ist, gibt es ein $g \in G-1$. Da $g \neq 1$ selbstinvers ist, ist $\ord g = 2$. Da $\ord g$ ein Teiler von $\ord G$ ist, ist $\ord G$ gerade.
\end{proof}

Dass $\Inn(G)$ nichttrival ist, ergibt sich daraus, dass $\Inn(G) \cong G/Z$. Wäre $\Inn(G)$ trivial, so wäre $G = Z$, also $G$ abelsch.

Zum anderen folgt, da jedes $\varphi \in \Inn(G)$ selbstinvers ist, dass $\Inn(G) \cong G/Z$ abelsch ist. Wegen der entsprechenden Minimalitätseigenschaft von $[G,G]$ folgt daraus, dass $[G,G] \subseteq Z$ eine Untergruppe ist. Da $[G,G]$ normal in $G$ ist, ist $[G,G]$ auch normal $Z$. (Dies folgt auch aus der Kommutativität von $Z$.)

Aus dem zweiten Isomorphiesatz folgt nun, dass
\[
 G/Z \cong (G/[G,G])/(Z/[G,G]).
\]
Insbesondere ist
\[
 \ord G/Z = \frac{\ord G/[G,G]}{\ord Z/[G,G]}.
\]
Da $\ord G/Z = \ord \Inn(G)$ gerade ist, ist also auch $(G : [G,G]) = \ord G/[G,G]$ gerade.












\end{document}
