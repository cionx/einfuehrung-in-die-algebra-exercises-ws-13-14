\documentclass[a4paper,10pt]{article}
%\documentclass[a4paper,10pt]{scrartcl}

\usepackage{xltxtra}
\usepackage[ngerman]{babel}
\usepackage{amsmath}
\usepackage{amssymb}
\usepackage{amsthm}
\usepackage{mathtools}
\usepackage{nicefrac}
\usepackage{enumerate}
\usepackage{leftidx}

\theoremstyle{definition}
\newtheorem{beh}{Behauptung}
\newtheorem{bem}[beh]{Bemerkung}
\newtheorem{lem}[beh]{Lemma}
\newtheorem*{anm}{Anmerkung}
\newtheorem*{ia}{Induktionsanfang}
\newtheorem*{is}{Induktionsschritt}

\renewcommand{\thesection}{Aufgabe 4.\arabic{section}.}
\renewcommand{\thesubsection}{(\roman{subsection})}
\renewcommand{\thesubsubsection}{}

\newcommand{\N}{\mathbb{N}}
\newcommand{\Z}{\mathbb{Z}}
\newcommand{\Q}{\mathbb{Q}}
\newcommand{\R}{\mathbb{R}}
\newcommand{\C}{\mathbb{C}}
\newcommand{\Sn}{\mathfrak{S}}
\newcommand{\dt}{\,\text{d}t}
\newcommand{\F}[1]{\mathbb{F}_{#1}}
\newcommand{\id}{\operatorname{id}}
\newcommand{\ord}{\operatorname{ord}}
\newcommand{\inn}{\operatorname{inn}}
\newcommand{\sgn}{\operatorname{sgn}}
\newcommand{\kchar}{\operatorname{char}}
\newcommand{\kgV}{\operatorname{kgV}}
\newcommand{\Id}{\operatorname{Id}}
\newcommand{\GL}[2]{\operatorname{GL}(#1,#2)}
\newcommand{\Img}{\operatorname{Im}}
\newcommand{\Ker}{\operatorname{Ker}}
\newcommand{\Hom}{\operatorname{Hom}}
\newcommand{\End}{\operatorname{End}}
\newcommand{\vect}[1]{\begin{pmatrix}#1\end{pmatrix}}
\newcommand{\gen}[1]{\left\langle#1\right\rangle}

\newenvironment{lgs}[1][c]{\left\{\setlength{\arraycolsep}{1pt}\begin{array}{#1}}{\end{array}\right.}

\makeatletter
\renewcommand*\env@matrix[1][*\c@MaxMatrixCols c]{%
  \hskip -\arraycolsep
  \let\@ifnextchar\new@ifnextchar
  \array{#1}}
\makeatother

\setromanfont[Mapping=tex-text]{Linux Libertine O}
% \setsansfont[Mapping=tex-text]{DejaVu Sans}
% \setmonofont[Mapping=tex-text]{DejaVu Sans Mono}
\parindent0pt

\title{\textsc{Einführung in die Algebra \\ \Large Blatt 4}}
\author{Jendrik Stelzner}
\date{\today}

\begin{document}
\maketitle





\section{}





\section{}


\subsection{}\label{ssc: zykelordnung}
Es ist nach Definition
\begin{equation}\label{eq: orddef}
 \ord \pi = \min\{n \in \N, n \geq 1 : \pi^n = \id \}.
\end{equation}
Nun sind die $x_i$ paarweise verschieden, und $\pi(x_i) = x_{i+1}$ für $i=1,\ldots,r-1$ und $\pi(x_r) = x_1$. Daher ist für $n=1,\ldots,r-1$
\[
 \pi^n(x_1) = x_{1+n} \neq x_1,
\]
also $\pi^n \neq \id$. Da allerdings für $i=1,\ldots,n$
\[
 \pi^r(x_i) = x_i 
\]
ist $\ord \pi = r$ nach \eqref{eq: orddef}. Analog ergibt sich, dass $\ord \tau = s$.

Da $\pi$ und $\tau$ fremd sind, kommutieren sie miteinander (aus der Vorlesung bekannt). Es kommutieren daher $\pi^n$ und $\tau^m$ ist daher für alle $n,m \in \N$, da
\begin{align*}
 \pi^n \tau^m
 &= \prod_{i=1}^n \pi \cdot \prod_{i=1}^m \tau
 = \tau \cdot \prod_{i=1}^n \pi \cdot \prod_{i=1}^{m-1} \tau
 = \tau^2 \cdot \prod_{i=1}^n \pi \cdot \prod_{i=1}^{m-2} \tau \\
 &= \ldots
 = \prod_{i=1}^{m-1} \tau \cdot \prod_{i=1}^n \pi \cdot \tau
 = \prod_{i=1}^m \tau \cdot \prod_{i=1}^n \pi
 = \tau^m \pi^n.
\end{align*}
Auch folgt aus der Fremdheit von $\pi$ und $\tau$, dass $\gen{\pi} \cap \gen{\tau} = \{1\}$: Für $\sigma \in \gen{\pi} \cap \gen{\tau}$ ist $\pi^n = \sigma = \tau^m$ für passende $n,m \in \N$ mit $0 \leq n \leq r-1$ und $0 \leq m \leq s-1$. Es ist dann für $i=1,\ldots,r$
\[
 x_i = \pi^{r}(x_i) = \pi^{r-n}(\pi^n(x_i)) = \pi^{r-n}(\tau^m(x_i)) = \pi^{r-n}(x_i),
\]
weshalb $r-n$ ein Teiler von $r$ sein muss; wegen $r-n \leq r$ muss also $r-n = r$ und daher $n=0$ und $\sigma = \pi^n = \id$.

Für alle $t \in \N, t \geq 1$ mit $(\pi \tau)^t = \id$ ist
\[
 \pi^t \tau^t = (\pi \tau)^t = \id,
\]
also $\pi^t = (\tau^t)^{-1} = \tau^{s-t} \in \gen{\tau}$. Wie oben bemerkt ist daher $\pi^t = \id$, also $t$ ein Vielfaches von $\ord \pi = r$. Analog ergibt sich, dass $t$ auch ein Vielfaches von $\ord \tau = s$ ist. Also ist $t \geq \kgV(r,s)$. Andererseits ist
\[
 (\pi \tau)^{\kgV(r,s)} = \pi^{\kgV(r,s)} \tau^{\kgV(r,s)} = \id^2 = \id.
\]
Also ist $\ord \pi\tau = \kgV(r,s)$.






\subsection{}
Es ist
\begin{align*}
 \sigma :&=
 \begin{pmatrix}[ccccccccccc]
  1 & 2 &  3 &  4 & 5 & 6 & 7 & 8 & 9 & 10 & 11\\
  4 & 1 & 10 & 11 & 8 & 9 & 7 & 2 & 3 &  6 & 5
 \end{pmatrix} \\
 &= \underbrace{\vect{1 & 4 & 11 & 5 & 8 & 2}}_{=: \pi} \underbrace{\vect{3 & 10 & 6 & 9}}_{=: \tau}.
\end{align*}
Nach Aufgabenteil \ref{ssc: zykelordnung} ist $\ord \pi = 6$ und $\ord \tau = 4$.
Da $\pi$ und $\tau$ fremde Zykeln sind ist daher
\begin{align*}
 \sigma^{2013}
 &= (\pi \tau)^{2013}
 = \pi^{2013}\ \tau^{2013}
 = \pi^3 \tau \\
 &= \vect{1 & 4 & 11 & 5 & 8 & 2}^3 \vect{3 & 10 & 6 & 9} \\
 &= \vect{1 & 5} \vect{2 & 11} \vect{4 & 8} \vect{3 & 10 & 6 & 9} \\
 &=
 \begin{pmatrix}[ccccccccccc]
  1 &  2 &  3 & 4 & 5 & 6 & 7 & 8 & 9 & 10 & 11\\
  5 & 11 & 10 & 8 & 1 & 9 & 7 & 4 & 3 &  6 &  2
 \end{pmatrix}.
\end{align*}








\section{}





\section{}





\section{}






\end{document}
