\documentclass[a4paper,10pt]{article}
%\documentclass[a4paper,10pt]{scrartcl}

\usepackage{xltxtra}
\usepackage[ngerman]{babel}
\usepackage{amsmath}
\usepackage{amssymb}
\usepackage{amsthm}
\usepackage{mathtools}
\usepackage{nicefrac}
\usepackage{enumerate}
\usepackage{leftidx}

\theoremstyle{definition}
\newtheorem{beh}{Behauptung}
\newtheorem{bem}[beh]{Bemerkung}
\newtheorem{lem}[beh]{Lemma}
\newtheorem*{anm}{Anmerkung}
\newtheorem*{ia}{Induktionsanfang}
\newtheorem*{is}{Induktionsschritt}

\renewcommand{\thesection}{Aufgabe 5.\arabic{section}.}
\renewcommand{\thesubsection}{(\roman{subsection})}
\renewcommand{\thesubsubsection}{}

\newcommand{\N}{\mathbb{N}}
\newcommand{\Z}{\mathbb{Z}}
\newcommand{\Q}{\mathbb{Q}}
\newcommand{\R}{\mathbb{R}}
\newcommand{\C}{\mathbb{C}}
\newcommand{\Sn}{\mathfrak{S}}
\newcommand{\mf}[1]{\mathfrak{#1}}
\newcommand{\dt}{\,\text{d}t}
\newcommand{\F}[1]{\mathbb{F}_{#1}}
\newcommand{\id}{\operatorname{id}}
\newcommand{\ord}{\operatorname{ord}}
\newcommand{\inn}{\operatorname{inn}}
\newcommand{\sgn}{\operatorname{sgn}}
\newcommand{\kchar}{\operatorname{char}}
\newcommand{\kgV}{\operatorname{kgV}}
\newcommand{\Id}{\operatorname{Id}}
\newcommand{\GL}[2]{\operatorname{GL}(#1,#2)}
\newcommand{\Img}{\operatorname{Im}}
\newcommand{\Ker}{\operatorname{Ker}}
\newcommand{\Hom}{\operatorname{Hom}}
\newcommand{\End}{\operatorname{End}}
\newcommand{\Aut}{\operatorname{Aut}}
\newcommand{\Inn}{\operatorname{Inn}}
\newcommand{\vect}[1]{\begin{pmatrix}#1\end{pmatrix}}
\newcommand{\gen}[1]{\left\langle#1\right\rangle}

\newenvironment{lgs}[1][c]{\left\{\setlength{\arraycolsep}{1pt}\begin{array}{#1}}{\end{array}\right.}

\renewcommand*{\arraystretch}{1.5}

\makeatletter
\renewcommand*\env@matrix[1][*\c@MaxMatrixCols c]{%
  \hskip -\arraycolsep
  \let\@ifnextchar\new@ifnextchar
  \array{#1}}
\makeatother

\setromanfont[Mapping=tex-text]{Linux Libertine O}
% \setsansfont[Mapping=tex-text]{DejaVu Sans}
% \setmonofont[Mapping=tex-text]{DejaVu Sans Mono}
\parindent0pt

\title{\textsc{Einführung in die Algebra \\ \Large Blatt 5}}
\author{Jendrik Stelzner}
\date{\today}

\begin{document}
\maketitle





\section{}


\subsection{}
Nach Definition von $N_H$ ist $gH = Hg$ für alle $g \in N_H$. Da $x \in N_H$, ist $\gen{x} \subseteq N_H$ eine Untergruppe, und insbesondere $\gen{x}H = H\gen{x}$.
Es ist
\[
 1 = 1 \cdot 1 \in \gen{x}H,
\]
und für $a,b \in \gen{x}$ mit $a = x^n h$ und $b = x^m \tilde{h}$ ist
\[
 ab^{-1} = x^n h \tilde{h}^{-1} x^{-m} \in \gen{x} H \gen{x} = \gen{x} \gen{x} H = \gen{x} H.
\]
Da $\gen{x}, H \subseteq N_H$ ist $\gen{x}H$ eine Untergruppe von $N_H$, also insbesondere von $G$.


\subsection{}
Angenommen, es ist $N_H \neq H$. Dann gibt es ein $x \in N_H$ mit $x \not\in H$. Wie oben gezeigt ist $\gen{x}H$ eine Untergruppe von $N_H$. Offenbar ist $H \subsetneq \gen{x}H$ eine echte Untergruppe, und da $H$ normal in $N_H$ ist, ist $H$ auch normal in $\gen{x}H$. Auch ist
\[
 \gen{x}\!H/H \cong \gen{x}/H \cap \gen{x}
\]
zyklisch, da $\gen{x}$ zyklisch ist, und somit insbesondere abelsch. Da $H$ auflösbar ist, gibt es eine Normalreihe
\[
 1 = H_0 \subsetneq H_1 \subsetneq \ldots \subsetneq H_n = H
\]
mit abelschen Faktoren. Es folgt nun, dass
\[
 1 = H_0 \subsetneq H_1 \subsetneq \ldots \subsetneq H_n \subsetneq H_{n+1} = \gen{x}H
\]
eine Normalreihe von $\gen{x}H$ mit abelschen Faktoren ist. Das steht aber im Widerspruch zur Maximalität von $H$, da $H$ eine echte Untergruppe von $\gen{x}H$ ist. Also ist bereits $N_H = H$.


\section{}

\subsection{}

\begin{bem}\label{bem: index A_n}
 Sei $n \geq 2$. Dann ist $(\Sn_n : \mf{A}_n) = 2$. Insbesondere ist $\mf{A}_n$ normal in $\Sn_n$ (dies folgt auch aus $\mf{A}_n = \Ker \sgn$).
\end{bem}
\begin{proof}
 Sei $\tau = \vect{1 & 2} \in \Sn_n$ und $\varphi$ die Linkstranslation mit $\tau$. Aufgrund der Injektivität von $\varphi$ induziert $\varphi$ eine injektive Abbildung von der Menge aller gerader Permutation $\mf{A}$ in die Menge aller ungerader Permutationen $\Sn_n - \mf{A}$, sowie auch eine injektive Abbildung von $\Sn_n-\mf{A}$ nach $\mf{A}$. Es ist daher
 \[
  \ord \mf{A}_n \leq |\Sn_n - \mf{A}_n| \leq \ord \mf{A}_n,
 \]
 also $\ord \mf{A}_n = |\Sn_n - \mf{A}_n|$ und somit $\ord \Sn_n = 2 \ord \mf{A}_n$.
\end{proof}



Sei $\sigma \in H$ eine ungerade Permutation.
Es ist $H \mf{A}_n = \Sn_n$: Da $\mf{A}_n \subseteq H \mf{A}_n$ enthält $H\mf{A}$ alle geraden Permutationen. Jede ungerade Permutation $\pi \in \Sn_n$ lässt sich als
\[
 \pi = \sigma \cdot \sigma \pi
\]
schreiben, wobei $\sigma \in H$ und $\sigma \pi$ als Produkt zweier ungerader Permutationen gerade ist, also in $\mf{A}_n$ ist. Also ist $\pi \in H \mf{A}_n$.

Nach Bemerkung \ref{bem: index A_n} ist $\mf{A}_n$ normal in $\Sn_n$ mit $(\Sn_n : \mf{A}_n) = 2$. Also ist $\mf{A}_n \cap H$ normal in $H$ mit
\[
 H / \mf{A}_n \cap H \cong H\mf{A}_n / \mf{A_n} = \Sn_n / \mf{A}_n.
\]
Insbesondere ist daher
\[
 (H : \mf{A} \cap H) = \ord H / \mf{A}_n \cap H = \ord \Sn_n / \mf{A}_n = (\Sn_n : \mf{A}_n) = 2.
\]


\subsection{}
Da $\ord H > 2$ enthält $H$ eine $\pi \neq \id$ gerader Ordnung: Da $H$ nichttrivial ist, gibt es ein $\sigma \in H$ mit $\sigma \neq \id$. Ist $\sigma$ gerade, so sei $\pi := \sigma$. Ist $\sigma$ ungerade so wird zwischen zwei Fällen unterschieden: Ist $\sigma$ nicht selbstinvers, so sei $\pi := \sigma^2$. Ist $\sigma$ selbstinvers, so muss $H$ wegen $\ord H > 2$ noch ein weiteres Element $\tau \in H-\{\id,\sigma\}$ beinhalten. Wiederholt man die oberen Schritte für $\tau$, so findet man entweder ein entsprechendes Element $\pi$ oder auch $\tau$ ist selbstinvers. Sind $\sigma$ und $\tau$ selbstinvers, so sei $\pi := \sigma \tau$.

Es folgt, dass $H \cap \mf{A}_n \supseteq \{\id, \pi\}$ nichttrivial ist. Da $\mf{A}_n$ normal in $\Sn_n$ ist, ist $H \cap \mf{A}_n$ normal in $H$. Da $H$ einfach ist, folgt, dass $H \cap \mf{A}_n = H$ ist. Also ist $H \subseteq \mf{A}_n$ eine Untergruppe.





\section{}

\begin{bem}\label{bem: 0 neq 1}
 Sei $R$ ein Ring mit mindestens zwei Elementen. Dann ist sind Null- und Einselement in $R$ verschieden.
\end{bem}
\begin{proof}
 Da $R$ mindestens zwei Elemente besitzt, gibt es ein $a \in R$ mit $a \neq 0$. Es ist
 \[
  1 \cdot a = a \neq 0 = 0 \cdot a,
 \]
 also $0 \neq 1$.
\end{proof}

\begin{bem}\label{bem: 1 ohne nullteiler eindeutig}
 Sei $R$ ein Integritätsring. Gibt es für $b \in R$ ein $a \in R$ mit $a \neq 0$ und $ab = a$ oder $ba = a$, so ist bereits $b = 1$. Insbesondere gilt für jede Ringerweiterung $R' \subseteq R$ mit $R' \neq 0$, dass $R'$ genau dann ein Einselement beinhaltet, wenn $1 \in R'$.
 \begin{proof}
  Da $a \neq 0$ impliziert die Nullteilerfreiheit von $R$ direkt die Injektivität der Links-, bzw. Rechtsmultiplikation mit $a$. Da $1 \cdot a = a = a \cdot 1$ ist daher $b = 1$.
 \end{proof}
\end{bem}


Nach Aufgabenstellung ist $R$ ein kommutativer Ring mit Einselement. Da $R$ mindestens zwei Elemente besitzt folgt aus Bemerkung \ref{bem: 0 neq 1}, dass $0 \neq 1$. Es gilt also nur noch zu zeigen, dass es für jedes $a \in R$ mit $a \neq 0$ ein multiplikativ Inverses $b \in R$ mit $ab = 1$ gibt.

Sei $a \in R$ mit $a \neq 0$ beliebig aber fest und $\mf{a} := (a)$ das von $a$ erzeugte Ideal in $R$. Da $a \in \mf{a}$ ist $\mf{a} \neq 0$, und es gilt bereits $\mf{a} = R$: Als Ideal ist $\mf{a}$ eine additive Untergruppe von $R$ sowie unter der Multiplikation abgeschlossen, wobei sich Assoziativität, Kommutativität und Distributivität der Multiplikation von $R$ auf $\mf{a}$ vererben. Aus der entsprechenden Eigenschaft von $R$ folgt, dass $\mf{a}$ ein Ring mit Einselement bildet. Aus Bemerkung \ref{bem: 1 ohne nullteiler eindeutig} folgt damit, dass $1 \in \mf{a}$, und daher bereits $\mf{a} = R$. Da $aR = \mf{a} = R$ gibt es insbesondere ein $b \in R$ mit $ab = 1$.




\section{}


\addtocounter{subsection}{1}
\subsection{}
Für alle $a \in R$ ist
\[
 a^2 + 1 = a+1 = (a+1)^2 = a^2 + 2a + 1,
\]
also $2a = 0$. Insbesondere ist $a = -a$.


\addtocounter{subsection}{-2}
\subsection{}
Für alle $a,b \in R$ ist
\[
 ab-ba = ab + ba = (a+b)^2 - a^2 - b^2 = a+b-a-b = 0,
\]
also $ab = ba$, und daher $R$ kommutativ.


\addtocounter{subsection}{1}
\subsection{}
Seien $a,b \in R$ mit $a \neq b$. Es ist
\[
 (a-b)ab = a^2 b - ab^2 = ab-ab = 0.
\]
Da $a \neq b$ ist $a-b \neq 0$, wegen der Nullteilerfreiheit von $R$ also $ab = 0$. Wegen der Nullteilerfreiheit ist also $a = 0$ oder $b = 0$. Aus der Beliebigkeit von $a$ und $b$ folgt, dass es neben $0$ nur ein weiters Element in $R$ gibt. Da aus Bemerkung \ref{bem: 0 neq 1} folgt, dass $0 \neq 1$, ist also $R = \{0,1\}$.
Betrachtet man die Verknüpfungstabellen von $R$,
\begin{equation*}
 \begin{matrix}[|c|c|c|]\hline
  + & 0 & 1 \\\hline
  0 & 0 & 1 \\\hline
  1 & 1 & 0 \\\hline
 \end{matrix}
 \quad
 \text{ und }
 \quad
 \begin{matrix}[|c|c|c|]\hline
  \cdot & 0 & 1 \\\hline
      0 & 0 & 0 \\\hline
      1 & 0 & 1 \\\hline
 \end{matrix}\quad,
\end{equation*}
so ist $R$ offenbar isomorph zu $\F{2}$.





\section{}


\subsection{}
Es bezeichne
\[
 \mf{a}[X] := \left\{f \in R[X] : f = \sum_{i=0}^n a_i X^i \text{ mit } n \geq 0, a_i \in \mf{a} \text{ für alle } i\right\}
\]
die Menge aller Polynome in $R[X]$ mit Koeffizienten in $\mf{a}$. Da $\mf{a}$ als Ideal abgeschlossen unter Addition ist, ist es auch $\mf{a}[X]$. Das von $\mf{a}$ in $R[X]$ erzeugte Ideal $\mf{b}$ hat die Form
\begin{align*}
 \mf{b} = \sum_{a \in \mf{a}} aR[X] = \sum_{a \in \mf{a}} \{af : f \in R[X] \}
\end{align*}

Sei $f \in \mf{a}[X]$. Dann hat $f$ die Form $f = \sum_{i=0}^n a_i X^i$ mit $a_i \in \mf{a}$ für alle $i$. Es ist $a_i X^i \in a_i R[X]$ für alle $i$, und daher $f \in \sum_{i=0}^n a_i R[X] \subseteq \mf{b}$. Also ist $\mf{a}[X] \subseteq \mf{b}$.

Sei $a \in \mf{a}$ und $f \in aR[X]$. Dann hat $f$ die Form $f = \sum_{i=0}^n (a a_i) X^i$ mit $a_i \in R$ für alle $i$. Da $a \in \mf{a}$ und $\mf{a}$ ein Ideal ist, ist $a a_i \in \mf{a}$ für alle $i$. Es ist daher $f \in \mf{a}[X]$. Also ist $aR[X] \subseteq \mf{a}[X]$ für alle $a \in \mf{a}$. Da $\mf{a}[X]$ abgeschlossen unter Addition ist, ist daher auch $\mf{b} = \sum_{a \in \mf{a}} aR[X] \subseteq \mf{a}[X]$.


\subsection{}

\begin{lem} \label{lem: ringhomo polynom}
 Seien $R, R'$ Ringe und $\phi: R \rightarrow R'$ ein Ringhomomorphismus. Dann induziert $\phi$ einen Ringhomomorphismus $\psi: R[X] \rightarrow R'[X]$ mit
  \[
   \psi\left( \sum_{i=0}^n a_i X^i \right) := \sum_{i=0}^n \phi(a_i) X^i.
  \]
  Dabei ist
  \[
   \Ker \psi = \left\{f \in R[X] : f = \sum_{i=0}^n a_i X^i \text{ mit } n \geq 0 \text{ und } a_i \in \Ker \phi \text{ für alle }i\right\}
  \]
  und
  \[
   \Img \psi = \left\{g \in R'[X] : g = \sum_{i=0}^n b_i X^i \text{ mit } n \geq 0 \text{ und } b_i \in \Img \phi \text{ für alle }i\right\}.
  \]
  Insbesondere ist $\psi$ genau dann injektiv, wenn $\phi$ injektiv ist, und $\psi$ genau dann surjektiv, wenn $\phi$ surjektiv ist.
\end{lem}
\begin{proof}
 Es gilt zunächst zu zeigen, dass $\psi$ ein Ringhomomorphismus ist. Es seien $f,g \in R[X]$ mit $f=\sum_{i=0}^n a_i X^i$ und $g=\sum_{i=0}^n b_i X^i$. Es ist
 \begin{align*}
  \psi(f+g)
  &= \psi\left( \sum_{i=0}^n (a_i+b_i) X^i \right)
  = \sum_{i=0}^n \phi(a_i+b_i) X^i \\
  &= \sum_{i=0}^n (\phi(a_i) + \phi(b_i)) X^i
  = \sum_{i=0}^n \phi(a_i) X^i  +  \sum_{i=0}^n \phi(b_i) X^i \\
  &= \psi\left( \sum_{i=0}^n a_i X^i \right) + \psi\left( \sum_{i=0}^n b_i X^i \right)
  = \psi(f) + \psi(g),
 \end{align*}
 sowie
 \begin{align*}
  \psi(f g)
  &= \psi\left( \sum_{i=0}^{2n} \left(\sum_{\mu+\nu=i} a_\mu b_\nu \right) X^i \right)
  = \sum_{i=0}^{2n} \phi \left(\sum_{\mu+\nu=i} a_\mu b_\nu \right) X^i \\
  &= \sum_{i=0}^{2n} \left(\sum_{\mu+\nu=i} \phi(a_\mu) \phi(b_\nu) \right) X^i
  = \left( \sum_{i=0}^n \phi(a_i) X^i \right)\left( \sum_{i=0}^n \phi(b_i) X^i \right)\\
  &= \psi \left( \sum_{i=0}^n a_i X^i \right) \psi \left( \sum_{i=0}^n b_i X^i \right)
  = \psi(f)\ \psi(g).
 \end{align*}
 $\psi$ ist auch unitär, da
 \[
  \psi(1) = \psi(1 \cdot X^0) = \phi(1) \cdot X^0 = 1 \cdot X^0 = 1.
 \]
 Dies zeigt, dass $\psi$ ein Ringhomomorphismus ist.
 
 Es ist $f = \sum_{i=0}^n a_i X^i \in R[X]$ genau dann in $\Ker \psi$, wenn $\psi(f) = 0$, also $\phi(a_i) = 0$ für alle $i$, also $a_i \in \Ker \phi$ für alle $i$.
 
 Andererseits ist $g = \sum_{i=0}^n b_i X^i \in R'[X]$ genau dann in $\Img \psi$, wenn es ein $f = \sum_{i=0}^n a_i X^i \in R[X]$ mit $\psi(f) = g$ gibt, also $\phi(a_i) = b_i$ für alle $i$, also $b_i \in \Img \phi$ für alle $i$.
\end{proof}

\begin{bem}
  Betrachtet man $R[X]$ als abzählbare direkte Summe der additiven Gruppe von $R$ mit sich selbst, so folgt das obige Lemma fast direkt daraus, dass dann $\psi = \bigoplus_{n \in \N} \phi$. Nur dass $\psi$ bezüglich $\cdot$ ein Monoidhomomorphismus ist, folgt dann nicht direkt, da die Multiplikation in $R[X]$ nicht komponentenweise ist.
\end{bem}

Es sei $\pi : R \twoheadrightarrow R/\mf{a}$ die kanonische Projektion. Da $\pi$ ein Ringepimorphismus ist, folgt aus Lemma \ref{lem: ringhomo polynom}, dass $\pi$ einen Ringepimorphismus $\psi: R[X] \twoheadrightarrow (R/\mf{a})[X]$ induziert. Auch folgt wegen $\Ker \pi = \mf{a}$ aus dem Lemma, dass $\Ker \psi$ genau aus den Polynomen besteht, deren Koeffizienten alle in $\mf{a}$ liegen; wie im vorherigen Aufgabenteil gezeigt, ist dies gerade $\mf{b}$. Es ist daher
\[
 R[X]/\mf{b} = R[X]/\Ker\psi \cong \Img\psi = (R/\mf{a})[X].
\]





















\end{document}
