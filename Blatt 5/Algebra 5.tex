\documentclass[a4paper,10pt]{article}
%\documentclass[a4paper,10pt]{scrartcl}

\usepackage{xltxtra}
\usepackage[ngerman]{babel}
\usepackage{amsmath}
\usepackage{amssymb}
\usepackage{amsthm}
\usepackage{mathtools}
\usepackage{nicefrac}
\usepackage{enumerate}
\usepackage{leftidx}

\theoremstyle{definition}
\newtheorem{beh}{Behauptung}
\newtheorem{bem}[beh]{Bemerkung}
\newtheorem{lem}[beh]{Lemma}
\newtheorem*{anm}{Anmerkung}
\newtheorem*{ia}{Induktionsanfang}
\newtheorem*{is}{Induktionsschritt}

\renewcommand{\thesection}{Aufgabe 5.\arabic{section}.}
\renewcommand{\thesubsection}{(\roman{subsection})}
\renewcommand{\thesubsubsection}{}

\newcommand{\N}{\mathbb{N}}
\newcommand{\Z}{\mathbb{Z}}
\newcommand{\Q}{\mathbb{Q}}
\newcommand{\R}{\mathbb{R}}
\newcommand{\C}{\mathbb{C}}
\newcommand{\Sn}{\mathfrak{S}}
\newcommand{\mf}[1]{\mathfrak{#1}}
\newcommand{\dt}{\,\text{d}t}
\newcommand{\F}[1]{\mathbb{F}_{#1}}
\newcommand{\id}{\operatorname{id}}
\newcommand{\ord}{\operatorname{ord}}
\newcommand{\inn}{\operatorname{inn}}
\newcommand{\sgn}{\operatorname{sgn}}
\newcommand{\kchar}{\operatorname{char}}
\newcommand{\kgV}{\operatorname{kgV}}
\newcommand{\Id}{\operatorname{Id}}
\newcommand{\GL}[2]{\operatorname{GL}(#1,#2)}
\newcommand{\Img}{\operatorname{Im}}
\newcommand{\Ker}{\operatorname{Ker}}
\newcommand{\Hom}{\operatorname{Hom}}
\newcommand{\End}{\operatorname{End}}
\newcommand{\Aut}{\operatorname{Aut}}
\newcommand{\Inn}{\operatorname{Inn}}
\newcommand{\vect}[1]{\begin{pmatrix}#1\end{pmatrix}}
\newcommand{\gen}[1]{\left\langle#1\right\rangle}

\newenvironment{lgs}[1][c]{\left\{\setlength{\arraycolsep}{1pt}\begin{array}{#1}}{\end{array}\right.}

\renewcommand*{\arraystretch}{1.5}

\makeatletter
\renewcommand*\env@matrix[1][*\c@MaxMatrixCols c]{%
  \hskip -\arraycolsep
  \let\@ifnextchar\new@ifnextchar
  \array{#1}}
\makeatother

\setromanfont[Mapping=tex-text]{Linux Libertine O}
% \setsansfont[Mapping=tex-text]{DejaVu Sans}
% \setmonofont[Mapping=tex-text]{DejaVu Sans Mono}
\parindent0pt

\title{\textsc{Einführung in die Algebra \\ \Large Blatt 5}}
\author{Jendrik Stelzner}
\date{\today}

\begin{document}
\maketitle





\section{}





\section{}





\section{}

\begin{bem}\label{bem: 0 neq 1}
 Sei $R$ ein Ring mit mindestens zwei Elementen. Dann ist $0 \neq 1$ in $R$.
\end{bem}
\begin{proof}
 Da $R$ mindestens zwei Elemente hat, gibt es ein $a \in R$ mit $a \neq 0$. Für $1 \in R$ ist
 \[
  1 \cdot a = a \neq 0 = 0 \cdot a,
 \]
 also $0 \neq 1$.
\end{proof}







\section{}


\addtocounter{subsection}{1}
\subsection{}
Für alle $a \in R$ ist
\[
 a^2 + 1 = a+1 = (a+1)^2 = a^2 + 2a + 1,
\]
also $2a = 0$. Insbesondere ist $a = -a$.


\addtocounter{subsection}{-2}
\subsection{}
Für alle $a,b \in R$ ist
\[
 ab-ba = ab + ba = (a+b)^2 - a^2 - b^2 = a+b-a-b = 0,
\]
also $ab = ba$, und daher $R$ kommutativ.


\addtocounter{subsection}{1}
\subsection{}
Seien $a,b \in R$ mit $a \neq b$. Es ist
\[
 (a-b)ab = a^2 b - ab^2 = ab-ab = 0.
\]
Da $a \neq b$ ist $a-b \neq 0$, wegen der Nullteilerfreiheit von $R$ also $ab = 0$. Wegen der Nullteilerfreiheit ist also $a = 0$ oder $b = 0$. Aus der Beliebigkeit von $a$ und $b$ folgt, dass es neben $0$ nur ein weiters Element in $R$ gibt. Also ist $R = \{0,1\}$.
Betrachtet man die Verknüpfungstabellen von R,
\begin{equation*}
 \begin{matrix}[|c|c|c|]\hline
  + & 0 & 1 \\\hline
  0 & 0 & 1 \\\hline
  1 & 1 & 0 \\\hline
 \end{matrix}
 \quad
 \text{ und }
 \quad
 \begin{matrix}[|c|c|c|]\hline
  \cdot & 0 & 1 \\\hline
      0 & 0 & 0 \\\hline
      1 & 0 & 1 \\\hline
 \end{matrix}
\end{equation*}
so ist $R$ offenbar isomorph zu $\F{2}$.








\section{}


\subsection{}
Da $\mf{a}$ ein Ideal in $R$ ist, ist $ar \in \mf{a}$ für alle $a \in \mf{a}$ und $r \in R$. Es ist nun
\begin{align}
 \mf{b} = (\mathfrak{a}) \nonumber
 &= \sum_{a \in \mf{a}} a R[X]
 = \sum_{a \in \mf{a}} \left\{ a \sum_{i=0}^n a_i X^i : n \geq 0, a_i \in R \right\} \nonumber \\
 &= \sum_{a \in \mf{a}} \left\{ \sum_{i=0}^n a a_i X^i : n \geq 0, a_i \in R \right\}
 = \left\{ \sum_{i=0}^n a_i X^i : n \geq 0, a_i \in \mf{a} \right\} \label{eq: polynome ideal}.
\end{align}
Dabei ergibt sich die Gleichheit bei \eqref{eq: polynome ideal} wie folgt:

Für alle $f = \sum_{i=0}^n a a_i X^i \in a R[X]$ ist $a a_i \in \mf{a}$, da $a \in \mf{a}$ und $\mf{a}$ ein Ideal in $R$ ist, also $f$ ein Polynom mit Koeffizienten in $\mf{a}$.

Andererseits ist jedes Polynom $f = \sum_{i=0}^n a_i X^i$ mit Koeffizienten $a_0, \ldots, a_n \in \mf{a}$ die Summe der Monome $f_i := a_i X^i \in a_i R[X]$. Also ist $f \in \sum_{i=1}^n a_i R[X]$.











\end{document}
