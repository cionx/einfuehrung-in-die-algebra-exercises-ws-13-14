\documentclass[a4paper,10pt]{article}
%\documentclass[a4paper,10pt]{scrartcl}

\usepackage{xltxtra}
\usepackage[ngerman]{babel}
\usepackage{amsmath}
\usepackage{amssymb}
\usepackage{amsthm}
\usepackage{mathtools}
\usepackage{nicefrac}
\usepackage{enumerate}
\usepackage{leftidx}
\usepackage{tikz}

\newcounter{satze}
\newtheorem{beh}[satze]{Behauptung}
\newtheorem{bem}[satze]{Bemerkung}
\newtheorem{lem}[satze]{Lemma}
\newtheorem{kor}[satze]{Korollar}
\newtheorem*{defi}{Definition}
\newtheorem*{anm}{Anmerkung}
\theoremstyle{definition}
\newtheorem*{ia}{Induktionsanfang}
\newtheorem*{is}{Induktionsschritt}

\renewcommand{\thesection}{Aufgabe 11.\arabic{section}.}
\renewcommand{\thesubsection}{(\roman{subsection})}
\renewcommand{\thesubsubsection}{}

\newcommand{\N}{\mathbb{N}}
\newcommand{\Z}{\mathbb{Z}}
\newcommand{\Q}{\mathbb{Q}}
\newcommand{\R}{\mathbb{R}}
\newcommand{\C}{\mathbb{C}}
\newcommand{\Sn}{\mathfrak{S}}
\newcommand{\mf}[1]{\mathfrak{#1}}
\newcommand{\mc}[1]{\mathcal{#1}}
\newcommand{\dt}{\,\text{d}t}
\newcommand{\F}{\mathbb{F}}
\newcommand{\id}{\operatorname{id}}
\newcommand{\ord}{\operatorname{ord}}
\newcommand{\inn}{\operatorname{inn}}
\newcommand{\sgn}{\operatorname{sgn}}
\newcommand{\kchar}{\operatorname{char}}
\newcommand{\kgV}{\operatorname{kgV}}
\newcommand{\ggT}{\operatorname{ggT}}
\newcommand{\nil}{\operatorname{nil}}
\newcommand{\Id}{\operatorname{Id}}
\newcommand{\GL}[2]{\operatorname{GL}(#1,#2)}
\newcommand{\Img}{\operatorname{Im}}
\newcommand{\Ker}{\operatorname{Ker}}
\newcommand{\Hom}{\operatorname{Hom}}
\newcommand{\End}{\operatorname{End}}
\newcommand{\Aut}{\operatorname{Aut}}
\newcommand{\Inn}{\operatorname{Inn}}
\newcommand{\vect}[1]{\begin{pmatrix}#1\end{pmatrix}}
\newcommand{\gen}[1]{\left\langle#1\right\rangle}
\newcommand{\lb}{[\![}
\newcommand{\rb}{]\!]}
\newcommand{\Deg}{\operatorname{Deg}}
\newcommand{\rg}{\operatorname{rg}}

\newenvironment{lgs}[1][c]{\left\{\setlength{\arraycolsep}{1pt}\begin{array}{#1}}{\end{array}\right.}

\renewcommand*{\arraystretch}{1.5}

\makeatletter
\renewcommand*\env@matrix[1][*\c@MaxMatrixCols c]{%
  \hskip -\arraycolsep
  \let\@ifnextchar\new@ifnextchar
  \array{#1}}
\makeatother

\setromanfont[Mapping=tex-text]{Linux Libertine O}
% \setsansfont[Mapping=tex-text]{DejaVu Sans}
% \setmonofont[Mapping=tex-text]{DejaVu Sans Mono}
\parindent0pt

\title{\textsc{Einführung in die Algebra \\ \Large Blatt 11}}
\author{Jendrik Stelzner}
\date{\today}

\begin{document}
\maketitle





\section{}
Es bezeichne $f \in \Q[X]$ das Minimalpolynom von $z = \sqrt{3}+i \in \C$ über $\Q$. Dieses existiert, denn $\sqrt{3}$ und $i$ sind algebraisch über $\Q$, also $z$.

Da $z$ eine Nullstelle von $f$ ist, und $f \in \Q[X] \subseteq \R[X]$, ist auch $\bar{z}$ eine Nullstelle von $f$. Da $z \not\in \R$ ist dabei $\bar{z} \neq z$. Folglich hat $f$ in $\C[X]$ die beiden Linearfaktoren $X-z, X-\bar{z} \in \C[X]$. Inbesondere ist $\deg f \geq 2$. Da
\[
 (X-z)(X-\bar{z}) = X^2 -(z+\bar{z})X + |z|^2 = X^2 -2\sqrt{3}X + 4 \not\in \Q[X]
\]
ist sogar $\deg f > 2$.

Es ist auch $\deg f > 3$: Wäre $\deg f = 3$, so hätte $f$ zusätzlich zu $z$ und $\bar{z}$ noch eine reelle Nullstelle (denn jedes Polynom ungeraden Gerades in $\R[X]$, und damit auch in $\Q[X] \subseteq \R[X]$, hat eine reelle Nullstelle). Da $f$ normiert ist gäbe es also ein $\alpha \in \R$ mit
\begin{align*}
 f &= (X-z)(X-\bar{z})(X-\alpha) = (X^2 -2\sqrt{3}X + 4)(X-\alpha) \\
   &= X^3 - (2\sqrt{3}+\alpha)X^2 + (4+2\sqrt{3}\,\alpha)X -4\alpha \in \Q[X].
\end{align*}
Es wäre daher $4\alpha \in \Q$, also bereits $\alpha \in \Q$, und wegen $2\sqrt{3} + \alpha \in \Q$ damit auch $\sqrt{3} \in \Q$. Da dies nicht gilt, ist $\deg f > 3$.

Da $z$ eine Nullstelle des normierten Polynoms
\[
 (X-z)(X-\bar{z})(X+z)(X+\bar{z})
 = X^4 -4X^2 + 16 \in \Q[X]
\]
ist, folgt aus dem obigen Beobachtungen, dass
\[
 f = X^4 -4X^2 + 16.
\]
Wäre nämlich $X^4 -4X^2 + 16$ nicht das Minimalpolynom von $z$ über $\Q$, so müsste $X^4 -4X^2 + 16$ reduzibel sein. Dann gebe es ein normiertes Polynom $g \in \Q[X]$ vom Grad $1 \leq \deg g \leq 3$ mit $g \mid f$ und $g(z) = 0$, was den obigen Beobachtungen widerspricht.





\section{}
$f$ ist normiert und nach dem Eisensteinkriterium mit der Primzahl $p = 3$ auch irreduzibel. Folglich ist $f$ bereits das Minimalpolynom von $x$. Deshalb ist $\Q[X]/(f) \cong \Q(x)$, wobei ein entsprechender Körperisomorphismus durch
\[
 \varphi: \Q[X]/(f) \rightarrow \Q(x) \text{ mit }
 \varphi(\bar{q}) = q \text{ für alle } q \in \Q \text{ und }
 \varphi\left(\overline{X}\right) = x
\]
gegeben ist. Da $\deg f = 3$ ist $\bar{1}, \overline{X}, \overline{X}^2$ eine $\Q$-Basis von $\Q/(f)$. Also ist $1, x, x^2$ als Bild dieser Basis unter $\varphi$ eine $\Q$-Basis von $\Q(x)$.

Da $x$ eine Nullstelle von $f$ ist, ist
\[
 0 = f(x) = x^3 -6x^2 +9x +3.
\]
Es ist daher
\begin{align*}
  &\, 105x^2 -261x -81 \\
 =&\, 105x^2 -261x -81 + (x^2 +6x +27)(x^3 -6x^2 +9x +3) \\
 =&\, x^5
\shortintertext{und}
  &\, 69x^2 -153x -47 \\
 =&\, 69x^2 -153x -47 + (3x +16)(x^3 -6x^2 +9x +3) \\
 =&\, 3x^4 -2x^3 +1.
\end{align*}
Da offenbar $x \neq -2$ existiert das Element $1/(x+2) \in \Q(\alpha)$. Da
\begin{align*}
 (x+2) \left( \frac{1}{47} x^2 -\frac{8}{47} x +\frac{25}{47} \right)
 &= \frac{1}{47} x^3 -\frac{6}{47} x^2 +\frac{9}{47} x +\frac{50}{47} \\
 &= \frac{1}{47} \left( x^3 -6x^2 +9x +3 \right) +1
  = 1
\end{align*}
ist
\[
 \frac{1}{x+2} = \frac{1}{47} x^2 -\frac{8}{47} x +\frac{25}{47}.
\]







\section{}
Sei $K$ ein endlicher Körper. Für das Polynom
\[
 f := 1 + \prod_{\lambda \in K} (X-\lambda) \in K[X]
\]
ist $f(x) = 1$ für alle $x \in K$. Es halt also $f$ keine Nullstelle in $K$, weshalb $K$ nicht algebraisch abgeschlossen ist. Es gibt daher keinen endlichen, algebraisch abgeschlossenen Körper.


Die Aussage folgt auch aus dem folgenden Lemma:


\begin{lem}\label{lem: unendlich viele irreduzible Polynome}
 Für einen Körper $K$ gibt es unendlich viele normierte, irreduzible Polynome in $K[X]$.
\end{lem}
\begin{proof}
 Der Beweis läuft analog zum klassischen Beweis des Satzes von Euklid. Angenommen, die Menge
 \[
  P := \{g \in K[X] : g \text{ ist normiert und irreduzibel}\}
 \]
 ist endlich. Wir wissen, dass $P$ ein Repräsentantensystem der Primelemente in $K[X]$ ist, und sich jedes Element $f \in K[X], f \neq 0$ eindeutig als Produkt
 \[
  f = \varepsilon g_1 \cdots g_n
 \]
 mit $\varepsilon \in K$ und $g_1, \ldots, g_n \in P$ schreiben lässt. Es sei
 \[
  f := \prod_{g \in P} g.
 \]
 Für alle $g \in P$ ist $g \mid f$ und $g \nmid 1$, also $g \nmid (f+1)$. Da offenbar $f+1 \neq 0$ steht dies im Widerspruch zur Existenz einer Primfaktorzerlegung von $f+1$.
\end{proof}

\begin{kor}
 Jeder algebraisch abgeschlossene Körper besitzt unendlich viele Elemente.
\end{kor}
\begin{proof}
 Sei $K$ ein algebraisch abgeschlossener Körper. Da jedes Polynom aus $K[X]$ in Linearfaktoren zerfällt ist
 \[
  \{f \in K[X] : f \text{ ist normiert und irreduzibel}\} = \{X-\lambda \in K[X] : \lambda \in K\}.
 \]
 Da die linke Menge nach Lemma \ref{lem: unendlich viele irreduzible Polynome} unendlich ist, ist es auch die rechte. Es muss also $K$ unendlich sein.
\end{proof}





\end{document}
