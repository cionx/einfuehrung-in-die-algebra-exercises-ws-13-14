\documentclass[a4paper,10pt]{article}
%\documentclass[a4paper,10pt]{scrartcl}

\usepackage{xltxtra}
\usepackage[ngerman]{babel}
\usepackage{amsmath}
\usepackage{amssymb}
\usepackage{amsthm}
\usepackage{mathtools}
\usepackage{nicefrac}
\usepackage{enumerate}
\usepackage{leftidx}

\theoremstyle{definition}
\newtheorem*{beh}{Behauptung}
\newtheorem*{bem}{Bemerkung}
\newtheorem*{lem}{Lemma}
\newtheorem*{anm}{Anmerkung}
\newtheorem*{ia}{Induktionsanfang}
\newtheorem*{is}{Induktionsschritt}

\renewcommand{\thesection}{Aufgabe 1.\arabic{section}.}
\renewcommand{\thesubsection}{(\roman{subsection})}
\renewcommand{\thesubsubsection}{}

\newcommand{\N}{\mathbb{N}}
\newcommand{\Z}{\mathbb{Z}}
\newcommand{\Q}{\mathbb{Q}}
\newcommand{\R}{\mathbb{R}}
\newcommand{\C}{\mathbb{C}}
\newcommand{\dt}{\,\text{d}t}
\newcommand{\F}[1]{\mathbb{F}_{#1}}
\newcommand{\ord}{\operatorname{ord}}
\newcommand{\sgn}{\operatorname{sgn}}
\newcommand{\kchar}{\operatorname{char}}
\newcommand{\Id}{\operatorname{Id}}
\newcommand{\GL}[2]{\operatorname{GL}(#1,#2)}
\newcommand{\Image}{\operatorname{Im}}
\newcommand{\Ker}{\operatorname{Ker}}
\newcommand{\Hom}{\operatorname{Hom}}
\newcommand{\End}{\operatorname{End}}
\newcommand{\vect}[1]{\begin{pmatrix}#1\end{pmatrix}}
\newcommand{\erz}[1]{\left\langle#1\right\rangle}

\newenvironment{lgs}[1][c]{\left\{\setlength{\arraycolsep}{1pt}\begin{array}{#1}}{\end{array}\right.}

\makeatletter
\renewcommand*\env@matrix[1][*\c@MaxMatrixCols c]{%
  \hskip -\arraycolsep
  \let\@ifnextchar\new@ifnextchar
  \array{#1}}
\makeatother

\setromanfont[Mapping=tex-text]{Linux Libertine O}
% \setsansfont[Mapping=tex-text]{DejaVu Sans}
% \setmonofont[Mapping=tex-text]{DejaVu Sans Mono}
\parindent0pt

\title{Einführung in die Algebre — Blatt 1}
\author{Jendrik Stelzner}
\date{\today}

\begin{document}
\maketitle





\section{}





\section{}





\section{}





\section{}


\subsection{}
Da, wie aus der Vorlesung bekannt, $\erz{g}$ für alle $g \in G$ eine Untergruppe von $G$ ist, ist nach dem Satz von Lagrange $\ord g = \ord \erz{g}$ für alle $g \in G$ ein Teiler von $\ord G$, und somit ebenfalls ungerade.

Da $aba = b \Leftrightarrow b = a^{-1} b a^{-1}$ ist für alle $n \in \N$
\[
 b^{2n+1} = a (b a a^{-1} b a^{-1} a)^n ba = ab^{2n+1}a.
\]
Da $\ord b$ ungerade ist, ist damit insbesondere
\[
 e = b^{\ord b} = ab^{\ord b}a = aea = a^2,
\]
also $a$ selbstinvers. Da damit $\erz{a} = \{e, a\}$, aber $\ord a$ ungerade ist, muss $a = e$.



\subsection{}
Da $c = abcba$ ist $cb = abcbab = ab \cdot cb \cdot ab$, nach Aufgabenteil \textbf{(i)} ist daher $ab = e$.




\section{}







\end{document}
