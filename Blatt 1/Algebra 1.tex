\documentclass[a4paper,10pt]{article}
%\documentclass[a4paper,10pt]{scrartcl}

\usepackage{xltxtra}
\usepackage[ngerman]{babel}
\usepackage{amsmath}
\usepackage{amssymb}
\usepackage{amsthm}
\usepackage{mathtools}
\usepackage{nicefrac}
\usepackage{enumerate}
\usepackage{leftidx}

\theoremstyle{definition}
\newtheorem*{beh}{Behauptung}
\newtheorem*{bem}{Bemerkung}
\newtheorem*{lem}{Lemma}
\newtheorem*{anm}{Anmerkung}
\newtheorem*{ia}{Induktionsanfang}
\newtheorem*{is}{Induktionsschritt}

\renewcommand{\thesection}{Aufgabe 1.\arabic{section}.}
\renewcommand{\thesubsection}{(\roman{subsection})}
\renewcommand{\thesubsubsection}{}

\newcommand{\N}{\mathbb{N}}
\newcommand{\Z}{\mathbb{Z}}
\newcommand{\Q}{\mathbb{Q}}
\newcommand{\R}{\mathbb{R}}
\newcommand{\C}{\mathbb{C}}
\newcommand{\dt}{\,\text{d}t}
\newcommand{\F}[1]{\mathbb{F}_{#1}}
\newcommand{\id}{\operatorname{id}}
\newcommand{\ord}{\operatorname{ord}}
\newcommand{\inn}{\operatorname{inn}}
\newcommand{\sgn}{\operatorname{sgn}}
\newcommand{\kchar}{\operatorname{char}}
\newcommand{\Id}{\operatorname{Id}}
\newcommand{\GL}[2]{\operatorname{GL}(#1,#2)}
\newcommand{\Img}{\operatorname{Im}}
\newcommand{\Ker}{\operatorname{Ker}}
\newcommand{\Hom}{\operatorname{Hom}}
\newcommand{\End}{\operatorname{End}}
\newcommand{\vect}[1]{\begin{pmatrix}#1\end{pmatrix}}
\newcommand{\gen}[1]{\left\langle#1\right\rangle}

\newenvironment{lgs}[1][c]{\left\{\setlength{\arraycolsep}{1pt}\begin{array}{#1}}{\end{array}\right.}

\makeatletter
\renewcommand*\env@matrix[1][*\c@MaxMatrixCols c]{%
  \hskip -\arraycolsep
  \let\@ifnextchar\new@ifnextchar
  \array{#1}}
\makeatother

\setromanfont[Mapping=tex-text]{Linux Libertine O}
% \setsansfont[Mapping=tex-text]{DejaVu Sans}
% \setmonofont[Mapping=tex-text]{DejaVu Sans Mono}
\parindent0pt

\title{\textsc{Einführung in die Algebra \\ \Large Blatt 1}}
\author{Jendrik Stelzner}
\date{\today}

\begin{document}
\maketitle





\section{}

\begin{bem}
 Sei $G$ eine Gruppe und $H \subset G$ ein Untergruppe mit $(G : H) = 2$. Dann ist $H$ ein Normalteiler in $G$.
\end{bem}
\begin{proof}[Beweis der Bemerkung]
 Da $(G : H) = 2$ zerfällt in $G$ in zwei Links- bzw. Rechtsnebenklassen, nämlich je $H$ und $H^c$. Für alle $g \in H$ ist damit $gH = H = Hg$ und für alle $g \in H^c$ ist $gH = H^c = Hg$.
\end{proof}

Es ist $S_3 = \{\id, \sigma, \sigma^2, \tau_{12}, \tau_{13}, \tau_{23} \}$, wobei in Zykelschreibweise $\sigma = (1,2,3)$, $\sigma^2 = (3,2,1)$, $\tau_{12} = (1,2)$, $\tau_{13} = (1,3)$ und $\tau_{23} = (2,3)$.

Da $\ord S_3 = 3! = 6$ folgt aus dem Satz von Lagrange, dass $\ord H \in \{1,2,3,6\}$ für jede Untergruppe $H \subseteq G$. Neben den beiden trivialen Untergruppen $\{\id\}$ und $S_3$ kann $S_3$ also nur zwei- oder dreielementige Untergruppen enthalten.

Offenbar sind $\gen{\tau_{12}} = \{\id,\tau_{12}\}$, $\gen{\tau_{13}} = \{\id, \tau_{13}\}$ und $\gen{\tau_{23}} = \{\id, \tau_{23}\}$ Untergruppen der Ordnung $2$ . Dies sind auch die einzigen Untergruppen dieser Ordnung: Ist $H = \{\id, a\}$ eine Untergruppe mit $\ord H = 2$, so muss $a^2 = \id$, also $a$ selbstinvers sein. Die einzigen selbstinversen Elemente in $S_3$ sind aber $e$, $\tau_{12}$, $\tau_{13}$ und $\tau_{23}$ (da $\sigma \sigma^2 = \sigma^2 \sigma = \sigma^3 = \id$).

Offenbar ist $\gen{\sigma} = \{\id, \sigma, \sigma^2\}$ eine Untergruppe der Ordnung $3$. Es ist auch die einzige Untergruppe dieser Ordnung: Ist $H = \{\id, a, b\}$ eine Untergruppe mit $\ord H = 3$, so ist, wie aus der Vorlesung bekannt, $H$ zyklisch und von $a$, bzw. $b$ erzeugt. Insbesondere muss daher $\ord a = \ord b = \ord H = 3$. Die einzigen beiden Elemente in $S_3$ mit Ordnung $3$ sind jedoch $\sigma$ und $\sigma^2$.

Die Untergruppen von $S_3$ sind also $\{\id\}$, $\gen{\tau_{12}}$, $\gen{\tau_{13}}$, $\gen{\tau{23}}$, $\gen{\sigma}$ und $S_3$.

$\{\id\}$ und $S_3$ sind trivialerweise Normalteiler in $S_3$. Aus der Bemerkung folgt, dass auch $\gen{\sigma}$ ein Normalteiler in $S_3$ ist, da $(S_3 : \gen{\sigma}) = 2$. $\gen{\tau_{12}}$, $\gen{\tau_{13}}$ und $\gen{\tau_{23}}$ sind keine Normalteiler in $S_3$, denn
\begin{align*}
 \tau_{23} \{ \id, \tau_{12}\}
 = \{ \tau_{23}, \sigma^2 \}
 \neq \{ \tau_{23}, \sigma \}
 &= \{\id, \tau_{12}\} \tau_{23},\\
 \tau_{12} \{ \id, \tau_{13}\}
 = \{ \tau_{12}, \sigma^2 \}
 \neq \{ \tau_{12}, \sigma \}
 &= \{ \id, \tau_{13} \} \tau_{12} \text{ und}\\
 \tau_{12} \{ \id, \tau_{23} \}
 = \{ \tau_{12}, \sigma \}
 \neq  \{ \tau_{12}, \sigma^2 \}
 &= \{ \id, \tau_{23} \} \tau_{12}.
\end{align*}





\section{}
Im Folgenden sei $Q := \{E,-E,I,-I,J,-J,K,-K\}$, wobei für die multiplikative Struktur $(Q,\cdot)$ (die sich als Gruppe herausstellen wird), ebenfalls $Q$ geschrieben wird.

\subsection{}
Wie die Multiplikationstabelle verrät, ist für $A,B \in \{E, I, J, K\}$ auch $AB \in Q$, folglich ist sogar für alle $A,B \in Q$ auch $AB \in Q$, $Q$ ist also abgeschlossen bezüglich der Multiplikation.
\begin{center}
 \begin{tabular}{c|cccc} 
  $\cdot$ & $E$ & $ I$ & $ J$ & $ K$  \\\hline 
      $E$ & $E$ & $ I$ & $ J$ & $ K$  \\ 
      $I$ & $I$ & $-E$ & $-K$ & $ J$  \\
      $J$ & $J$ & $ K$ & $-E$ & $-I$ \\
      $K$ & $K$ & $-J$ & $ I$ & $-E$
 \end{tabular}
\end{center}
Die Assoziativität der Multiplikation vererbt sich direkt von $M(2 \times 2, \C)$ auf $Q$.
Da $E \in Q$ die Einheitsmatrix ist, gibt es in $Q$ ein neutrales Element. Wie mit Aufgabenteil \textbf{(ii)} folgt, ist $E^{-1} = E, (-E)^{-1} = -E, I^{-1} = -I, J^{-1} = -J$ und $K^{-1} = -K$, also gibt es für alle $A \in Q$ ein Inverses $A^{-1} \in Q$. Damit ist $Q$ eine Gruppe.


\subsection{}
Stupides Nachrechnen ergibt, dass
\begin{align*}
 I^2
 &=
 \begin{pmatrix}
   0 & 1\\
  -1 & 0
 \end{pmatrix}^2
 =
 \begin{pmatrix}
  -1 &  0\\
   0 & -1
 \end{pmatrix}
 = -
 \begin{pmatrix}
  1 & 0\\
  0 & 1
 \end{pmatrix}
 = -E,\\
 J^2
 &=
 \begin{pmatrix}
  0 & i\\
  i & 0
 \end{pmatrix}^2
 =
 \begin{pmatrix}
  -1 &  0\\
   0 & -1
 \end{pmatrix}
 = -
 \begin{pmatrix}
  1 & 0\\
  0 & 1
 \end{pmatrix}
 = -E,\\
 K^2
 &=
 \begin{pmatrix}
  -i & 0\\
   0 & i
 \end{pmatrix}^2
 =
 \begin{pmatrix}
  -1 &  0\\
   0 & -1
 \end{pmatrix}
 = -
 \begin{pmatrix}
  1 & 0\\
  0 & 1
 \end{pmatrix}
 = -E,
\intertext{sowie, entgegen der falschen Angabe in der Aufgabenstellung,}
 IJK
 &=
 \begin{pmatrix}
   0 & 1\\
  -1 & 0
 \end{pmatrix}
 \begin{pmatrix}
  0 & i\\
  i & 0
 \end{pmatrix}
 \begin{pmatrix}
  -i & 0\\
   0 & i
 \end{pmatrix}
 =
 \begin{pmatrix}
  i &  0\\
  0 & -i
 \end{pmatrix}
 \begin{pmatrix}
  -i & 0\\
   0 & i
 \end{pmatrix}\\
 &= -K^2 = E, \text{ dafür aber} \\
 KJI
 &=
 \begin{pmatrix}
  -i & 0\\
   0 & i
 \end{pmatrix}
 \begin{pmatrix}
  0 & i\\
  i & 0
 \end{pmatrix}
 \begin{pmatrix}
   0 & 1\\
  -1 & 0
 \end{pmatrix}
 =
 \begin{pmatrix}
   0 & 1\\
  -1 & 0
 \end{pmatrix}
 \begin{pmatrix}
   0 & 1\\
  -1 & 0
 \end{pmatrix} \\
 &= I^2 = -E.
\end{align*}


\subsection{}
Da $\ord Q = 8$ ist nach dem Satz von Lagrange $\ord H \in \{1,2,4,8\}$ für jede Untergruppe $H \subseteq Q$. Neben den trivialen Untergruppen $\{E\}$ und $Q$ kann $Q$ also nur Untergruppen der Ordnung $2$ und $4$ haben.

Offenbar ist $\{E,-E\}$ eine Untergruppe der Ordnung $2$ von $Q$; dies ist auch die einzige Untergruppe dieser Ordung: Ist $H = \{E, A\} \subseteq Q$ eine Untergruppe mit $\ord H = 2$, so muss $A^2 = E$, also $A$ selbstinvers sein. Aus \textbf{Aufgabenteil (ii)} folgt jedoch, dass $E$ und $-E$ die einzige selbstinverse Elemente in $Q$ sind, weshalb $A = -E$ gelten muss.

Für alle $A \in \{I,J,K\}$ ist $\{E,-E,A,-A\}$ offenbar eine Untergruppe der Ordnung $4$ von $Q$. Dies sind auch die einzigen Untergruppen dieser Ordung: Ist $H \subset Q$ eine Untergruppe mit $\ord H = 4$, so gibt es ein $A \in H$ mit $A \not \in \{E,-E\}$. Es muss dann auch $E \in H$, $-E = A^2 \in H$, und $A^{-1} = -A \in H$. Also muss bereits $H = \{E,-E,A,-A\}$.

Die Untergruppen von $Q$ sind also $\{E\}$, $\{E,-E\}$, $\{E,-E,I,-I\}$, $\{E,-E,J,-J\}$, $\{E,-E,K,-K\}$ und $Q$. Diese Untergruppen sind sogar alle normal in $Q$: Für $\{E\}$ und $Q$ ist dies offensichtlich, für die Untergruppen der Ordnung $4$, und damit Index $2$, folgt es aus der Bemerkung in \textbf{Aufgabe 1.1}, und für $\{E,-E\}$ folgt es daraus, dass $A\{E,-E\} = \{A,-A\} = \{E,-E\}A$ für alle $A \in Q$, da $E$ und $-E$ mit allen Matrizen in $M(2 \times 2, \C)$ kommutieren.





\section{}


\subsection{}
Wie aus der Vorlesung bekannt genügt es zu zeigen, dass $gHg^{-1} = H$ für alle $g \in G$. Sei hierzu $g \in G$ beliebig aber fest. Wie aus Lineare Algebra I bekannt ist $\inn_g : G \rightarrow G, h \mapsto ghg^{-1}$ ein Gruppenautomorphismus von $G$. Daher ist insbesondere $\ord H = \ord \inn_g(H)$. Da aber $H$ nach Annahme die einzige Untergruppe von $G$ mit Ordung $\ord H$ ist, muss $gHg^{-1} = \inn_g(H) = H$. Aus der Beliebigkeit von $g$ folgt damit die zu zeigende Aussage.


\subsection{}
\begin{bem}
 Seien $G$ und $G'$ Gruppen, $G$ endlich, und $\varphi : G \rightarrow G'$ ein Gruppenhomomorphismus. Dann ist $\ord G = \ord \Ker \varphi \cdot \ord \Img \varphi$.
\end{bem}
\begin{proof}[Beweis der Bemerkung]
 Wie aus der Vorlesung bekannt ist $G/\Ker \varphi \cong \Img \varphi$, also insbesondere $(G : \Ker \varphi) = \ord G/\Ker \varphi = \ord \Img \varphi$. Nach dem Satz von Lagrange gilt $\ord G = \ord \Ker \varphi \cdot (G : \Ker \varphi)$. Einsetzen ergibt, dass \mbox{$\ord G = \ord \Ker \varphi \cdot \ord \Img \varphi$}.
\end{proof}

Sei $F$ eine Untergruppe von $G$ mit $\ord F = \ord H$. Es gilt zu zeigen, dass $F = H$.

Hierzu betrachte man die kanonische Abbildung $\pi : G \rightarrow G/H$. Da $F$ eine Untergruppe von $G$ ist, ist $\pi(F)$ eine Untergruppe von $\pi(G) = G/H$, insbesondere ist nach dem Satz von Lagrange daher $\ord \pi(F)$ ein Teiler von $\ord G/H = (G : H)$. Betrachtet man die Komposition
\[
 \varphi: F \hookrightarrow G \xrightarrow{\pi} G/H
\]
so ist diese ein Homomorphismus mit $\Ker \varphi = F \cap H$ und $\Img \varphi = \pi(F)$, nach der Bemerkung also
\[
 \ord H = \ord F = \ord F \cap H \cdot \ord \pi(F).
\]
Es ist also $\ord \pi(F)$ auch ein Teiler von $\ord H$. Da $\ord H$ und $(G : H)$ teilerfremd sind, muss $\ord \pi(F) = 1$, also $\pi(F) = \{1\}$ und daher $F \subseteq \Ker \pi = H$. Da $\ord F = \ord H$ gilt schon $F = H$.









\section{}


\subsection{}
Da, wie aus der Vorlesung bekannt, $\gen{g}$ für alle $g \in G$ eine Untergruppe von $G$ ist, ist nach dem Satz von Lagrange $\ord g = \ord \gen{g}$ für alle $g \in G$ ein Teiler von $\ord G$, und somit ebenfalls ungerade.

Da $aba = b \Leftrightarrow b = a^{-1} b a^{-1}$ ist für alle $n \in \N$
\[
 b^{2n+1} = a (b a a^{-1} b a^{-1} a)^n ba = ab^{2n+1}a.
\]
Da $\ord b$ ungerade ist, ist damit insbesondere
\[
 e = b^{\ord b} = ab^{\ord b}a = aea = a^2,
\]
also $a$ selbstinvers. Da damit $\gen{a} = \{e, a\}$, aber $\ord a$ ungerade ist, muss $a = e$.



\subsection{}
Da $c = abcba$ ist $cb = abcbab = ab \cdot cb \cdot ab$, nach Aufgabenteil \textbf{(i)} also $ab = e$.




\section{}
Es ist
\[
 U \cong U / \{1\} = U / (U \cap N) \cong UN/N,
\]
wobei die letzte Isomorphie aus dem ersten Isomorphiesatz folgt. Analog ergibt sich, dass $V \cong VN/N$. Um zu zeigen, dass $U \cong V$ genügt es daher zu zeigen, dass \mbox{$UN = VN = G$}. Dies ergibt sich durch Fallunterscheidung:

Ist $N = \{1\}$, so ist $N \subset U$ und $N \subset V$, wegen der Maximalitätseigenschaft von $N$ daher $U=V=G$ und damit insbesondere $UN=VN=G$.

Ist $N \neq \{1\}$, so ist gibt es wegen $U \cap N = \{1\}$ ein $u \in U$ mit $u \not \in N$. Es ist dann $uN \subseteq{UN}$ aber $uN \cap N = \emptyset$, da $aN \in N \Leftrightarrow a \in N$ für alle $a \in G$. Daher ist $UN \neq N$, also $N \subset UN$ und damit $UN = G$. Analog ergibt sich, dass auch $VN = G$.








\end{document}
