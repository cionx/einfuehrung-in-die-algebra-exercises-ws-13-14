\documentclass[a4paper,10pt]{article}
%\documentclass[a4paper,10pt]{scrartcl}

\usepackage{xltxtra}
\usepackage[ngerman]{babel}
\usepackage{amsmath}
\usepackage{amssymb}
\usepackage{amsthm}
\usepackage{mathtools}
\usepackage{nicefrac}
\usepackage{enumerate}
\usepackage{leftidx}



\newcounter{satze}
\newtheorem{beh}[satze]{Behauptung}
\newtheorem{bem}[satze]{Bemerkung}
\newtheorem{lem}[satze]{Lemma}
\newtheorem*{defi}{Definition}
\newtheorem*{anm}{Anmerkung}
\theoremstyle{definition}
\newtheorem*{ia}{Induktionsanfang}
\newtheorem*{is}{Induktionsschritt}

\renewcommand{\thesection}{Aufgabe 7.\arabic{section}.}
\renewcommand{\thesubsection}{(\roman{subsection})}
\renewcommand{\thesubsubsection}{}

\newcommand{\N}{\mathbb{N}}
\newcommand{\Z}{\mathbb{Z}}
\newcommand{\Q}{\mathbb{Q}}
\newcommand{\R}{\mathbb{R}}
\newcommand{\C}{\mathbb{C}}
\newcommand{\Sn}{\mathfrak{S}}
\newcommand{\mf}[1]{\mathfrak{#1}}
\newcommand{\mc}[1]{\mathcal{#1}}
\newcommand{\dt}{\,\text{d}t}
\newcommand{\F}[1]{\mathbb{F}_{#1}}
\newcommand{\id}{\operatorname{id}}
\newcommand{\ord}{\operatorname{ord}}
\newcommand{\inn}{\operatorname{inn}}
\newcommand{\sgn}{\operatorname{sgn}}
\newcommand{\kchar}{\operatorname{char}}
\newcommand{\kgV}{\operatorname{kgV}}
\newcommand{\ggT}{\operatorname{ggT}}
\newcommand{\nil}{\operatorname{nil}}
\newcommand{\Id}{\operatorname{Id}}
\newcommand{\GL}[2]{\operatorname{GL}(#1,#2)}
\newcommand{\Img}{\operatorname{Im}}
\newcommand{\Ker}{\operatorname{Ker}}
\newcommand{\Hom}{\operatorname{Hom}}
\newcommand{\End}{\operatorname{End}}
\newcommand{\Aut}{\operatorname{Aut}}
\newcommand{\Inn}{\operatorname{Inn}}
\newcommand{\vect}[1]{\begin{pmatrix}#1\end{pmatrix}}
\newcommand{\gen}[1]{\left\langle#1\right\rangle}
\newcommand{\lb}{[\![}
\newcommand{\rb}{]\!]}
\newcommand{\Deg}{\operatorname{Deg}}

\newenvironment{lgs}[1][c]{\left\{\setlength{\arraycolsep}{1pt}\begin{array}{#1}}{\end{array}\right.}

\renewcommand*{\arraystretch}{1.5}

\makeatletter
\renewcommand*\env@matrix[1][*\c@MaxMatrixCols c]{%
  \hskip -\arraycolsep
  \let\@ifnextchar\new@ifnextchar
  \array{#1}}
\makeatother

\setromanfont[Mapping=tex-text]{Linux Libertine O}
% \setsansfont[Mapping=tex-text]{DejaVu Sans}
% \setmonofont[Mapping=tex-text]{DejaVu Sans Mono}
\parindent0pt

\title{\textsc{Einführung in die Algebra \\ \Large Blatt 7}}
\author{Jendrik Stelzner}
\date{\today}

\begin{document}
\maketitle





\section{}
Für $x,y \in \C$ mit $xy = 1$ muss $|x||y| = 1$, also $|x| \leq 1$ oder $|y| \leq 1$. Für $x,y \in \Z\Big[\sqrt{-n}\Big]$ mit $xy = 1$ ist also $x \in \{1,-1\}$ oder $y \in \{1,-1\}$ für $n > 1$ und $x \in \{1,-1,i,-i\}$ oder $y \in \{1,-1,i,-i\}$ für $n=1$. Es ist daher
\[
 \left(\Z\Big[\sqrt{-n}\Big]\right)^* =
 \begin{cases}
  \{1,-1,i,-i\} & \text{ für } n = 1, \\
  \{1,-1\}      & \text{ für } n > 1,
 \end{cases}
\]
da die entsprechenden Elemente, wenn in $\Z\Big[\sqrt{-n}\Big]$ enthalten, jeweils in Paaren von multiplikativ Inversen enthalten sind.






\section{}
Für $x,y \in \C$ mit $xy=21$ ist $|x||y| = 21$, also muss $|x| \leq \sqrt{21}$ oder $|y| \leq \sqrt{21}$. Es genügt daher die $a+\sqrt{5}bi = z \in \Z[\sqrt{-5}]$ mit $|z| \leq \sqrt{21}$, also $a^2 + 5b^2 \leq 21$ auf Teilbarkeit zu überprüfen. Da für jeden Teiler $z \in \Z[\sqrt{-5}]$ auch $-z, \bar{z}, -\bar{z} \in \Z[\sqrt{-5}]$ Teiler von $21$ sind, genügt es auch die $a+\sqrt{5}bi \in \Z[\sqrt{-n}]$ mit $a,b \geq 0$ auf Teilbarkeit zu überprüfen.

Es ergeben sich mit diesen beiden Beschränkungen die möglichen Kandidaten
\[
 1, 2, 3, 4, 1+\sqrt{5}i, 1+2\sqrt{5}i, 2+\sqrt{5}i, 3+\sqrt{5}i, 4+\sqrt{5}i.
\]
Einfaches Hinsehen und kurzes Nachrechnen ergibt, dass von diesen Zahlen nur
\[
 1, 3, 1+2\sqrt{5}i \text{ und } 4+\sqrt{5}i
\]
Teiler von $21$ sind. Die Teiler von $21$ in $\Z[i]$ sind also
\begin{gather*}
 1, -1, 21, -21, 3, -3, 7, -7, \\
 1+2\sqrt{5}i, -1-2\sqrt{5}i, 1-2\sqrt{5}i, -1+2\sqrt{5}i, \\
 4+\sqrt{5}i, -4-\sqrt{5}i, 4-\sqrt{5}i, -4-\sqrt{5}i.
\end{gather*}








\section{}

\begin{defi}
 Für einen Ring $R$ bezeichnet
 \[
  \nil(R) := \{x \in R : x^n = 0 \text{ für ein } n \in \N\}
 \]
 das Nilradikal von $R$.
\end{defi}

\begin{bem}\label{bem: nilradikal}
 Sei $R$ ein kommutativer Ring. Dann gilt
 \begin{enumerate}[(i)]
  \item $\nil(R)$ ist ein Ideal von $R$.
  \item Für $e \in R^*$ und $a \in \nil(R)$ ist $e+a \in R^*$.
 \end{enumerate}
\end{bem}

\begin{proof}
 \subsubsection*{(i)}
  Es ist $0 \in \nil(R)$, also $\nil(R)$ nicht leer. Für $a,b \in \nil(R)$ gibt es $n,m \in \N$ mit $a^n = b^m = 0$, also ist
  \[
   (a+b)^{n+m} = \sum_{k=1}^{n+m} \binom{n+m}{k} a^{n+m-k} b^k = 0
  \]
  und daher $a+b \in \nil(R)$. Auch ist für alle $r \in R$
  \[
   (ar)^n = a^n r^n = 0,
  \]
  also $ar \in R$. Insbesondere ist daher für alle $a \in \nil(R)$ auch $-a = (-1) \cdot a \in R$.
 \subsubsection*{(ii)}
  Für $e \in R^*$ und $a \in \nil(R)$ mit $a^n = 0$ ist $1-ae^{-1} \in R^*$, da $\left(ae^{-1}\right)^n = 0$ und daher
  \[
   \left(\sum_{k=0}^{n-1} \left(-ae^{-1}\right)^k\right) \left(1+ae^{-1}\right)
   = 1+ (-1)^{n-1} \left(ae^{-1}\right)^n
   = 1.
  \]
  Daher ist auch $e+a = e\left(1+ae^{-1}\right) \in R^*$.
\end{proof}

Da $\nil(R) \subseteq \nil(R[X])$ ist auch $\left(\nil(R)\right) \subseteq \nil(R[X])$. Dabei ist, wie in einem früheren Übungsblatt gezeigt,
\[
 \left(\nil(R)\right)
 = \left\{\sum_{i=0}^n a_i X^i : n \geq 0, a_i \in \nil(R) \text{ für alle } i\right\}.
\]
Nach Bemerkung \ref{bem: nilradikal} ist also das Polynom $f = \sum_{i=0}^n a_i X^i$ mit $n \geq 0$, $a_0 \in R^*$ und $a_i \in \nil(R)$ für alle $i$ invertierbar.

Sei andererseits $f = \sum_{i=0}^n a_i X^i \in R[X]$, mit $n \geq 0$ und $a_n \neq 0$, invertierbar, d.h. es gibt ein $g = \sum_{i=0}^m b_i X^i \in R[X]$, mit $m \geq 0$ und $b_m \neq 0$, so dass $fg = 1$. Da damit $a_0 b_0 = 1$ müssen $a_0$ und $b_0$ invertierbar sein. Ist $n > 0$, so bemerken wir:

\begin{beh}\label{beh: a_n nilpotent}
 Es ist $a_n^{k+1} b_{m-k} = 0$ für $k=0,\ldots,m$.
\end{beh}
\begin{proof}
 Der Beweis verläuft per Induktion über $k$.
 \begin{ia}Für $k=0$ gilt: Wäre $a_n b_m \neq 0$, so wäre
  \[
   0 = \deg(1) = \deg(fg) = \deg(f)+\deg(g) = n+m \geq n > 0.
  \]
 \end{ia}
 \begin{is}
 Sei $1 \leq k \leq n$ und gelte die Aussage für $k-1$. Da $fg = 1$ ist
 \[
  0 = \sum_{\mu+\nu = n+m-k} a_\mu b_\nu.
 \]
 Multiplikation der Gleichung mit $a_n^k$ ergibt
 \[
  0 = \sum_{\mu+\nu = n+m-k} a_n^k a_\mu b_\nu \underset{\text{IV.}}= a_n^{k+1} b_{m-k}.
 \]
 \end{is}
\end{proof}
Aus Behauptung \ref{beh: a_n nilpotent} folgt insbesondere, dass $a_n^{m+1} b_0 = 0$. Da $b_0$ invertierbar ist, ist $a_n$ daher nilpotent. Da nach Bemerkung \ref{bem: nilradikal} daher auch $f- a_n X^n$ invertierbar ist, ergibt sich durch Wiederholung der obigen Argumentation induktiv, dass $a_i$ für alle $1 \leq i \leq n$ nilpotent ist.










\section{}

\begin{defi}
Sei $R$ ein kommutativer Ring. Für $p = \sum_{i=0}^\infty a_i X^i \in R[\![x]\!]$ bezeichnet
\[
 \Deg(p) :=
 \begin{cases}
  \min \{i \in \N : a_i \neq 0\} & \text{falls } f \neq 0, \\
                          \infty & \text{sonst}.
 \end{cases}
\]
den Grad von $p$.
\end{defi}
Für einen kommutativen Ring $R$ und $p,q \in R[\![x]\!]$ ist
\begin{equation}\label{eq: Deg Ungleichungen}
 \Deg(p+q) \geq \min \{\Deg(p), \Deg(q)\} \text{ und }
 \Deg(pq) \geq \Deg(p) + \Deg(q).
\end{equation}
Ist $R$ darüber hinaus nullteilerfrei, so gilt sogar
\begin{equation}\label{eq: Deg nullteilerfrei}
 \Deg(pq) = \Deg(p) + \Deg(q).
\end{equation}
Die Beweise der entsprechenden Aussage laufen analog zu den Beweisen der entsprechenden Aussagen für die Gradfunktion $\deg$ von $R[X]$.


\subsection{}
Ist $R$ kein Integritätsring, so ist auch $R[X] \subsetneq R[\![x]\!]$ kein Integritätsring, also auch $R[\![x]\!]$ nicht. Ist $R[\![x]\!]$ kein Integritätsring, so gibt es $p, q \in R[\![x]\!]$ mit $p,q \neq 0$, also $\Deg(p), \Deg(q) < \infty$, aber $pq = 0$, also $\Deg(pq) = \infty$. Mit \eqref{eq: Deg nullteilerfrei} folgt, dass $R$ kein Integritätsring ist.


\subsection{}
Ist $p = \sum_{i=0}^\infty a_i X^i \in R[\![x]\!]$ invertierbar, so gibt es $q = \sum_{i=0}^\infty b_i X^i \in R[\![x]\!]$ mit $pq = 1$. Insbesondere ist daher
\[
 1 = (pq)_1 = a_0 b_0,
\]
also $a_0$ invertierbar.

Ist $p = \sum_{i=0}^\infty a_i X^i \in R[\![x]\!]$ mit $a_0$ invertierbar, so definieren wir eine Folge $(b_i)_{i \in \N}$ auf $R$ rekursiv durch
\[
 b_0 := a_0^{-1} \text{ und } b_i := -a_0^{-1} \sum_{j=1}^i a_j b_{i-j},
\]
und $q := \sum_{i=0}^\infty b_i X^i$ als die entsprechende Potenzreihe. Für $e = pq$ ergibt sich dann für alle $i \in \N$
\[
 e_i
 = \sum_{j=0}^i a_j b_{i-j}
 = \sum_{j=1}^i a_j b_{i-j} + a_0 b_i
 = \sum_{j=1}^i a_j b_{i-j} - \sum_{j=1}^i a_j b_{i-j}
 = 0.
\]
Also ist $e = 1$ und $p$ daher invertierbar mit $p^{-1} = q$. Inbesondere ergibt sich das folgende Lemma:

\begin{lem} \label{lem: starkes Lemma}
 Sei $K$ ein Körper und seien $p,q \in K[\![x]\!]$. Dann gilt:
 \begin{enumerate}[(i)]
  \item $p$ ist genau dann invertierbar, wenn $\Deg p = 0$. \label{item: körper invertierbar}
  \item Ist $\Deg p = \Deg q$, so sind $p$ und $q$ assoziiert. Ist $\Deg p = \Deg q < \infty$, so sind $p$ und $q$ assoziiert zu $X^{\Deg p}$.
  \item Ist $\Deg p \geq \Deg q$, so ist $q \mid p$.
 \end{enumerate}
\end{lem}
\begin{proof}
 \subsubsection*{\emph{(i)}}
  $p = \sum_{i=0}^{\infty} a_i X^i$ ist genau dann invertierbar, wenn $a_0$ invertierbar ist, also genau dann wenn $a_0 \neq 0$, was wiederum äquivalent zu $\Deg a_0 = 0$ ist.
 \subsubsection*{\emph{(ii)}}
  Ist $p = q = 0$ so ist nichts zu zeigen. Ansonsten ist $p = \sum_{i=0}^\infty a_i X^i \neq 0$, also $p = X^{\Deg p} p'$ für $p' = \sum_{i=0}^\infty a_{i+\Deg p} X^i$ mit $a_{\Deg p} \neq 0$. Nach \emph{(\ref{item: körper invertierbar})} ist $p'$ invertierbar, also $p$ assoziiert zu $X^{\Deg p}$. Analog ergibt sich, dass $q$ assoziiert zu $X^{\Deg q}$ ist. Mit $\Deg p = \Deg q$ folgt damit auch die Assoziiertheit von $p$ und $q$.
 \subsubsection*{\emph{(iii)}}
 Ist $\Deg p = \infty$, so ist $p = 0$ und nichts zu zeigen. Ansonsten ist $p = X^{\Deg p - \Deg q} p'$ wobei $p'$ assozziert zu $q$ ist, also $p = X^{\Deg p - \Deg q} c q$ für $c \in K^*$.
\end{proof}


\subsection{}
$f$ ist in $\Z[X]$ nicht irreduzibel, da $f = (X+1)(X+2)$.

Seien $p,q \in \Z[\![x]\!]$ mit $p = \sum_{i=0}^\infty a_i X^i$ und $q = \sum_{j=0}^\infty b_i X^i$ so dass $pq = f$. Dann ergibt sich durch Koeffizientenvergleich, dass $a_0 b_0 = 2$. Da $a_0, b_0 \in \Z$, und $2 \in \Z$ irreduzibel ist, ist $a_0$ oder $b_0$ eine Einheit. Entsprechend ist $p$ oder $q$ eine Einheit. Also ist $f$ irreduzibel in $\Z[\![x]\!]$.





\section{}
\begin{lem}\label{lem: K[[x]] euklidisch}
 $K[\![x]\!]$ bildet mit der Gradabbildung $\Deg$ einen euklidischen Ring.
\end{lem}
\begin{proof}
 Da $K$ nullteilerfrei ist, ist $K[\![x]\!]$ ein Integritätsring. Seien $f,g \in K[\![x]\!]$ mit $g \neq 0$. Es gilt zu zeigen, dass es $q,r \in K[\![x]\!]$ gibt, so dass $f = qg + r$ mit $r = 0$ oder $\Deg r < \Deg g$. Ist $\Deg f < \Deg g$ so genügt es $q=0$ und $r=f$ zu wählen. Ist $\Deg f \geq \Deg g$, so folgt aus \ref{lem: starkes Lemma}, dass $g \mid f$, es kann also $q$ mit $f = qg$ und $r=0$ gewählt werden.
\end{proof}

Aus Lemma \ref{lem: K[[x]] euklidisch} folgt direkt, dass $K[\![x]\!]$ ein Hautidealring ist. Für jedes Ideal $(a) \neq 0$ von $K[\![x]\!]$ folgt mit Lemma \ref{lem: starkes Lemma}, dass $a$ assoziert zu $X^{\Deg a}$ ist, und da $K[\![x]\!]$ ein Integritätsring ist, daher $(a) = \left(X^{\Deg a}\right)$. Folglich sind die Ideale in $K[\![x]\!]$ gerade $0$ und $\left(X^n\right)$ für $n \in \N$. Insbesondere ist $(X)$ das eindeutige maximale Ideal in $K[\![x]\!]$, weshalb $K[\![x]\!]$ lokal ist (dies lässt sich auch direkt aus Lemma \ref{lem: starkes Lemma} folgern).









\end{document}
