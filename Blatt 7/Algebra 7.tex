\documentclass[a4paper,10pt]{article}
%\documentclass[a4paper,10pt]{scrartcl}

\usepackage{xltxtra}
\usepackage[ngerman]{babel}
\usepackage{amsmath}
\usepackage{amssymb}
\usepackage{amsthm}
\usepackage{mathtools}
\usepackage{nicefrac}
\usepackage{enumerate}
\usepackage{leftidx}



\newcounter{satze}
\newtheorem{beh}[satze]{Behauptung}
\newtheorem{bem}[satze]{Bemerkung}
\newtheorem{lem}[satze]{Lemma}
\newtheorem*{defi}{Definition}
\newtheorem*{anm}{Anmerkung}
\theoremstyle{definition}
\newtheorem*{ia}{Induktionsanfang}
\newtheorem*{is}{Induktionsschritt}

\renewcommand{\thesection}{Aufgabe 7.\arabic{section}.}
\renewcommand{\thesubsection}{(\roman{subsection})}
\renewcommand{\thesubsubsection}{}

\newcommand{\N}{\mathbb{N}}
\newcommand{\Z}{\mathbb{Z}}
\newcommand{\Q}{\mathbb{Q}}
\newcommand{\R}{\mathbb{R}}
\newcommand{\C}{\mathbb{C}}
\newcommand{\Sn}{\mathfrak{S}}
\newcommand{\mf}[1]{\mathfrak{#1}}
\newcommand{\mc}[1]{\mathcal{#1}}
\newcommand{\dt}{\,\text{d}t}
\newcommand{\F}[1]{\mathbb{F}_{#1}}
\newcommand{\id}{\operatorname{id}}
\newcommand{\ord}{\operatorname{ord}}
\newcommand{\inn}{\operatorname{inn}}
\newcommand{\sgn}{\operatorname{sgn}}
\newcommand{\kchar}{\operatorname{char}}
\newcommand{\kgV}{\operatorname{kgV}}
\newcommand{\ggT}{\operatorname{ggT}}
\newcommand{\nil}{\operatorname{nil}}
\newcommand{\Id}{\operatorname{Id}}
\newcommand{\GL}[2]{\operatorname{GL}(#1,#2)}
\newcommand{\Img}{\operatorname{Im}}
\newcommand{\Ker}{\operatorname{Ker}}
\newcommand{\Hom}{\operatorname{Hom}}
\newcommand{\End}{\operatorname{End}}
\newcommand{\Aut}{\operatorname{Aut}}
\newcommand{\Inn}{\operatorname{Inn}}
\newcommand{\vect}[1]{\begin{pmatrix}#1\end{pmatrix}}
\newcommand{\gen}[1]{\left\langle#1\right\rangle}
\newcommand{\lb}{[\![}
\newcommand{\rb}{]\!]}
\newcommand{\Deg}{\operatorname{Deg}}

\newenvironment{lgs}[1][c]{\left\{\setlength{\arraycolsep}{1pt}\begin{array}{#1}}{\end{array}\right.}

\renewcommand*{\arraystretch}{1.5}

\makeatletter
\renewcommand*\env@matrix[1][*\c@MaxMatrixCols c]{%
  \hskip -\arraycolsep
  \let\@ifnextchar\new@ifnextchar
  \array{#1}}
\makeatother

\setromanfont[Mapping=tex-text]{Linux Libertine O}
% \setsansfont[Mapping=tex-text]{DejaVu Sans}
% \setmonofont[Mapping=tex-text]{DejaVu Sans Mono}
\parindent0pt

\title{\textsc{Einführung in die Algebra \\ \Large Blatt 7}}
\author{Jendrik Stelzner}
\date{\today}

\begin{document}
\maketitle





\section{}
Für $x,y \in \C$ mit $xy = 1$ muss $|x||y| = 1$, also $|x| \leq 1$ oder $|y| \leq 1$. Die einzigen Elemente $z \in \Z\left[\sqrt{-n}\,\right]$ mit $|z| \leq 1$ sind $\{1,-1\}$ für $n>1$ und $\{1,-1,i,-i\}$ für $n=1$. Für $x,y \in \Z\left[\sqrt{-n}\,\right]$ mit $xy = 1$ ist also $x \in \{1,-1\}$ oder $y \in \{1,-1\}$ für $n > 1$ und $x \in \{1,-1,i,-i\}$ oder $y \in \{1,-1,i,-i\}$ für $n=1$. Da all diese Elemente in $\Z\left[\sqrt{-n}\,\right]$ invertierbar sind, ist daher
\[
 \left(\Z\left[\sqrt{-n}\,\right]\right)^* =
 \begin{cases}
  \{1,-1,i,-i\} & \text{ für } n = 1, \\
  \{1,-1\}      & \text{ für } n > 1.
 \end{cases}
\]





\section{}
Für $x,y \in \C$ mit $xy=21$ ist $|x||y| = 21$, also muss $|x| \leq \sqrt{21}$ oder $|y| \leq \sqrt{21}$. Es genügt daher die $a+\sqrt{5}bi = z \in \Z[\sqrt{-5}]$ mit $|z| \leq \sqrt{21}$, also $a^2 + 5b^2 \leq 21$ auf Teilbarkeit zu überprüfen. Da für jeden Teiler $z \in \Z[\sqrt{-5}]$ auch $-z, \bar{z}, -\bar{z} \in \Z[\sqrt{-5}]$ Teiler von $21$ sind, genügt es zudem auch die $a+\sqrt{5}bi \in \Z[\sqrt{-n}]$ mit $a,b \geq 0$ auf Teilbarkeit zu überprüfen.

Es ergeben sich mit diesen beiden Beschränkungen die möglichen Kandidaten
\[
 1, 2, 3, 4, 1+\sqrt{5}i, 1+2\sqrt{5}i, 2+\sqrt{5}i, 3+\sqrt{5}i \text{ und } 4+\sqrt{5}i.
\]
Einfaches Hinsehen und gegebenenfalls kurzes Nachrechnen ergibt, dass von diesen Zahlen nur
\[
 1, 3, 1+2\sqrt{5}i \text{ und } 4+\sqrt{5}i
\]
Teiler von $21$ sind. Die Teiler von $21$ in $\Z\left[\sqrt{-n}\,\right]$, die sich nun nach den obigen Zusammenhängen ergeben, sind also
\begin{gather*}
 1, -1, 21, -21, 3, -3, 7, -7, \\
 1+2\sqrt{5}i, -1-2\sqrt{5}i, 1-2\sqrt{5}i, -1+2\sqrt{5}i, \\
 4+\sqrt{5}i, -4-\sqrt{5}i, 4-\sqrt{5}i, -4-\sqrt{5}i.
\end{gather*}





\section{}

\begin{defi}
 Für einen Ring $R$ bezeichnet
 \[
  \nil(R) := \{x \in R : x^n = 0 \text{ für ein } n \in \N\}
 \]
 das Nilradikal von $R$.
\end{defi}

\begin{bem}\label{bem: nilradikal}
 Sei $R$ ein kommutativer Ring. Dann gilt
 \begin{enumerate}[(i)]
  \item $\nil(R)$ ist ein Ideal von $R$.
  \item Für $e \in R^*$ und $a \in \nil(R)$ ist $e+a \in R^*$.
 \end{enumerate}
\end{bem}

\begin{proof}
 \subsubsection*{(i)}
  Es ist $0 \in \nil(R)$, also $\nil(R)$ nicht leer. Für $a,b \in \nil(R)$ gibt es $n,m \in \N$ mit $a^n = b^m = 0$, weshalb
  \[
   (a+b)^{n+m} = \sum_{k=0}^{n+m} \binom{n+m}{k} a^{n+m-k} b^k = 0,
  \]
  und deshalb $a+b \in \nil(R)$. Auch ist für alle $r \in R$
  \[
   (ra)^n = r^n a^n = 0,
  \]
  also $ra \in R$. Insbesondere ist daher für alle $a \in \nil(R)$ auch $-a = (-1) \cdot a \in R$.
 \subsubsection*{(ii)}
  Für $e \in R^*$ und $a \in \nil(R)$ mit $a^n = 0$ ist $1+e^{-1}a \in R^*$, da $\left(e^{-1}a\right)^n = 0$ und daher
  \[
   \left(\sum_{k=0}^{n-1} \left(-e^{-1}a\right)^k\right) \left(1+e^{-1}a\right)
   = 1+ (-1)^{n-1} \left(e^{-1}a\right)^n
   = 1.
  \]
  Daher ist auch $e+a = e\left(1+e^{-1}a\right) \in R^*$.
\end{proof}

Da $\nil(R) \subseteq \nil(R[X])$ ist auch $\left(\nil(R)\right) \subseteq \nil(R[X])$. Dabei ist, wie in einem früheren Übungsblatt gezeigt,
\[
 \left(\nil(R)\right)
 = \left\{f \in R[X] : f = \sum_{i=0}^n a_i X^i \text{ mit } n \geq 0, a_i \in \nil(R) \text{ für alle } i\right\}.
\]
Nach Bemerkung \ref{bem: nilradikal} ist also das Polynom $f = \sum_{i=0}^n a_i X^i$ mit $n \geq 0$, $a_0 \in R^*$ und $a_i \in \nil(R)$ für alle $i > 0$ invertierbar.

Sei andererseits $f = \sum_{i=0}^n a_i X^i \in R[X]$, mit $n \geq 0$ und $a_n \neq 0$, invertierbar, d.h. es gibt $g = \sum_{i=0}^m b_i X^i \in R[X]$, mit $m \geq 0$ und $b_m \neq 0$, so dass $fg = 1$. Da insbesondere $a_0 b_0 = 1$ müssen $a_0$ und $b_0$ in $R$ invertierbar sein. Für $n=0$ ist nichts weiter zu zeigen. Ist $n > 0$, so bemerken wir:

\begin{beh}\label{beh: a_n nilpotent}
 Es ist $a_n^{k+1} b_{m-k} = 0$ für $k=0,\ldots,m$.
\end{beh}
\begin{proof}
 Der Beweis verläuft per Induktion über $k$.
 \begin{ia}Betrachte $k=0$. Wäre $a_n b_m \neq 0$, so wäre
  \[
   0 = \deg(1) = \deg(fg) = n+m \geq n > 0.
  \]
 \end{ia}
 \begin{is}
 Sei $1 \leq k \leq n$ und gelte die Aussage für $0,\ldots,k-1$. Da $fg = 1$ ergibt sich für den $n+m-k$-ten Koeffizienten des Produktes $fg$, dass
 \[
  0 = \sum_{\mu+\nu = n+m-k} a_\mu b_\nu.
 \]
 Multiplikation der Gleichung mit $a_n^k$ ergibt
 \[
  0 = \sum_{\mu+\nu = n+m-k} a_n^k a_\mu b_\nu \underset{\text{IV.}}= a_n^{k+1} b_{m-k}.
 \]
 \end{is}
\end{proof}
Aus Behauptung \ref{beh: a_n nilpotent} folgt insbesondere, dass $a_n^{m+1} b_0 = 0$. Da $b_0$ invertierbar ist, ist $a_n^{m+1} = 0$, also $a_n$ nilpotent. Da nach Bemerkung \ref{bem: nilradikal} daher auch $f- a_n X^n$ invertierbar ist, ergibt sich durch Wiederholung der obigen Argumentation induktiv, dass $a_i$ für alle $1 \leq i \leq n$ nilpotent ist.





\section{}

\begin{defi}
Sei $R$ ein kommutativer Ring. Für $f = \sum_{i=0}^\infty a_i X^i \in R[\![x]\!]$ bezeichnet
\[
 \Deg(f) :=
 \begin{cases}
  \min \{i \in \N : a_i \neq 0\} & \text{falls } f \neq 0, \\
                          \infty & \text{falls } f = 0
 \end{cases}
\]
den \emph{Grad} von $f$.
\end{defi}
\begin{bem}\label{bem: grad potenzreihe}
 Für einen kommutativen Ring $R$ und $f,g \in R[\![x]\!]$ ist
 \begin{align*}
  \Deg(f+g) &\geq \min \{\Deg(f), \Deg(g)\} \text{ und } \\
  \Deg(fg)  &\geq \Deg(f) + \Deg(g).
 \end{align*}
 Ist $R$ darüber hinaus nullteilerfrei, so gilt sogar
 \begin{equation*}
  \Deg(fg) = \Deg(f) + \Deg(g).
 \end{equation*}
\end{bem}
\begin{proof}
 Der Beweis läuft analog zu dem der entsprechenden (Un)gleichungen der Gradfunktion $\deg$ von $R[X]$.
\end{proof}



\subsection{}
Ist $R$ kein Integritätsring, so auch $R[\![x]\!]$ nicht, da man $R$ in kanonischer Weise als Unterring von $R[\![x]\!]$ auffassen kann. Ist $R[\![x]\!]$ kein Integritätsring, so gibt es $f, g \in R[\![x]\!]$ mit $f,g \neq 0$, also $\Deg(f), \Deg(g) < \infty$, aber $fg = 0$, also $\Deg(fg) = \infty$. Aus Bemerkung \ref{bem: grad potenzreihe} folgt, dass $R$ kein Integritätsring ist.


\subsection{}
Ist $f = \sum_{i=0}^\infty a_i X^i \in R[\![x]\!]$ invertierbar, so gibt es $g = \sum_{i=0}^\infty b_i X^i \in R[\![x]\!]$ mit $fg = 1$. Insbesondere ergibt sich für den $0$-ten Koeffizienten von $fg$, dass $1 = a_0 b_0$. Also ist $a_0$ invertierbar in $R$.

Ist $f = \sum_{i=0}^\infty a_i X^i \in R[\![x]\!]$ mit $a_0$ invertierbar, so definieren wir eine Folge $(b_i)_{i \in \N}$ auf $R$ rekursiv durch
\[
 b_0 := a_0^{-1} \text{ und } b_i := -a_0^{-1} \sum_{j=1}^i a_j b_{i-j},
\]
und definieren $g := \sum_{i=0}^\infty b_i X^i$ als die entsprechende Potenzreihe. Es ergibt sich für \mbox{$fg =: e = \sum_{i=0}^\infty e_i X^i$}, dass $e_0 = a_0 b_0 = 1$ und für alle $i \geq 1$
\[
 e_i
 = \sum_{j=0}^i a_j b_{i-j}
 = \sum_{j=1}^i a_j b_{i-j} + a_0 b_i
 = \sum_{j=1}^i a_j b_{i-j} - \sum_{j=1}^i a_j b_{i-j}
 = 0.
\]
Also ist $e = 1$ und $f$ daher invertierbar mit $f^{-1} = g$. Inbesondere ergibt sich das folgende Lemma:

\begin{lem} \label{lem: starkes Lemma}
 Sei $K$ ein Körper und seien $f,g \in K[\![x]\!]$. Dann gilt:
 \begin{enumerate}[(i)]
  \item $f$ ist genau dann invertierbar, wenn $\Deg f = 0$. \label{item: körper invertierbar}
  \item Ist $\Deg f = \Deg g$, so sind $f$ und $g$ assoziiert. Genauer: Ist $\Deg f < \infty$, so ist $f$ assoziiert zu $X^{\Deg f}$.
  \item Ist $\Deg f \leq \Deg g$, so ist $f \mid g$.
 \end{enumerate}
\end{lem}
\begin{proof}
 \subsubsection*{\emph{(i)}}
  $f = \sum_{i=0}^{\infty} a_i X^i$ ist genau dann invertierbar, wenn $a_0$ invertierbar ist, also genau dann wenn $a_0 \neq 0$, was äquivalent zu $\Deg f = 0$ ist.
 \subsubsection*{\emph{(ii)}}
  Ist $f = g = 0$ so ist nichts zu zeigen. Ansonsten ist $f = \sum_{i=0}^\infty a_i X^i \neq 0$, also $f = X^{\Deg f} f'$ mit $f' = \sum_{i=0}^\infty a_{i+\Deg f} X^i$ und $a_{\Deg f} \neq 0$. Nach \emph{(\ref{item: körper invertierbar})} ist $f'$ invertierbar, also $f$ assoziiert zu $X^{\Deg f}$.
 \subsubsection*{\emph{(iii)}}
 Ist $\Deg g = \infty$, so ist $g = 0$ und nichts zu zeigen. Ansonsten ist $g = X^{\Deg g - \Deg f} g'$, wobei $g'$ wegen $\Deg g' = \Deg f$ assozziert zu $f$ ist, also $g = X^{\Deg g - \Deg f} c f$ für $c \in K^*$.
\end{proof}


\subsection{}
$f$ ist in $\Z[X]$ nicht irreduzibel, da $f = (X+1)(X+2)$.

Seien $p,q \in \Z[\![x]\!]$ mit $p = \sum_{i=0}^\infty a_i X^i$ und $q = \sum_{j=0}^\infty b_i X^i$ so dass $pq = f$. Dann ist $a_0 b_0 = 2$. Da $a_0, b_0 \in \Z$, und $2$ in $\Z$ irreduzibel ist, ist $a_0$ oder $b_0$ eine Einheit in $\Z$. Entsprechend ist $p$ oder $q$ nach dem vorherigen Aufgabenteil eine Einheit. Also ist $f$ irreduzibel in $\Z[\![x]\!]$.





\section{}
\begin{lem}\label{lem: K[[x]] euklidisch}
 $K[\![x]\!]$ bildet mit der Gradabbildung $\Deg$ einen euklidischen Ring.
\end{lem}
\begin{proof}
 Da $K$ nullteilerfrei ist, ist $K[\![x]\!]$ ein Integritätsring. Seien $f,g \in K[\![x]\!]$ mit $g \neq 0$. Es gilt zu zeigen, dass es $q,r \in K[\![x]\!]$ gibt, so dass $f = qg + r$ mit $r = 0$ oder $\Deg r < \Deg g$.
 
 Ist $\Deg f < \Deg g$ so genügt es $q=0$ und $r=f$ zu wählen. Ist $\Deg f \geq \Deg g$, so folgt aus Lemma \ref{lem: starkes Lemma}, dass $g \mid f$, es kann also $q$ mit $f = qg$ und $r=0$ gewählt werden.
\end{proof}

Aus Lemma \ref{lem: K[[x]] euklidisch} folgt direkt, dass $K[\![x]\!]$ ein Hauptidealring ist. Für jedes Ideal $(a) \neq 0$ von $K[\![x]\!]$ folgt mit Lemma \ref{lem: starkes Lemma}, dass $a$ assoziert zu $X^{\Deg a}$ ist, und da $K[\![x]\!]$ ein Integritätsring ist, daher $(a) = \left(X^{\Deg a}\right)$. Folglich sind die Ideale in $K[\![x]\!]$ gerade $0$ und $\left(X^n\right)$ für $n \in \N$. Insbesondere ist $(X)$ das eindeutige maximale Ideal in $K[\![x]\!]$, weshalb $K[\![x]\!]$ lokal ist (dies lässt sich auch direkt aus Lemma \ref{lem: starkes Lemma} folgern).









\end{document}
