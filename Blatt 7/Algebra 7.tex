\documentclass[a4paper,10pt]{article}
%\documentclass[a4paper,10pt]{scrartcl}

\usepackage{xltxtra}
\usepackage[ngerman]{babel}
\usepackage{amsmath}
\usepackage{amssymb}
\usepackage{amsthm}
\usepackage{mathtools}
\usepackage{nicefrac}
\usepackage{enumerate}
\usepackage{leftidx}



\newcounter{satze}
\newtheorem{beh}[satze]{Behauptung}
\newtheorem{bem}[satze]{Bemerkung}
\newtheorem{lem}[satze]{Lemma}
\newtheorem*{defi}{Definition}
\newtheorem*{anm}{Anmerkung}
\newtheorem*{ia}{Induktionsanfang}
\newtheorem*{is}{Induktionsschritt}

\renewcommand{\thesection}{Aufgabe 7.\arabic{section}.}
\renewcommand{\thesubsection}{(\roman{subsection})}
\renewcommand{\thesubsubsection}{}

\newcommand{\N}{\mathbb{N}}
\newcommand{\Z}{\mathbb{Z}}
\newcommand{\Q}{\mathbb{Q}}
\newcommand{\R}{\mathbb{R}}
\newcommand{\C}{\mathbb{C}}
\newcommand{\Sn}{\mathfrak{S}}
\newcommand{\mf}[1]{\mathfrak{#1}}
\newcommand{\mc}[1]{\mathcal{#1}}
\newcommand{\dt}{\,\text{d}t}
\newcommand{\F}[1]{\mathbb{F}_{#1}}
\newcommand{\id}{\operatorname{id}}
\newcommand{\ord}{\operatorname{ord}}
\newcommand{\inn}{\operatorname{inn}}
\newcommand{\sgn}{\operatorname{sgn}}
\newcommand{\kchar}{\operatorname{char}}
\newcommand{\kgV}{\operatorname{kgV}}
\newcommand{\ggT}{\operatorname{ggT}}
\newcommand{\Id}{\operatorname{Id}}
\newcommand{\GL}[2]{\operatorname{GL}(#1,#2)}
\newcommand{\Img}{\operatorname{Im}}
\newcommand{\Ker}{\operatorname{Ker}}
\newcommand{\Hom}{\operatorname{Hom}}
\newcommand{\End}{\operatorname{End}}
\newcommand{\Aut}{\operatorname{Aut}}
\newcommand{\Inn}{\operatorname{Inn}}
\newcommand{\vect}[1]{\begin{pmatrix}#1\end{pmatrix}}
\newcommand{\gen}[1]{\left\langle#1\right\rangle}
\newcommand{\lb}{[\![}
\newcommand{\rb}{]\!]}
\newcommand{\Deg}{\operatorname{Deg}}

\newenvironment{lgs}[1][c]{\left\{\setlength{\arraycolsep}{1pt}\begin{array}{#1}}{\end{array}\right.}

\renewcommand*{\arraystretch}{1.5}

\makeatletter
\renewcommand*\env@matrix[1][*\c@MaxMatrixCols c]{%
  \hskip -\arraycolsep
  \let\@ifnextchar\new@ifnextchar
  \array{#1}}
\makeatother

\setromanfont[Mapping=tex-text]{Linux Libertine O}
% \setsansfont[Mapping=tex-text]{DejaVu Sans}
% \setmonofont[Mapping=tex-text]{DejaVu Sans Mono}
\parindent0pt

\title{\textsc{Einführung in die Algebra \\ \Large Blatt 7}}
\author{Jendrik Stelzner}
\date{\today}

\begin{document}
\maketitle





\section{}





\section{}





\section{}





\section{}

\begin{defi}
Sei $R$ ein kommutativer Ring. Für $p = \sum_{i=0}^\infty a_i X^i \in R[\![x]\!]$ bezeichnet
\[
 \Deg(p) :=
 \begin{cases}
  \min \{i \in \N : a_i \neq 0\} & \text{falls } f \neq 0, \\
                          \infty & \text{sonst}.
 \end{cases}
\]
den Grad von $p$.
\end{defi}
Für einen kommutativen Ring $R$ und $p,q \in R[\![x]\!]$ ist
\begin{equation}\label{eq: Deg Ungleichungen}
 \Deg(p+q) \geq \min \{\Deg(p), \Deg(q)\} \text{ und }
 \Deg(pq) \geq \Deg(p) + \Deg(q).
\end{equation}
Ist $R$ nullteilerfrei, so gilt sogar
\begin{equation}\label{eq: Deg nullteilerfrei}
 \Deg(pq) = \Deg(p) + \Deg(q).
\end{equation}
Die Beweise der entsprechenden Aussage laufen analog zu den Beweisen der entsprechenden Aussagen für die Gradfunktion $\deg$ von $R[X]$.


\subsection{}
Ist $R$ kein Integritätsring, so ist auch $R[X] \subsetneq R[\![x]\!]$ kein Integritätsring, also auch $R[\![x]\!]$ nicht. Ist $R[\![x]\!]$ kein Integritätsring, so gibt es $p, q \in R[\![x]\!]$ mit $p,q \neq 0$, also $\Deg(p), \Deg(q) < \infty$, aber $pq = 0$, also $\Deg(pq) = \infty$. Mit \eqref{eq: Deg nullteilerfrei} folgt, dass $R$ kein Integritätsring ist.


\subsection{}
Ist $p = \sum_{i=0}^\infty a_i X^i \in R[\![x]\!]$ invertierbar, so gibt es $q = \sum_{i=0}^\infty b_i X^i \in R[\![x]\!]$ mit $pq = 1$. Insbesondere ist daher
\[
 1 = (pq)_1 = a_0 b_0,
\]
also $a_0$ invertierbar.

Ist $p = \sum_{i=0}^\infty a_i X^i \in R[\![x]\!]$ mit $a_0$ invertierbar, so definieren wir eine Folge $(b_i)_{i \in \N}$ auf $R$ rekursiv durch
\[
 b_0 := a_0^{-1} \text{ und } b_i := -a_0^{-1} \sum_{j=1}^i a_j b_{n-j},
\]
und $q := \sum_{i=0}^\infty b_i X^i$ als die entsprechende Potenzreihe. Für $e = pq$ ergibt sich dann für alle $i \in \N$
\[
 e_i
 = \sum_{j=0}^i a_j b_{i-j}
 = \sum_{j=1}^i a_j b_{i-j} + a_0 b_i
 = \sum_{j=1}^i a_j b_{i-j} - \sum_{j=1}^i a_j b_{n-j}
 = 0.
\]
Also ist $e = 1$ und $p$ daher invertierbar mit $p^{-1} = q$. Inbesondere ergibt sich das folgende Lemma:

\begin{lem} \label{lem: starkes Lemma}
 Sei $K$ ein Körper und seien $p,q \in K[\![x]\!]$. Dann gilt:
 \begin{enumerate}
  \item $p$ ist genau dann invertierbar, wenn $\Deg p = 0$.
  \item Ist $\Deg p = \Deg q$, so sind $p$ und $q$ assoziiert. Ist $\Deg p = \Deg q < \infty$, so sind $p$ und $q$ assoziiert zu $X^{\Deg p}$.
  \item Ist $\Deg p \geq \Deg q$, so ist $q \mid p$.
 \end{enumerate}
\end{lem}
\begin{proof}
 \subsubsection*{\textit{(i)}}
  $p = \sum_{i=0}^{\infty} a_i X^i$ ist genau dann invertierbar, wenn $a_0$ invertierbar ist, also genau dann wenn $a_0 \neq 0$, was wiederum äquivalent zu $\Deg a_0 = 0$ ist.
 \subsubsection*{\textit{(ii)}}
  Ist $p = q = 0$ so ist nichts zu zeigen. Ansonsten ist $p = \sum_{i=0}^\infty a_i X^i \neq 0$, also $p = X^{\Deg p} p'$ für $p' = \sum_{i=0}^\infty a_{i+\Deg p} X^i$ mit $a_{\Deg p} \neq 0$. Nach \textit{(i)} ist $p'$ invertierbar, also $p$ assoziiert zu $X^{\Deg p}$. Analog ergibt sich, dass $q$ assoziiert zu $X^{\Deg q}$ ist. Mit $\Deg p = \Deg q$ folt damit auch die Assoziiertheit von $p$ und $q$.
 \subsubsection*{\textit{(iii)}}
 Ist $\Deg p = \infty$, so ist $p = 0$ und nichts zu zeigen. Ansonsten ist $p = X^{\Deg p - \Deg q} p'$ wobei $p'$ assozziert zu $q$ ist, also $p = X^{\Deg p - \Deg q} c q$ für $c \in K^*$.
\end{proof}


\subsection{}
$f$ ist in $\Z[X]$ nicht irreduzibel, da $f = (X+1)(X+2)$.

Seien $p,q \in \Z[\![x]\!]$ mit $p = \sum_{i=0}^\infty a_i X^i$ und $q = \sum_{j=0}^\infty b_i X^i$ so dass $pq = f$. Dann ergibt sich durch Koeffizientenvergleich, dass $a_0 b_0 = 2$. Da $a_0, b_0 \in \Z$, und $2 \in \Z$ irreduzibel ist, ist $a_0$ oder $b_0$ eine Einheit. Entsprechend ist $p$ oder $q$ eine Einheit. Also ist $f$ irreduzibel in $\Z[\![x]\!]$.








\end{document}
