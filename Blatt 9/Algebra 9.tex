\documentclass[a4paper,10pt]{article}
%\documentclass[a4paper,10pt]{scrartcl}

\usepackage{xltxtra}
\usepackage[ngerman]{babel}
\usepackage{amsmath}
\usepackage{amssymb}
\usepackage{amsthm}
\usepackage{mathtools}
\usepackage{nicefrac}
\usepackage{enumerate}
\usepackage{leftidx}
\usepackage{tikz}

\newcounter{satze}
\newtheorem{beh}[satze]{Behauptung}
\newtheorem{bem}[satze]{Bemerkung}
\newtheorem{lem}[satze]{Lemma}
\newtheorem*{defi}{Definition}
\newtheorem*{anm}{Anmerkung}
\theoremstyle{definition}
\newtheorem*{ia}{Induktionsanfang}
\newtheorem*{is}{Induktionsschritt}

\renewcommand{\thesection}{Aufgabe 9.\arabic{section}.}
\renewcommand{\thesubsection}{(\roman{subsection})}
\renewcommand{\thesubsubsection}{}

\newcommand{\N}{\mathbb{N}}
\newcommand{\Z}{\mathbb{Z}}
\newcommand{\Q}{\mathbb{Q}}
\newcommand{\R}{\mathbb{R}}
\newcommand{\C}{\mathbb{C}}
\newcommand{\Sn}{\mathfrak{S}}
\newcommand{\mf}[1]{\mathfrak{#1}}
\newcommand{\mc}[1]{\mathcal{#1}}
\newcommand{\dt}{\,\text{d}t}
\newcommand{\F}{\mathbb{F}}
\newcommand{\id}{\operatorname{id}}
\newcommand{\ord}{\operatorname{ord}}
\newcommand{\inn}{\operatorname{inn}}
\newcommand{\sgn}{\operatorname{sgn}}
\newcommand{\kchar}{\operatorname{char}}
\newcommand{\kgV}{\operatorname{kgV}}
\newcommand{\ggT}{\operatorname{ggT}}
\newcommand{\nil}{\operatorname{nil}}
\newcommand{\Id}{\operatorname{Id}}
\newcommand{\GL}[2]{\operatorname{GL}(#1,#2)}
\newcommand{\Img}{\operatorname{Im}}
\newcommand{\Ker}{\operatorname{Ker}}
\newcommand{\Hom}{\operatorname{Hom}}
\newcommand{\End}{\operatorname{End}}
\newcommand{\Aut}{\operatorname{Aut}}
\newcommand{\Inn}{\operatorname{Inn}}
\newcommand{\vect}[1]{\begin{pmatrix}#1\end{pmatrix}}
\newcommand{\gen}[1]{\left\langle#1\right\rangle}
\newcommand{\lb}{[\![}
\newcommand{\rb}{]\!]}
\newcommand{\Deg}{\operatorname{Deg}}

\newenvironment{lgs}[1][c]{\left\{\setlength{\arraycolsep}{1pt}\begin{array}{#1}}{\end{array}\right.}

\renewcommand*{\arraystretch}{1.5}

\makeatletter
\renewcommand*\env@matrix[1][*\c@MaxMatrixCols c]{%
  \hskip -\arraycolsep
  \let\@ifnextchar\new@ifnextchar
  \array{#1}}
\makeatother

\setromanfont[Mapping=tex-text]{Linux Libertine O}
% \setsansfont[Mapping=tex-text]{DejaVu Sans}
% \setmonofont[Mapping=tex-text]{DejaVu Sans Mono}
\parindent0pt

\title{\textsc{Einführung in die Algebra \\ \Large Blatt 9}}
\author{Jendrik Stelzner}
\date{\today}

\begin{document}
\maketitle





\section{}
Die Multiplikation $\Z \times \Q \rightarrow \Q$ von $\Q$ als $\Z$-Modul ist offenbar die Einschränkung der Multiplikation $\Q \times \Q \rightarrow \Q$ des Körpers $\Q$. Die Torsionsfreiheit von $\Q$ als $\Z$-Modul folgt daher direkt aus der Nullteilerfreiheit von $\Q$ als Köper.

$\Q$ ist als $\Z$-Modul nicht endlich erzeugt, denn für $\frac{r_1}{s_1}, \ldots, \frac{r_n}{s_n} \in \Q$ ist offenbar
\begin{align*}
 \Z \frac{r_1}{s_1} + \ldots + \Z \frac{r_n}{s_n}
 &\subseteq \Z \frac{1}{s_1} + \ldots + \Z \frac{1}{s_n} \\
 &\subseteq \Z \frac{1}{s_1 \cdots s_n} + \ldots + \Z \frac{1}{s_1 \cdots s_n} 
 \subseteq \Z \frac{1}{s_1 \cdots s_n}
 \subsetneq \Q.
\end{align*}


$\Q$ ist als $\Z$-Modul nicht frei: Für $\frac{p}{q}, \frac{r}{s} \in \Q$ besitzt
\[
 rq \frac{p}{q} = pr = ps \frac{r}{s}
\]
zwei unterschiedliche Linearkombinationen. Daher ist jede Familie von mindestens zwei rationalen Zahlen linear abhängig, insbesondere also jedes Erzeugendensystem von $\Q$ als $\Z$-Modul, da ein solches unendlich ist.





\section{}


\subsection{}
Es handelt sich bei $\sim$ um eine Äquivalenzrelation: Für  alle $(m,s) \in M \times S$ ist \mbox{$1 \cdot ms = 1 \cdot ms$} mit $1 \in S$ und deshalb $(m,s) \sim (m,s)$. Also ist $\sim$ reflexiv. Ist $(m,s) \tilde (m', s')$ für $(m,s), (m',s') \in M \times S$, so gibt es ein $t \in S$ mit $ts'm = tsm'$. Da damit auch $tsm' = ts'm$ ist dann $(m',s') \sim (m,s)$. Also ist $\sim$ symmetrisch. Ist $(m,s) \tilde (m',s') \tilde (m'', s'')$, so gibt es $t, \tilde{t} \in S$ mit $ts'm = tsm'$ und $\tilde{t}s'm'' = \tilde{t}s''m'$. Da $S$ unter der Multiplikation abgeschlossen ist, ist auch $t\tilde{t}s' \in S$. Da wegen der Kommutativität von $R$
\[
 t\tilde{t}s' \cdot s''m
 = \tilde{t}s'' \cdot ts'm
 = \tilde{t}s'' \cdot tsm'
 = ts \cdot \tilde{t}s''m'
 = ts \cdot \tilde{t}s'm''
 = t\tilde{t}s' \cdot sm''
\]
ist $(m,s) \sim (m'',s'')$. Daher ist $\sim$ transitiv.

Die Addition ist wohldefiniert: Für $\frac{m'}{s'}, \frac{\tilde{m}}{\tilde{s}} \in M[S^{-1}]$ mit $\frac{m'}{s'} = \frac{\tilde{m}}{\tilde{s}}$ gibt es ein $t \in S$ mit $t\tilde{s}m' = ts'\tilde{m}$. Wegen der Kommutativität von $R$ ist dann für alle $(m,s) \in M \times S$
\[
 ts\tilde{s}(s'm + sm')
 = ts\tilde{s}s'm + ts^2\tilde{s}m'
 = ts\tilde{s}s'm + ts^2s'\tilde{m}
 = tss'(\tilde{s}m+s\tilde{m}),
\]
und daher $\frac{s'm+sm'}{ss'} = \frac{\tilde{s}m+s\tilde{m}}{s\tilde{s}}$. Da der Ausdruck $\frac{sm'+s'm}{ss'}$ wegen der Kommutativität von $R$ symmetrisch in $(m,s)$ und $(m',s')$ ist, folgt damit die Wohldefiniertheit der Addition.

Auch die Multiplikation ist wohldefiniert:Für $\frac{r}{s},\frac{\tilde{r}}{\tilde{s}} \in R[S^{-1}]$ mit $\frac{r}{s} = \frac{\tilde{r}}{\tilde{s}}$ gibt es ein $t \in S$ mit $tr\tilde{s} = t\tilde{r}s$, weshalb wegen der Kommutativität von $R$ für alle $(m',s') \in M \times S$
\[
 t \tilde{s}s' rm' = t ss' \tilde{r}m',
\]
also $\frac{rm'}{ss'} = \frac{\tilde{r}m'}{\tilde{s}s'}$. Für $\frac{m'}{s'}, \frac{\tilde{m}}{\tilde{s}} \in M[S^{-1}]$ mit $\frac{m'}{s'} = \frac{\tilde{m}}{\tilde{s}}$ gibt es ein $t \in S$ mit $t\tilde{s}m' = ts'\tilde{m}$, weshalb wegen der Kommutativität von $R$ für alle $(r,s) \in R \times S$
\[
 t s\tilde{s} rm' = t ss' r\tilde{m},
\]
also $\frac{rm'}{ss'} = \frac{r\tilde{m}}{s\tilde{s}}$. Dies zeigt die Wohldefiniertheit der Multiplikation.

$M[S^{-1}]$ bildet bezüglich der Addition ein abelsche Gruppe: Die Addition ist assoziativ, da für alle $\frac{m}{s}, \frac{m'}{s'}, \frac{m''}{s''} \in M[S^{-1}]$
\begin{align*}
 \frac{m}{s} + \left( \frac{m'}{s'} + \frac{m''}{s''} \right)
 &= \frac{m}{s} + \frac{s''m' + s'm''}{s's''}
 = \frac{s's''m + ss''m' + ss'm''}{ss's''} \\
 &= \frac{s'm + sm'}{ss'} + \frac{m''}{s''}
 = \left( \frac{m}{s} + \frac{m'}{s'} \right) + \frac{m''}{s''},
\end{align*}
und kommutativ, da für alle $\frac{m}{s}, \frac{m'}{s'} \in M[S^{-1}]$
\[
 \frac{m}{s} + \frac{m'}{s'}
 = \frac{ms' + m's}{ss'}
 = \frac{m's + ms'}{s's}
 = \frac{m'}{s'} + \frac{m}{s}.
\]
Das Element $\frac{0}{1} \in M[S^{-1}]$ ist bezüglich der Addition neutral, da für alle $\frac{m}{s} \in M[S^{-1}]$
\[
 \frac{m}{s} + \frac{0}{1} = \frac{1 \cdot m + s \cdot 0}{s \cdot 1} = \frac{m}{s}.
\]
Dabei ist klar, dass $\frac{0}{1} = \frac{0}{s}$ für alle $s \in S$. Jedes Element $\frac{m}{s} \in M[S^{-1}]$ hat $\frac{-m}{s} \in M[S^{-1}]$ als additiv inverses Element, da
\[
 \frac{m}{s} + \frac{-m}{s} = \frac{sm - sm}{s^2} = \frac{0}{s^2} = \frac{0}{1}.
\]
Dies zeigt, dass $M[S^{-1}]$ bezüglich der Addition eine abelsche Gruppe bildet.

Durch die definierte Multiplikation wird $M[S^{-1}]$ zu einem $R[S^{-1}]$-Modul: Seien im Folgenden $\frac{r}{s}, \frac{\tilde{r}}{\tilde{s}} \in R[S^{-1}]$ und $\frac{m'}{s'}, \frac{m''}{s''} \in M[S^{-1}]$  beliebig aber fest. Es ist
\[
 1_{R[S^{-1}]} \,\frac{m'}{s'}
 = \frac{1_R}{1_R} \, \frac{m'}{s'}
 = \frac{1_R \, m'}{1_R \, s'}
 = \frac{m'}{s'}.
\]
Daher gilt für alle $\hat{s} \in S$ die Kürzungsregel
\[
 \frac{\hat{s}m'}{\hat{s}s'}
 = \frac{\hat{s}}{\hat{s}} \, \frac{m'}{s'}
 = 1_{R[S^{-1}]} \, \frac{m'}{s'}
 = \frac{m'}{s'}.
\]
Es ist daher
\begin{align*}
 \frac{r}{s} \left( \frac{m'}{s'} + \frac{m''}{s''} \right)
 &= \frac{r}{s} \frac{s''m' + s'm''}{s's''}
 = \frac{rs''m' + rs'm''}{ss's''} \\
 &= \frac{rss''m' + rss'm''}{s^2s's''}
 = \frac{rm'}{ss'} + \frac{rm''}{ss''}
 = \frac{r}{s} \, \frac{m'}{s'} + \frac{r}{s} \, \frac{m''}{s''}.
\end{align*}
Auch ist deshalb
\begin{align*}
 \left(\frac{r}{s} + \frac{\tilde{r}}{\tilde{s}}\right) \frac{m'}{s'}
 &= \frac{r\tilde{s}+\tilde{r}s}{s\tilde{s}} \, \frac{m'}{s'}
 = \frac{r\tilde{s}m' + \tilde{r}sm'}{s\tilde{s}s'} \\
 &= \frac{r\tilde{s}s'm' + \tilde{r}ss'm'}{s\tilde{s}(s')^2}
 = \frac{rm'}{ss'} + \frac{\tilde{r}m'}{\tilde{s}s'}
 = \frac{r}{s} \, \frac{m'}{s'} + \frac{\tilde{r}}{\tilde{s}} \frac{m'}{s'}.
\end{align*}
Da auch
\[
 \frac{r}{s} \left(\frac{\tilde{r}}{\tilde{s}} \, \frac{m'}{s'}\right)
 = \frac{r}{s} \, \frac{\tilde{r} m'}{\tilde{s} s'}
 = \frac{r \tilde{r} m'}{s \tilde{s} s'}
 = \frac{r \tilde{r}}{s \tilde{s}} \, \frac{m'}{s'}
 = \left( \frac{r}{s} \, \frac{\tilde{r}}{\tilde{s}} \right) \frac{m'}{s'}
\]
ist $M[S^{-1}]$ bezüglich der definierten Multiplikation ein $R[S^{-1}]$-Modul.


\subsection{}
Da für jedes $s \in S$ die Abbildung $n \mapsto s n$ in $N$ bijektiv ist, gibt es für alle $n' \in N$ und $s \in S$ ein eindeutiges $n \in N$ mit $sn = n'$, für das wir im Folgenden $\frac{n'}{s}$ schreiben werden. Für alle $n \in N$ und $s \in S$ ist nach Definition ist $s \frac{n}{s} = n$.

\begin{beh}
 Es gelten die folgenden Rechenregeln:
 \begin{enumerate}[(i)]
  \item Für alle $n, n' \in N$ und $s, s' \in S$ ist $\frac{n}{s} = \frac{n'}{s'}$ genau dann wenn $s'n = sn'$. \label{enum: brüche gleich}
  \item Für alle $n, n' \in N$ und $s \in S$ ist $\frac{n+n'}{s} = \frac{n}{s} + \frac{n'}{s}$. \label{enum: zähler auseinander}
  \item Für alle $n, n' \in N$ und $s, s' \in S$ ist $\frac{n}{s} + \frac{n'}{s'} = \frac{s'n + sn'}{ss'}$. \label{enum: brüche addieren}
  \item Für alle $n \in N$, $s \in S$, und $r \in R$ ist $r\, \frac{n}{s} = \frac{rn}{s}$. \label{enum: zähler multiplizieren}
 \end{enumerate}
\end{beh}
\begin{proof}
 \subsubsection*{\emph{(\ref{enum: brüche gleich})}}
 Ist $\frac{n}{s} = \frac{n'}{s'}$, so ist
 \[
  s'n = s's\frac{n}{s} = s's\frac{n'}{s'} = ss'\frac{n'}{s'} = sn'.
 \]
 Gilt andererseits $sn' = s'n$, so ist
 \[
  s's \, \frac{n}{s} = s'n = sn' = ss'\frac{n'}{s'} = s's \, \frac{n'}{s'},
 \]
 wegen der Injektivität der Multiplikation mit $s's \in S$ also $\frac{n}{s} = \frac{n'}{s'}$.
 \subsubsection*{\emph{(\ref{enum: zähler auseinander})}}
 Es ist
 \[
  s\left( \frac{n}{s} + \frac{n'}{s} \right)
  = s \, \frac{n}{s} + s \, \frac{n'}{s}
  = n + n',
 \]
 also $\frac{n+n'}{s} = \frac{n}{s} + \frac{n'}{s}$.
 \subsubsection*{\emph{}(\ref{enum: brüche addieren})}
 Es ist
 \[
  ss' \left( \frac{n}{s} + \frac{n'}{s'} \right)
  = s' s \frac{n}{s} + s s' \frac{n'}{s'}
  = s' n + s n',
 \]
 also $\frac{s' n + s n'}{ss'} = \frac{n}{s} + \frac{n'}{s'}$.
 \subsubsection*{\emph{(\ref{enum: zähler multiplizieren})}}
 Es ist
 \[
  s r \, \frac{n}{s}
  = r s \, \frac{n}{s}
  = rn,
 \]
 also $r \, \frac{n}{s} = \frac{rn}{s}$.
\end{proof}

Gibt es eine entsprechende Abbildung $\psi$, so ist ist für alle $\frac{m}{1} \in M[S^{-1}]$
\[
 \psi\left(\frac{m}{1}\right)
 = \psi(\varphi(m))
 = \varphi'(m),
\]
und damit auch für alle $\frac{m}{s} \in M[S^{-1}]$
\[
 s \, \psi\left( \frac{m}{s} \right)
 = \psi\left( s \, \frac{m}{s} \right)
 = \psi\left( \frac{sm}{s} \right)
 = \psi\left( \frac{m}{1} \right)
 = \varphi'(m),
\]
also $\psi\left(\frac{m}{s}\right) = \frac{\varphi'(m)}{s}$. $\psi$ ist also eindeutig.

Sei nun $\psi$ so definert. $\psi$ ist dann wohldefiniert: Für $\frac{m}{s}, \frac{m'}{s'} \in M[S^{-1}]$ mit $\frac{m}{s} = \frac{m'}{s'}$ gibt es ein $t \in S$ mit $ts'm = tsm'$. Es ist daher
\[
 t s' \varphi'(m) = \varphi'(t s' m) = \varphi(tsm') = ts \varphi'(m'),
\]
also $s' \varphi'(m) = s \varphi'(m')$ und deshalb $\frac{\varphi'(m)}{s} = \frac{\varphi'(m')}{s'}$. Es gilt zu zeigen, dass $\psi$ ein $R$-Modulhomomorphismus ist. Dies ist der Fall, da für alle $\frac{m}{s}, \frac{m'}{s'} \in M[S^{-1}]$ ist
\begin{align*}
 \psi\left( \frac{m}{s} + \frac{m'}{s'} \right)
 &= \psi\left( \frac{s'm + sm'}{ss'} \right)
 = \frac{ \varphi'(s'm + sm') }{ss'} \\
 &= \frac{ s'\varphi'(m) + s\varphi'(m') }{ss'}
 = \frac{\varphi'(m)}{s} + \frac{\varphi'(m')}{s'}
 = \psi\left(\frac{m}{s}\right) + \psi\left(\frac{m'}{s'}\right)
\end{align*}
und für alle $r \in R$ und $\frac{m}{s} \in M[S^{-1}]$
\[
 \psi\left(r \, \frac{m}{s}\right)
 = \psi\left(\frac{rm}{s}\right)
 = \frac{\varphi'(rm)}{s}
 = \frac{r\varphi'(m)}{s}
 = r \, \frac{\varphi'(m)}{s}
 = r \, \psi\left(\frac{m}{s}\right).
\]


\subsection{}
Es sei $\varphi_M : M \rightarrow M[S^{-1}], m \mapsto \frac{m}{1}$, $\varphi_N$ und $\varphi_P$ seien analog definiert. Zusammen mit der exakten Sequenz
\begin{center}
 \begin{tikzpicture}
  \node (M) {$M$};
  \node (N) [right of = M] {$N$};
  \node (P) [right of = N] {$P$};
  \draw[->] (M) to node[above] {$f$} (N);
  \draw[->] (N) to node[above] {$g$} (P);
 \end{tikzpicture}
\end{center}
ergibt sich damit das folgende Diagram, in welcher die obere Zeile exakt ist.
\begin{center}
 \begin{tikzpicture}[node distance=2.6cm]
  \node (M) {$M$};
  \node (N) [right of = M] {$N$};
  \node (P) [right of = N] {$P$};
  \node (MS) [below of = M] {$M[S^{-1}]$};
  \node (NS) [below of = N] {$N[S^{-1}]$};
  \node (PS) [below of = P] {$P[S^{-1}]$};
  \draw[->] (M) to node[left] {$\varphi_M$} (MS);
  \draw[->] (N) to node[left] {$\varphi_N$} (NS);
  \draw[->] (P) to node[left] {$\varphi_P$} (PS);
  \draw[->] (M) to node[above] {$f$} (N);
  \draw[->] (N) to node[above] {$g$} (P);
 \end{tikzpicture}
\end{center}
Wir bemerken, dass für alle $s \in S$ die Abbildung $\tau_s: M[S^{-1}] \rightarrow M[S^{-1}]$, $\frac{m'}{s'} \mapsto s \frac{m'}{s'} = \frac{sm'}{s'}$ auf dem $R$-Modul $M[S^{-1}]$ bijektiv ist, denn für die Abbildung $\tau_{1/s} : M[S^{-1}] \rightarrow M[S^{-1}], \frac{m'}{s'} \mapsto \frac{1}{s} \, \frac{m'}{s'} = \frac{m'}{ss'}$ ist $\tau_s \tau_{1/s} = \tau_{1/s} \tau_s = \id_{M[S^{-1}]}$. Analoges gilt für $N[S^{-1}]$ und $P[S^{-1}]$.

Nach dem vorherigen Aufgabenteil gibt es für den $R$-Modulhomomorphismus $\varphi_N f : M \rightarrow N[S^{-1}]$ einen eindeutigen $R$-Modulhomomorphismus $\bar{f} : M[S^{-1}] \rightarrow N[S^{-1}]$ mit $\bar{f} \varphi_M = \varphi_N f$. Analog gibt es einen eindeutigen $R$-Modulhomomorphismus $\bar{g} : N[S^{-1}] \rightarrow P[S^{-1}]$ mit $\bar{g} \varphi_N = \varphi_P g$. Das bedeutet, dass diese beiden Homomorphismen die beiden eindeutigen sind, für die das folgende Diagram kommutiert:
\begin{center}
 \begin{tikzpicture}[node distance=2.6cm]
  \node (M) {$M$};
  \node (N) [right of = M] {$N$};
  \node (P) [right of = N] {$P$};
  \node (MS) [below of = M] {$M[S^{-1}]$};
  \node (NS) [below of = N] {$N[S^{-1}]$};
  \node (PS) [below of = P] {$P[S^{-1}]$};
  \draw[->] (M) to node[left] {$\varphi_M$} (MS);
  \draw[->] (N) to node[left] {$\varphi_N$} (NS);
  \draw[->] (P) to node[left] {$\varphi_P$} (PS);
  \draw[->] (M) to node[above] {$f$} (N);
  \draw[->] (N) to node[above] {$g$} (P);
  \draw[->] (MS) to node[below] {$\exists!\, \bar{f}$} (NS);
  \draw[->] (NS) to node[below] {$\exists!\, \bar{g}$} (PS);
 \end{tikzpicture}
\end{center}

Aus dem bisherigen Aufgabenteil ergibt sich auch direkt, dass $\bar{f}\left(\frac{m}{s}\right) = \frac{f(s)}{s}$ für alle $\frac{m}{s} \in M[S^{-1}]$, und $\bar{g}\left(\frac{n}{s}\right) = \frac{g(n)}{s}$ für alle $\frac{n}{s} \in N[S^{-1}]$. (Man beachte etwa, dass $\varphi' = \varphi_N f$ und für das Element $n = \frac{n'}{s'} \in N[S^{-1}]$ und $s \in S$ die Notation aus dem vorherigen Aufgabenteil $\frac{n}{s}$ das Element $\frac{n'}{ss'}$ beschreibt.)

Die $R$-Modulhomomorphismen $\bar{f}$ und $\bar{g}$ sind auch $R[S^{-1}]$-Modulhomomorphismen. Für alle $\frac{r}{s} \in R[S^{-1}]$ und $\frac{m'}{s'} \in M[S^{-1}]$ ist
\[
 \bar{f}\left( \frac{r}{s} \, \frac{m'}{s'} \right)
 = \bar{f}\left( \frac{rm'}{ss'} \right)
 = \frac{f(rm')}{ss'}
 = \frac{r f(m')}{ss'}
 = \frac{r}{s} \, \frac{f(m')}{s'}
 = \frac{r}{s} \, \bar{f}\left( \frac{m'}{s'} \right).
\]
Für $\bar{g}$ läuft der Beweis analog.

Aus der Exaktheit der $R$-Modulhomomorphismen
\begin{center}
 \begin{tikzpicture}
  \node (M) {$M$};
  \node (N) [right of = M] {$N$};
  \node (P) [right of = N] {$P$};
  \draw[->] (M) to node[above] {$f$} (N);
  \draw[->] (N) to node[above] {$g$} (P);
 \end{tikzpicture}
\end{center}
folgt die Exaktheit der $R[S^{-1}]$-Modulhomomorphismen
\begin{center}
 \begin{tikzpicture}[node distance = 2cm]
  \node (MS) {$M[S^{-1}]$};
  \node (NS) [right of = MS] {$N[S^{-1}]$};
  \node (PS) [right of = NS] {$P[S^{-1}].$};
  \draw[->] (MS) to node[above] {$\bar{f}$} (NS);
  \draw[->] (NS) to node[above] {$\bar{g}$} (PS);
 \end{tikzpicture}
\end{center}

Für alle $\frac{m}{s} \in M[S^{-1}]$ ist
\[
 (\bar{g} \bar{f})\left( \frac{m}{s} \right)
 = \frac{gf(m)}{s}
 = \frac{0}{s}
 = \frac{0}{1},
\]
da $gf = 0$, also $\Img \bar{f} \subseteq \Ker \bar{g}$.

Für alle $\frac{n}{s} \in N[S^{-1}]$ ist
\[
 \bar{g}\left(\frac{n}{s}\right) = \frac{0}{1}
 \Leftrightarrow \frac{g(n)}{s} = \frac{0}{1}
 \Leftrightarrow \exists\, t \in S : tg(n) = 0.
\]
Für ein solches $t$ ist
\[
 tg(n) = 0 
 \Leftrightarrow g(tn) = 0
 \Leftrightarrow tn \in \Ker g = \Img f
 \Leftrightarrow \exists\, m \in M : f(m) = tn.
\]
Also gibt es für $\frac{n}{s} \in \Ker \bar{g}$ ein $t \in S$ und $m \in M$ mit $f(m) = tn$, weshalb
\[
 \bar{f}\left( \frac{m}{ts} \right)
 = \frac{f(m)}{ts}
 = \frac{tn}{ts}
 = \frac{n}{s}.
\]
Also ist $\Ker \bar{g} \subseteq \Img \bar{f}$.





\section{}
Für die kurze exakte Sequenz
\begin{center}
 \begin{tikzpicture}
  \node (0_1) {$0$};
  \node (M) [right of = 0_1] {$M$};
  \node (N) [right of = M] {$N$};
  \node (P) [right of = N] {$P$};
  \node (0_2) [right of = P] {$0$};
  \draw[->] (0_1) to (M);
  \draw[->] (M) to node[above] {$f$} (N);
  \draw[->] (N) to node[above] {$g$} (P);
  \draw[->] (P) to (0_2);
 \end{tikzpicture}
\end{center}
ist $f$ injektiv, also $M \cong \Img f \subseteq N$, und $g$ surjektiv, also $P \cong N / \ker g = N / \Img f$. Da die Länge eines Moduls invariant unter Isomorphie ist, können wir daher o.B.d.A. davon ausgehen, dass $M \subseteq N$ ein Untermodul ist und $P = N/M$. Es gilt also zu zeigen, dass
\[
 l_A(N) = l_A(M) + l_A(N/M).
\]

Es bezeichne $\pi : N \rightarrow N/M$ die kanonische Projektion. Offenbar induziert $\pi$ ein Bijektion zwischen den Untermodulen von $N$, die $M$ beinhalten, und den Untermodulen von $N/M$. Daher ergibt sich aus jeder Kette von $M$
\[
 0 = M_0 \subsetneq M_1 \subsetneq \ldots \subsetneq M_r = M
\]
der Länge $r$ und Kette von $N/M$
\[
 0 = P_0 \subsetneq P_1 \subsetneq \ldots \subsetneq P_s = N/M
\]
der Länge $s$ eine Kette von $N$
\[
 0 = M_0 \subsetneq \ldots \subsetneq M_r = \pi^{-1}(P_0) \subsetneq \ldots \subsetneq \pi^{-1}(P_r) = N
\]
der Länge $r+s$. Daher ist
\[
 l_A(N) \geq l_A(M) + l_A(N/M).
\]

Andererseits ergibt sich aus einer Kette
\[
 0 = N_0 \subsetneq N_1 \subsetneq \ldots \subsetneq N_t = N
\]
der Länge $t$ von $N$ eine Kette
\[
 0 = M \cap N_0 \subsetneq M \cap N_1 \subseteq \ldots \subseteq M \cap N_t = M
\]
von $M$ und eine Kette
\[
 0 = \pi(N_0) \subseteq \pi(N_1) \subseteq \ldots \subseteq \pi(N_t) = N
\]
von $N$. Da $\ker \pi = N$ und $N_i \subsetneq N_{i+1}$ für alle $i=0, \ldots, t-1$ ist $M \cap N_i \subsetneq M \cap N_{i+1}$ oder $\pi(N_i) \subsetneq \pi(N_{i+1})$ für alle $i=0, \ldots, t-1$. Deshalb ist
\[
 l_A(N) \leq l_A(M) + l_A(N/M).
\]





\section{}





\section{}


\subsection{}
Es bezeichne $T(\bigoplus_{i \in I} M_i)$ den Torsionsuntermodul von $\bigoplus_{i \in I} M_i$ und für alle $i \in I$ bezeichne $T(M_i) = T_i$ den Torsionsuntermodul von $M_i$. Es ist $\bigoplus_{i \in I} T_i \subseteq \bigoplus_{i \in I} M_i$ und
\[
 T\left( \bigoplus_{i \in I} M_i \right) = \bigoplus_{i \in I} T(M_i) = \bigoplus_{i \in I} T_i.
\]

Für $(m_i)_{i \in I} \in T\left( \bigoplus_{i \in I} M_i \right)$ gibt ein $r \in R \smallsetminus \{0\}$ mit $r(m_i)_{i \in I} = (rm_{i})_{i \in I} = 0$, also $rm_i = 0$ für alle $i \in I$. Daher ist $m_i \in T_i$ für alle $i \in I$. Da $m_i = 0$ für fast alle $i \in I$ ist $(m_i)_{i \in I} \in \bigoplus_{i \in I} T_i$.

Für $(m_i)_{i \in I} \in \bigoplus T_i$ ist $m_i = 0$ für fast alle $i \in I$. Es seien $i_1, \ldots, i_n \in I$ genau die Indizes mit $m_{i_j} \neq 0$. Da $m_{i_j} \in T_{i_j}$ für alle $j=1,\ldots,n$ gibt es für alle $j=1,\ldots,n$ ein $r_j \in R \smallsetminus \{0\}$ mit $r_j m_{i_j} \neq 0$. Da $R$ kommutativ ist, ist daher $(r_1 \cdots r_n) m_i = 0$ für alle $i \in I$, also $(r_1 \cdots r_n) (m_i)_{i \in I} = 0$. Da $R$ ein Integritätsring ist, ist $r_1 \cdots r_n \neq 0$, da $r_j \neq 0$ für alle $j=1,\ldots,n$. Daher ist $(m_i)_{i \in I} \in T(\bigoplus_{i \in I} M_i)$.


\subsection{}
Es bezeichne $P \subsetneq \N$ die Menge aller Primzahlen. Für alle $p \in P$ ist $\Z/p\Z$ eine abelsche Gruppe, die wir in naheliegender Weise als $\Z$-Modul auffassen. Jedes $x \in \Z/p\Z$ mit $x \neq 0$ hat Ordnung $p$, weshalb $n \cdot x = 0 \Leftrightarrow p \mid n$ für alle $n \in \Z$. Da jedes $\Z/p\Z$ ein Torsionsmodul ist, ist
\[
 \prod_{p \in P} T(\Z/p\Z) = \prod_{p \in P} \Z/p\Z.
\]
Dies ist kein Torsionsmodul: Für $(1_{\Z/p\Z})_{p \in P} \in \prod_{p \in P} \Z/p\Z$ und $n \in \Z$ mit
\[
 n \cdot (1_{\Z/p\Z})_{p \in P} = (n \cdot 1_{\Z/p\Z})_{p \in P} = 0
\]
muss $n \mid p$ für alle $p \in P$, also $n = 0$. Deshalb ist $\prod_{p \in P} T(\Z/p\Z)$ nicht isomorph zum Torsionsmodul $T(\prod_{p \in P} \Z/p\Z)$.





\end{document}
