\documentclass[a4paper,10pt]{article}
%\documentclass[a4paper,10pt]{scrartcl}

\usepackage{xltxtra}
\usepackage[ngerman]{babel}
\usepackage{amsmath}
\usepackage{amssymb}
\usepackage{amsthm}
\usepackage{mathtools}
\usepackage{nicefrac}
\usepackage{enumerate}
\usepackage{leftidx}
\usepackage{tikz}

\newcounter{satze}
\newtheorem{beh}[satze]{Behauptung}
\newtheorem{bem}[satze]{Bemerkung}
\newtheorem{lem}[satze]{Lemma}
\newtheorem*{defi}{Definition}
\newtheorem*{anm}{Anmerkung}
\theoremstyle{definition}
\newtheorem*{ia}{Induktionsanfang}
\newtheorem*{is}{Induktionsschritt}

\renewcommand{\thesection}{Aufgabe 9.\arabic{section}.}
\renewcommand{\thesubsection}{(\roman{subsection})}
\renewcommand{\thesubsubsection}{}

\newcommand{\N}{\mathbb{N}}
\newcommand{\Z}{\mathbb{Z}}
\newcommand{\Q}{\mathbb{Q}}
\newcommand{\R}{\mathbb{R}}
\newcommand{\C}{\mathbb{C}}
\newcommand{\Sn}{\mathfrak{S}}
\newcommand{\mf}[1]{\mathfrak{#1}}
\newcommand{\mc}[1]{\mathcal{#1}}
\newcommand{\dt}{\,\text{d}t}
\newcommand{\F}{\mathbb{F}}
\newcommand{\id}{\operatorname{id}}
\newcommand{\ord}{\operatorname{ord}}
\newcommand{\inn}{\operatorname{inn}}
\newcommand{\sgn}{\operatorname{sgn}}
\newcommand{\kchar}{\operatorname{char}}
\newcommand{\kgV}{\operatorname{kgV}}
\newcommand{\ggT}{\operatorname{ggT}}
\newcommand{\nil}{\operatorname{nil}}
\newcommand{\Id}{\operatorname{Id}}
\newcommand{\GL}[2]{\operatorname{GL}(#1,#2)}
\newcommand{\Img}{\operatorname{Im}}
\newcommand{\Ker}{\operatorname{Ker}}
\newcommand{\Hom}{\operatorname{Hom}}
\newcommand{\End}{\operatorname{End}}
\newcommand{\Aut}{\operatorname{Aut}}
\newcommand{\Inn}{\operatorname{Inn}}
\newcommand{\vect}[1]{\begin{pmatrix}#1\end{pmatrix}}
\newcommand{\gen}[1]{\left\langle#1\right\rangle}
\newcommand{\lb}{[\![}
\newcommand{\rb}{]\!]}
\newcommand{\Deg}{\operatorname{Deg}}

\newenvironment{lgs}[1][c]{\left\{\setlength{\arraycolsep}{1pt}\begin{array}{#1}}{\end{array}\right.}

\renewcommand*{\arraystretch}{1.5}

\makeatletter
\renewcommand*\env@matrix[1][*\c@MaxMatrixCols c]{%
  \hskip -\arraycolsep
  \let\@ifnextchar\new@ifnextchar
  \array{#1}}
\makeatother

\setromanfont[Mapping=tex-text]{Linux Libertine O}
% \setsansfont[Mapping=tex-text]{DejaVu Sans}
% \setmonofont[Mapping=tex-text]{DejaVu Sans Mono}
\parindent0pt

\title{\textsc{Einführung in die Algebra \\ \Large Blatt 9}}
\author{Jendrik Stelzner}
\date{\today}

\begin{document}
\maketitle





\section{}





\section{}





\section{}
Für die kurze exakte Sequenz
\begin{center}
 \begin{tikzpicture}
  \node (0_1) {$0$};
  \node (M) [right of = 0_1] {$M$};
  \node (N) [right of = M] {$N$};
  \node (P) [right of = N] {$P$};
  \node (0_2) [right of = P] {$0$};
  \draw[->] (0_1) to (M);
  \draw[->] (M) to node[above] {$f$} (N);
  \draw[->] (N) to node[above] {$g$} (P);
  \draw[->] (P) to (0_2);
 \end{tikzpicture}
\end{center}
ist $f$ injektiv, also $M \cong \Img f \subseteq N$, und $g$ surjektiv, also $P \cong N / \ker g = N / \Img f$. Da die Länge eines Moduls invariant unter Isomorphie ist, können wir daher o.B.d.A. davon ausgehen, dass $M \subseteq N$ ein Untermodul ist und $P = N/M$. Es gilt also zu zeigen, dass
\[
 l_A(N) = l_A(M) + l_A(N/M).
\]

Es bezeichne $\pi : N \rightarrow N/M$ die kanonische Projektion. Offenbar induziert $\pi$ ein Bijektion zwischen den Untermodulen von $N$, die $M$ beinhalten, und den Untermodulen von $N/M$. Daher ergibt sich aus jeder Kette von $M$
\[
 0 = M_0 \subsetneq M_1 \subsetneq \ldots \subsetneq M_r = M
\]
der Länge $r$ und Kette von $N/M$
\[
 0 = P_0 \subsetneq P_1 \subsetneq \ldots \subsetneq P_s = N/M
\]
der Länge $s$ eine Kette von $N$
\[
 0 = M_0 \subsetneq \ldots \subsetneq M_r = \pi^{-1}(P_0) \subsetneq \ldots \subsetneq \pi^{-1}(P_r) = N
\]
der Länge $r+s$. Daher ist
\[
 l_A(N) \geq l_A(M) + l_A(N/M).
\]

Andererseits ergibt sich aus einer Kette
\[
 0 = N_0 \subsetneq N_1 \subsetneq \ldots \subsetneq N_t = N
\]
der Länge $t$ von $N$ eine Kette
\[
 0 = M \cap N_0 \subsetneq M \cap N_1 \subseteq \ldots \subseteq M \cap N_t = M
\]
von $M$ und eine Kette
\[
 0 = \pi(N_0) \subseteq \pi(N_1) \subseteq \ldots \subseteq \pi(N_t) = N
\]
von $N$. Da $\ker \pi = N$ und $N_i \subsetneq N_{i+1}$ für alle $i=0, \ldots, t-1$ ist $M \cap N_i \subsetneq M \cap N_{i+1}$ oder $\pi(N_i) \subsetneq \pi(N_{i+1})$ für alle $i=0, \ldots, t-1$. Deshalb ist
\[
 l_A(N) \leq l_A(M) + l_A(N/M).
\]





\section{}





\section{}


\subsection{}
Es bezeichne $T(\bigoplus_{i \in I} M_i)$ den Torsionsuntermodul von $\bigoplus_{i \in I} M_i$ und für alle $i \in I$ bezeichne $T(M_i) = T_i$ den Torsionsuntermodul von $M_i$. Es ist $\bigoplus_{i \in I} T_i \subseteq \bigoplus_{i \in I} M_i$ und
\[
 T\left( \bigoplus_{i \in I} M_i \right) = \bigoplus_{i \in I} T(M_i) = \bigoplus_{i \in I} T_i.
\]

Für $(m_i)_{i \in I} \in T\left( \bigoplus_{i \in I} M_i \right)$ gibt ein $r \in R \smallsetminus \{0\}$ mit $r(m_i)_{i \in I} = (rm_{i})_{i \in I} = 0$, also $rm_i = 0$ für alle $i \in I$. Daher ist $m_i \in T_i$ für alle $i \in I$. Da $m_i = 0$ für fast alle $i \in I$ ist $(m_i)_{i \in I} \in \bigoplus_{i \in I} T_i$.

Für $(m_i)_{i \in I} \in \bigoplus T_i$ ist $m_i = 0$ für fast alle $i \in I$. Es seien $i_1, \ldots, i_n \in I$ genau die Indizes mit $m_{i_j} \neq 0$. Da $m_{i_j} \in T_{i_j}$ für alle $j=1,\ldots,n$ gibt es für alle $j=1,\ldots,n$ ein $r_j \in R \smallsetminus \{0\}$ mit $r_j m_{i_j} \neq 0$. Da $R$ kommutativ ist, ist daher $(r_1 \cdots r_n) m_i = 0$ für alle $i \in I$, also $(r_1 \cdots r_n) (m_i)_{i \in I} = 0$. Da $R$ ein Integritätsring ist, ist $r_1 \cdots r_n \neq 0$, da $r_j \neq 0$ für alle $j=1,\ldots,n$. Daher ist $(m_i)_{i \in I} \in T(\bigoplus_{i \in I} M_i)$.


\subsection{}
Es bezeichne $P \subsetneq \N$ die Menge aller Primzahlen. Für alle $p \in P$ ist $\Z/p\Z$ eine abelsche Gruppe, die wir in naheliegender Weise als $\Z$-Modul auffassen. Jedes $x \in \Z/p\Z$ mit $x \neq 0$ hat Ordnung $p$, weshalb $n \cdot x = 0 \Leftrightarrow p \mid n$ für alle $n \in \Z$. Da jedes $\Z/p\Z$ ein Torsionsmodul ist, ist
\[
 \prod_{p \in P} T(\Z/p\Z) = \prod_{p \in P} \Z/p\Z.
\]
Dies ist kein Torsionsmodul: Für $(1_{\Z/p\Z})_{p \in P} \in \prod_{p \in P} \Z/p\Z$ und $n \in \Z$ mit
\[
 n \cdot (1_{\Z/p\Z})_{p \in P} = (n \cdot 1_{\Z/p\Z})_{p \in P} = 0
\]
muss $n \mid p$ für alle $p \in P$, also $n = 0$. Deshalb ist $\prod_{p \in P} T(\Z/p\Z)$ nicht isomorph zum Torsionsmodul $T(\prod_{p \in P} \Z/p\Z)$.























\end{document}
