\documentclass[a4paper,10pt]{article}
%\documentclass[a4paper,10pt]{scrartcl}

\usepackage{xltxtra}
\usepackage[ngerman]{babel}
\usepackage{amsmath}
\usepackage{amssymb}
\usepackage{amsthm}
\usepackage{mathtools}
\usepackage{nicefrac}
\usepackage{enumerate}
\usepackage{leftidx}
\usepackage{tikz}

\newcounter{satze}
\newtheorem{beh}[satze]{Behauptung}
\newtheorem{bem}[satze]{Bemerkung}
\newtheorem{lem}[satze]{Lemma}
\newtheorem*{defi}{Definition}
\newtheorem*{anm}{Anmerkung}
\theoremstyle{definition}
\newtheorem*{ia}{Induktionsanfang}
\newtheorem*{is}{Induktionsschritt}

\renewcommand{\thesection}{Aufgabe 10.\arabic{section}.}
\renewcommand{\thesubsection}{(\roman{subsection})}
\renewcommand{\thesubsubsection}{}

\newcommand{\N}{\mathbb{N}}
\newcommand{\Z}{\mathbb{Z}}
\newcommand{\Q}{\mathbb{Q}}
\newcommand{\R}{\mathbb{R}}
\newcommand{\C}{\mathbb{C}}
\newcommand{\Sn}{\mathfrak{S}}
\newcommand{\mf}[1]{\mathfrak{#1}}
\newcommand{\mc}[1]{\mathcal{#1}}
\newcommand{\dt}{\,\text{d}t}
\newcommand{\F}{\mathbb{F}}
\newcommand{\id}{\operatorname{id}}
\newcommand{\ord}{\operatorname{ord}}
\newcommand{\inn}{\operatorname{inn}}
\newcommand{\sgn}{\operatorname{sgn}}
\newcommand{\kchar}{\operatorname{char}}
\newcommand{\kgV}{\operatorname{kgV}}
\newcommand{\ggT}{\operatorname{ggT}}
\newcommand{\nil}{\operatorname{nil}}
\newcommand{\Id}{\operatorname{Id}}
\newcommand{\GL}[2]{\operatorname{GL}(#1,#2)}
\newcommand{\Img}{\operatorname{Im}}
\newcommand{\Ker}{\operatorname{Ker}}
\newcommand{\Hom}{\operatorname{Hom}}
\newcommand{\End}{\operatorname{End}}
\newcommand{\Aut}{\operatorname{Aut}}
\newcommand{\Inn}{\operatorname{Inn}}
\newcommand{\vect}[1]{\begin{pmatrix}#1\end{pmatrix}}
\newcommand{\gen}[1]{\left\langle#1\right\rangle}
\newcommand{\lb}{[\![}
\newcommand{\rb}{]\!]}
\newcommand{\Deg}{\operatorname{Deg}}
\newcommand{\rg}{\operatorname{rg}}

\newenvironment{lgs}[1][c]{\left\{\setlength{\arraycolsep}{1pt}\begin{array}{#1}}{\end{array}\right.}

\renewcommand*{\arraystretch}{1.5}

\makeatletter
\renewcommand*\env@matrix[1][*\c@MaxMatrixCols c]{%
  \hskip -\arraycolsep
  \let\@ifnextchar\new@ifnextchar
  \array{#1}}
\makeatother

\setromanfont[Mapping=tex-text]{Linux Libertine O}
% \setsansfont[Mapping=tex-text]{DejaVu Sans}
% \setmonofont[Mapping=tex-text]{DejaVu Sans Mono}
\parindent0pt

\title{\textsc{Einführung in die Algebra \\ \Large Blatt 10}}
\author{Jendrik Stelzner}
\date{\today}

\begin{document}
\maketitle





\addtocounter{section}{3}




\section{}
Für einen beliebigen Körper $K$ und beliebiges $g \in K[X]$ mit $\deg g \geq 1$ gilt, da $K[X]$ ein Hauptidealring ist, bekanntermaßen
\[
 K[X]/(g) \text{ ist ein Körper }
 \Leftrightarrow (g) \text{ ist maximal }
 \Leftrightarrow g \text{ ist irreduzibel}.
\]
Da das Polynom $f = X^3 - 2$ irreduzibel in $\Q[X]$ ist, nicht jedoch in $\R[X]$, ist $\Q[X]/(f)$ ein Körper, $\R[X]/(f)$ jedoch nicht.





\section{}


\subsection{}
Da $\alpha$ und $\beta$ algebraisch über $K$ sind, ist die Körpererweiterung $K(\alpha,\beta)/K$ algebraisch und $K(\alpha, \beta) = K[\alpha, \beta]$. Insbesondere ist daher $(\alpha^k \beta^l)_{k,l \in \N}$ ein $K$-Erzeugenden\-system von $K(\alpha,\beta)$.

Sei nun $x_1, \ldots, x_m \in K(\alpha)$ eine $K$-Basis von $K(\alpha)$ und $y_1, \ldots, y_n \in K(\beta)$ eine $K$-Basis von $K(\beta)$. Es ist $(x_i y_j)_{i=1,\ldots,m, j=1,\ldots,n}$ ein $K$-Erzeugendensystem von $K(\alpha,\beta)$, und damit insbesondere
\[
 K[(\alpha,\beta) : K] = \dim_K(K(\alpha,\beta)) \leq mn
\]
Für alle $k,l \in \N$ gibt es nämlich (eindeutige) $\lambda^k_1, \ldots, \lambda^k_m \in K$ mit $\alpha^k = \sum_{i=1}^m \lambda^k_i x_i$ und $\mu^l_1, \ldots, \mu^l_n \in K$ mit $\beta^l = \sum_{j=1}^n \mu^l_j y_j$, weshalb
\[
 \alpha^k \beta^l
 = \left( \sum_{i=1}^m \lambda^k_i x_i \right)\left( \sum_{j=1}^n \mu^l_j y_j \right)
 = \sum_{i=1}^m \sum_{j=1}^n \lambda^k_i \mu^l_j x_i y_j.
\]
da $(\alpha^k \beta^l)_{k,l \in \N}$ ein $K$-Erzeugendensystem von $K(\alpha,\beta)$ ist, ist es daher auch $(x_i y_j)_{i,j}$. 


\subsection{}
Aus der Kette von Körpererweiterungen
\[
 K \subseteq K(\alpha) \subseteq K(\alpha,\beta)
\]
ergibt sich durch den Gradsatz, dass
\begin{gather*}
 [K(\alpha, \beta) : K]
 = [K(\alpha, \beta) : K(\alpha)] \cdot [K(\alpha) : K],
\shortintertext{also}
 m = [K(\alpha) : K] \mid [K(\alpha,\beta) : K]
\end{gather*}
Analog ergibt sich, dass auch $n \mid [K(\alpha, \beta) : K]$. Folglich ist auch
\[
 \kgV(m,n) \mid [K(\alpha, \beta)].
\]
Dabei ist $\kgV(m,n) = mn$, da $m$ und $n$ teilerfremd sind. Mit $[K(\alpha,\beta) : K] \geq 1$ ergibt sich damit, dass $mn \leq [K(\alpha, \beta) : K]$. Da auch $[K(\alpha,\beta) : K] \leq mn$ ist also
\[
 [K(\alpha,\beta) : K] = mn.
\]
















\end{document}
