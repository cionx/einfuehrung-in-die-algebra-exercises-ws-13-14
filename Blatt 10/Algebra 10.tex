\documentclass[a4paper,10pt]{article}
%\documentclass[a4paper,10pt]{scrartcl}

\usepackage{xltxtra}
\usepackage[ngerman]{babel}
\usepackage{amsmath}
\usepackage{amssymb}
\usepackage{amsthm}
\usepackage{mathtools}
\usepackage{nicefrac}
\usepackage{enumerate}
\usepackage{leftidx}
\usepackage{tikz}

\newcounter{satze}
\newtheorem{beh}[satze]{Behauptung}
\newtheorem{bem}[satze]{Bemerkung}
\newtheorem{lem}[satze]{Lemma}
\newtheorem*{defi}{Definition}
\newtheorem*{anm}{Anmerkung}
\theoremstyle{definition}
\newtheorem*{ia}{Induktionsanfang}
\newtheorem*{is}{Induktionsschritt}

\renewcommand{\thesection}{Aufgabe 10.\arabic{section}.}
\renewcommand{\thesubsection}{(\roman{subsection})}
\renewcommand{\thesubsubsection}{}

\newcommand{\N}{\mathbb{N}}
\newcommand{\Z}{\mathbb{Z}}
\newcommand{\Q}{\mathbb{Q}}
\newcommand{\R}{\mathbb{R}}
\newcommand{\C}{\mathbb{C}}
\newcommand{\Sn}{\mathfrak{S}}
\newcommand{\mf}[1]{\mathfrak{#1}}
\newcommand{\mc}[1]{\mathcal{#1}}
\newcommand{\dt}{\,\text{d}t}
\newcommand{\F}{\mathbb{F}}
\newcommand{\id}{\operatorname{id}}
\newcommand{\ord}{\operatorname{ord}}
\newcommand{\inn}{\operatorname{inn}}
\newcommand{\sgn}{\operatorname{sgn}}
\newcommand{\kchar}{\operatorname{char}}
\newcommand{\kgV}{\operatorname{kgV}}
\newcommand{\ggT}{\operatorname{ggT}}
\newcommand{\nil}{\operatorname{nil}}
\newcommand{\Id}{\operatorname{Id}}
\newcommand{\GL}[2]{\operatorname{GL}(#1,#2)}
\newcommand{\Img}{\operatorname{Im}}
\newcommand{\Ker}{\operatorname{Ker}}
\newcommand{\Hom}{\operatorname{Hom}}
\newcommand{\End}{\operatorname{End}}
\newcommand{\Aut}{\operatorname{Aut}}
\newcommand{\Inn}{\operatorname{Inn}}
\newcommand{\vect}[1]{\begin{pmatrix}#1\end{pmatrix}}
\newcommand{\gen}[1]{\left\langle#1\right\rangle}
\newcommand{\lb}{[\![}
\newcommand{\rb}{]\!]}
\newcommand{\Deg}{\operatorname{Deg}}
\newcommand{\rg}{\operatorname{rg}}

\newenvironment{lgs}[1][c]{\left\{\setlength{\arraycolsep}{1pt}\begin{array}{#1}}{\end{array}\right.}

\renewcommand*{\arraystretch}{1.5}

\makeatletter
\renewcommand*\env@matrix[1][*\c@MaxMatrixCols c]{%
  \hskip -\arraycolsep
  \let\@ifnextchar\new@ifnextchar
  \array{#1}}
\makeatother

\setromanfont[Mapping=tex-text]{Linux Libertine O}
% \setsansfont[Mapping=tex-text]{DejaVu Sans}
% \setmonofont[Mapping=tex-text]{DejaVu Sans Mono}
\parindent0pt

\title{\textsc{Einführung in die Algebra \\ \Large Blatt 10}}
\author{Jendrik Stelzner}
\date{\today}

\begin{document}
\maketitle





\section{}
Da ich in den Ferien keine Lust auf Algebra hatte und mich daher nicht mit Elementarteilertheorie beschäftigt habe, habe ich nur die anderen Aufgaben bearbeitet.





\section{}


\begin{bem}\label{bem: elemente der ordnung p}
 Sei $p \in \N$ eine Primzahl und $k \in \N, k \geq 1$. Dann gibt es in $\Z/p^k\Z$ genau $p$ Elemente $g \in \Z/p^k\Z$ mit $pg = 0$.
\end{bem}
\begin{proof}
 Es bezeichne $\pi : \Z \rightarrow \Z/p^k\Z$ die kanonische Projektion und $\bar{n} = \pi(n)$ für alle $n \in \Z$. Für alle $n \in \Z$ ist
 \begin{align*}
  p \cdot \bar{n} = 0
  &\Leftrightarrow \overline{np} = 0
  \Leftrightarrow np \in \Ker \pi = p^k\Z
  \Leftrightarrow n \in p^{k-1}\Z \\
  &\Leftrightarrow \bar{n} \in \{0,\overline{p^{k-1}},\ldots,(p-1)\overline{p^{k-1}}\}.
 \end{align*}
\end{proof}


\begin{bem}\label{bem: Ordnung im Produkt}
 Seien $G_1, \ldots, G_n$ Gruppen. Für $(g_1, \ldots, g_n) \in G_1 \times \cdots \times G_n$ ist
 \[
  \ord (g_1, \ldots, g_n) = \kgV(\ord g_1, \ldots, \ord g_n).
 \]
\end{bem}
\begin{proof}
 Wir schreiben $k = \kgV(\ord g_1, \ldots, g_n)$ und $l = \ord (g_1, \ldots, g_n)$. Es ist offenbar
 \[
  (g_1, \ldots, g_n)^k = (g_1^k, \ldots, g_n^k) = (0, \ldots, 0),
 \]
 also $l \mid k$. Da
 \[
  (0, \ldots, 0) = (g_1, \ldots, g_n)^l = (g_1^l, \ldots, g_n^l)
 \]
 ist $\ord g_i \mid l$ für alle $i = 1, \ldots, n$, also auch $k \mid l$.
\end{proof}

Es bezeichne $P \subsetneq \N$ die Menge der Primzahlen.
Sei $G$ eine abelsche Gruppe mit $\ord G = 15625 = 5^6$. Aus dem Hauptsatz über endlich erzeugte abelsche Gruppen folgt, dass
\[
 G \cong \Z^d \oplus \bigoplus_{p \in P} \bigoplus_{n \geq 1} (\Z/p^n\Z)^{\nu(p,n)},
\]
wobei die $\nu(p,n)$ eindeutig bestimmt sind und $\nu(p,n) = 0$ für fast alle $(p,n) \in P \times (\N\setminus\{0\})$. Da $G$ endlich ist, ist in diesem Fall $d = 0$, und da $\ord G = 5^6$ ist $\nu(p,n) = 0$ für alle $p \in P$, $p \neq 0$, und $\sum_{n \geq 1} n \cdot \nu(5,n) = 6$. Übersichtlich ausgedrückt ist also
\[
 G \cong (\Z/5\Z)^{\nu_1} \times \cdots \times (\Z/5^6\Z)^{\nu_6}
\]
mit eindeutig bestimmten $\nu_1, \ldots, \nu_6 \in \N$ und $\nu_1 + 2 \nu_2 + \ldots + 6 \nu_6 = 6$. Da wir nur die Zahl der Isomorphieklassen ermitteln wollen, können wir im Folgenden o.B.d.A. davon ausgehen, dass $G$ von der obigen Form ist. Für
\[
 (g_1, \ldots, g_n) \in (\Z/5\Z)^{\nu_1} \times \cdots \times (\Z/5^6\Z)^{\nu_6}
\]
ist wegen Bemerkung \ref{bem: Ordnung im Produkt} genau dann
\[
 5 = \ord (g_1, \ldots, g_n) = \kgV(\ord g_1, \ldots, \ord g_n),
\]
wenn $\ord g_1, \ldots, \ord g_n \in \{1,5\}$ und $\ord g_i = 5$ für (mindestens) ein $i \in \{1, \ldots, n\}$. (Denn man bemerke, dass $\ord g_i = 5^k$ für ein $k \in \N$ für alle $i=1,\ldots,n$.)
Da es genau $124$ Elemente der Ordnung $5$ in $G$ gibt, gibt es genau $125 = 5^3$ Elemente der Ordnung $1$ oder $5$ in $G$; dies sind genau die Element $(g_1, \ldots, g_n) \in G$ mit $\ord g_i \in \{1,5\}$ für alle $i=1,\ldots,n$. Aus Bemerkung \ref{bem: Ordnung im Produkt} folgt damit, dass
\[
 G = \Z/5^{\mu_1}\Z \times \Z/5^{\mu_2}\Z \times \Z/5^{\mu_3}\Z,
\]
mit $\mu_1, \mu_2, \mu_3 \geq 1$ und $\mu_1 + \mu_2 + \mu_3 = 6$. Die Isomorphieklassen dieser Gruppen entsprechen offenbar gerade den Partitionen von $6$ in drei natürliche, positive Zahlen. Diese sind $(1,1,4)$, $(1,2,3)$ und $(2,2,2)$, d.h. $G$ ist isomorph zu einer der drei Gruppen
\[
 (\Z/5\Z)^2 \times \Z/5^4\Z \text{ oder } \Z/5\Z \times \Z/5^2\Z, \Z/5^3\Z \text{ oder } (\Z/2\Z)^3,
\]
die zueinander paarweise nicht isomorph sind. Offenbar erfüllt auch jede dieser drei Gruppen die geforderten Bedingungen. Also gibt es genau drei Isomorphieklassen.





\section{}


\subsection{}
Es ist $\kchar K = p$: Ist $P \cong \F_q$ der Primkörper von $K$, so besitzt $K$ als endlichdimensionaler $P$-Vektorraum genau $q^m$, $m \geq 1$, Elemente. Wegen $p^n = q^m$ muss $p = q$ und $n = m$. Also ist $P \cong \F_p$ und daher $\kchar K = \kchar P = p$.

Da $\kchar K = p$ ist für alle $\alpha, \beta \in K$
\[
 \sigma(\alpha + \beta) = (\alpha + \beta)^p = \alpha^p + \beta^p = \sigma(\alpha) + \sigma(\beta).
\]
Auch ist $\sigma(1) = 1^p = 1$ und
\[
 \sigma(\alpha\beta) = (\alpha\beta)^p = \alpha^p \beta^p = \sigma(\alpha) \sigma(\beta).
\]
Also ist $\sigma$ ein Ringendomorphismus. Da $K$ ein Körper ist, ist $\sigma : K \rightarrow K$ injektiv. Da $K$ endlich ist, ist $\sigma$ damit auch surjektiv. Also ist $\sigma$ ein Körperautomorphismus.


\subsection{}
Da $K$ genau $p^n$ Elemente hat, ist $\alpha^{(p^n)} = \alpha$ für alle $\alpha \in K$ (denn $K^*$ hat $p^n-1$ Elemente, und $0$ ist ein Fixpunkt von $\sigma$). Es ist also $\sigma^n = \id_K$ und daher $\ord \sigma \mid n$.

Für $t \geq 1$ mit
\[
 \alpha = \sigma^t(\alpha) = \alpha^{(p^t)} \text{ für alle } \alpha \in K
\]
ist jedes $\alpha \in K$ eine Nullstelle des Polynoms $X^{(p^t)} -1 \in K[X]$, weshalb
\[
 p^n \geq \deg \left( X^{(p^t)}-1 \right) = p^t,
\]
also $n \geq t$. Insbesondere ist $n \geq \ord \sigma$. Also ist $\ord \sigma = n$.





\section{}
Für einen beliebigen Körper $K$ und beliebiges $g \in K[X]$ mit $\deg g \geq 1$ gilt, da $K[X]$ ein Hauptidealring ist, bekanntermaßen
\[
 K[X]/(g) \text{ ist ein Körper }
 \Leftrightarrow (g) \text{ ist maximal }
 \Leftrightarrow g \text{ ist irreduzibel}.
\]
Da das Polynom $f = X^3 - 2$ irreduzibel in $\Q[X]$ ist, nicht jedoch in $\R[X]$, ist $\Q[X]/(f)$ ein Körper, $\R[X]/(f)$ jedoch nicht.





\section{}


\subsection{}
Da $\alpha$ und $\beta$ algebraisch über $K$ sind, ist die Körpererweiterung $K(\alpha,\beta)/K$ algebraisch und $K(\alpha, \beta) = K[\alpha, \beta]$. Insbesondere ist daher $(\alpha^k \beta^l)_{k,l \in \N}$ ein $K$-Erzeugenden\-system von $K(\alpha,\beta)$.

Sei nun $x_1, \ldots, x_m \in K(\alpha)$ eine $K$-Basis von $K(\alpha)$ und $y_1, \ldots, y_n \in K(\beta)$ eine $K$-Basis von $K(\beta)$. Es ist $(x_i y_j)_{i=1,\ldots,m, j=1,\ldots,n}$ ein $K$-Erzeugendensystem von $K(\alpha,\beta)$, und damit insbesondere
\[
 K[(\alpha,\beta) : K] = \dim_K(K(\alpha,\beta)) \leq mn
\]
Für alle $k,l \in \N$ gibt es nämlich (eindeutige) $\lambda^k_1, \ldots, \lambda^k_m \in K$ mit $\alpha^k = \sum_{i=1}^m \lambda^k_i x_i$ und $\mu^l_1, \ldots, \mu^l_n \in K$ mit $\beta^l = \sum_{j=1}^n \mu^l_j y_j$, weshalb
\[
 \alpha^k \beta^l
 = \left( \sum_{i=1}^m \lambda^k_i x_i \right)\left( \sum_{j=1}^n \mu^l_j y_j \right)
 = \sum_{i=1}^m \sum_{j=1}^n \lambda^k_i \mu^l_j x_i y_j.
\]
da $(\alpha^k \beta^l)_{k,l \in \N}$ ein $K$-Erzeugendensystem von $K(\alpha,\beta)$ ist, ist es daher auch $(x_i y_j)_{i,j}$. 


\subsection{}
Aus der Kette von Körpererweiterungen
\[
 K \subseteq K(\alpha) \subseteq K(\alpha,\beta)
\]
ergibt sich durch den Gradsatz, dass
\begin{gather*}
 [K(\alpha, \beta) : K]
 = [K(\alpha, \beta) : K(\alpha)] \cdot [K(\alpha) : K],
\shortintertext{also}
 m = [K(\alpha) : K] \mid [K(\alpha,\beta) : K]
\end{gather*}
Analog ergibt sich, dass auch $n \mid [K(\alpha, \beta) : K]$. Folglich ist auch
\[
 \kgV(m,n) \mid [K(\alpha, \beta)].
\]
Dabei ist $\kgV(m,n) = mn$, da $m$ und $n$ teilerfremd sind. Mit $[K(\alpha,\beta) : K] \geq 1$ ergibt sich damit, dass $mn \leq [K(\alpha, \beta) : K]$. Da auch $[K(\alpha,\beta) : K] \leq mn$ ist also
\[
 [K(\alpha,\beta) : K] = mn.
\]





\end{document}
