\documentclass[a4paper,10pt]{article}
%\documentclass[a4paper,10pt]{scrartcl}

\usepackage{xltxtra}
\usepackage[ngerman]{babel}
\usepackage{amsmath}
\usepackage{amssymb}
\usepackage{amsthm}
\usepackage{mathtools}
\usepackage{nicefrac}
\usepackage{enumerate}
\usepackage{leftidx}

\theoremstyle{definition}
\newtheorem{beh}{Behauptung}
\newtheorem{bem}[beh]{Bemerkung}
\newtheorem{lem}[beh]{Lemma}
\newtheorem*{anm}{Anmerkung}
\newtheorem*{ia}{Induktionsanfang}
\newtheorem*{is}{Induktionsschritt}

\renewcommand{\thesection}{Aufgabe 5.\arabic{section}.}
\renewcommand{\thesubsection}{(\roman{subsection})}
\renewcommand{\thesubsubsection}{}

\newcommand{\N}{\mathbb{N}}
\newcommand{\Z}{\mathbb{Z}}
\newcommand{\Q}{\mathbb{Q}}
\newcommand{\R}{\mathbb{R}}
\newcommand{\C}{\mathbb{C}}
\newcommand{\Sn}{\mathfrak{S}}
\newcommand{\mf}[1]{\mathfrak{#1}}
\newcommand{\dt}{\,\text{d}t}
\newcommand{\F}[1]{\mathbb{F}_{#1}}
\newcommand{\id}{\operatorname{id}}
\newcommand{\ord}{\operatorname{ord}}
\newcommand{\inn}{\operatorname{inn}}
\newcommand{\sgn}{\operatorname{sgn}}
\newcommand{\kchar}{\operatorname{char}}
\newcommand{\kgV}{\operatorname{kgV}}
\newcommand{\Id}{\operatorname{Id}}
\newcommand{\GL}[2]{\operatorname{GL}(#1,#2)}
\newcommand{\Img}{\operatorname{Im}}
\newcommand{\Ker}{\operatorname{Ker}}
\newcommand{\Hom}{\operatorname{Hom}}
\newcommand{\End}{\operatorname{End}}
\newcommand{\Aut}{\operatorname{Aut}}
\newcommand{\Inn}{\operatorname{Inn}}
\newcommand{\vect}[1]{\begin{pmatrix}#1\end{pmatrix}}
\newcommand{\gen}[1]{\left\langle#1\right\rangle}

\newenvironment{lgs}[1][c]{\left\{\setlength{\arraycolsep}{1pt}\begin{array}{#1}}{\end{array}\right.}

\renewcommand*{\arraystretch}{1.5}

\makeatletter
\renewcommand*\env@matrix[1][*\c@MaxMatrixCols c]{%
  \hskip -\arraycolsep
  \let\@ifnextchar\new@ifnextchar
  \array{#1}}
\makeatother

\setromanfont[Mapping=tex-text]{Linux Libertine O}
% \setsansfont[Mapping=tex-text]{DejaVu Sans}
% \setmonofont[Mapping=tex-text]{DejaVu Sans Mono}
\parindent0pt

\title{\textsc{Einführung in die Algebra \\ \Large Blatt 6}}
\author{Jendrik Stelzner}
\date{\today}

\begin{document}
\maketitle





\section{}
Es sei $n > 1$ so dass
\begin{equation}\label{eq: a^n = a}
 a^n = a \text{ für alle } a \in R,
\end{equation}
und $\mf{p}$ ein Primideal in $R$. Da $\mf{p}$ ein Primideal ist, ist $R/\mf{p}$ ein Integritätsring, sowie $R/\mf{p} \neq 0$, da $\mf{p}$ von $R$ verschieden ist. Da $R$ kommutativ ist, ist es auch $R/\mf{p}$, und es ist offensichtlich, dass die Bedingung \eqref{eq: a^n = a} auf $R/\mf{p}$ vererbt wird. Da für alle $r \in R/\mf{p}$ mit $r \neq 0$
\[
 r \cdot r^{n-1} = r^n = r = r \cdot 1,
\]
folgt, wie bereits letzte Woche gezeigt, wegen der Nullteilerfreiheit von $R/\mf{p}$, dass $r^{n-1} = 1$ für alle $r \in R/\mf{p}$. Als ist für alle $r \in R/\mf{p}$ mit $r \neq 0$
\[
 r r^{n-2} = r^{n-1} = 1,
\]
d.h. alle $r \in R/\mf{p}$ mit $r \neq 0$ sind multiplikativ invertierbar. Zusammen mit der Kommutativität von $R/\mf{p}$ und $R/\mf{p} \neq 0$ zeigt dies, dass $R/\mf{p}$ ein Körper ist. Dies ist äquivalent dazu, dass $\mf{p}$ ein maximales Ideal ist.

\section{}
Für alle $a \in \ker \varphi$ ist $1-a$ multiplikativ invertierbar: Für $n \geq 1$ mit $a^n = 0$ ergibt sich, dass
\begin{align*}
 (1+a+a^2+\ldots+a^{n-1})(1-a) &= 1-a^n = 1 \text{ und} \\
 (1-a)(1+a+a^2+\ldots+a^{n-1}) &= 1-a^n = 1.
\end{align*}
Folglich ist
\[
 1 + \ker \varphi = 1 - \ker \varphi \subseteq R^*.
\]
Wir bemerken auch, dass
\[
 x \in 1 + \ker \varphi \Leftrightarrow \varphi(x) = 1,
\]
denn da $1 \in \varphi^{-1}(\{1\})$ ist $1 + \ker \varphi$ als Nebenklasse von $1$ bezüglich $\ker \varphi$ die Faser $\varphi^{-1}(\{1\})$ von $1 \in S$ unter $\varphi$.

Bekanntermaßen induziert $\varphi$ einen Gruppenhomomorphismus $\varphi_{|R^*} : R^* \rightarrow S^*$ der entsprechenden Einheitengruppen. Die Surjektivität von $\varphi$ vererbt sich dabei auf $\varphi_{|R^*}$: Für $s\in S^*$ gibt es $r,r' \in R$ mit $\varphi(r) = s$ und $\varphi(r') = s^{-1}$. Es ist
\[
 \varphi(r r') = \varphi(r) \varphi(r') = s s^{-1} = 1,
\]
also wie oben bemerkt $r r' \in 1 + \ker \varphi \subseteq R^*$. Es ist nun nach den obigen Beobachtungen
\[
 \ker \varphi_{|R^*}
 = \{x \in R^* : \varphi(x) = 1\}
 = R^* \cap \varphi^{-1}(\{1\})
 = R^* \cap (1+ \ker \varphi)
 = 1 + \ker \varphi.
\]
Folglich ist $1 + \ker \varphi$ ein Normalteiler von $R^*$ und
\[
 R^* / (1+ \ker \varphi) \cong S^*.
\]























\end{document}
