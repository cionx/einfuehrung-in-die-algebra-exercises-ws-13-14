\documentclass[a4paper,10pt]{article}
%\documentclass[a4paper,10pt]{scrartcl}

\usepackage{xltxtra}
\usepackage[ngerman]{babel}
\usepackage{amsmath}
\usepackage{amssymb}
\usepackage{amsthm}
\usepackage{mathtools}
\usepackage{nicefrac}
\usepackage{enumerate}
\usepackage{leftidx}



\newcounter{satze}
\newtheorem{beh}[satze]{Behauptung}
\newtheorem{bem}[satze]{Bemerkung}
\newtheorem{lem}[satze]{Lemma}
\newtheorem*{defi}{Definition}
\newtheorem*{anm}{Anmerkung}
\theoremstyle{definition}
\newtheorem*{ia}{Induktionsanfang}
\newtheorem*{is}{Induktionsschritt}

\renewcommand{\thesection}{Aufgabe 8.\arabic{section}.}
\renewcommand{\thesubsection}{(\roman{subsection})}
\renewcommand{\thesubsubsection}{}

\newcommand{\N}{\mathbb{N}}
\newcommand{\Z}{\mathbb{Z}}
\newcommand{\Q}{\mathbb{Q}}
\newcommand{\R}{\mathbb{R}}
\newcommand{\C}{\mathbb{C}}
\newcommand{\Sn}{\mathfrak{S}}
\newcommand{\mf}[1]{\mathfrak{#1}}
\newcommand{\mc}[1]{\mathcal{#1}}
\newcommand{\dt}{\,\text{d}t}
\newcommand{\F}[1]{\mathbb{F}_{#1}}
\newcommand{\id}{\operatorname{id}}
\newcommand{\ord}{\operatorname{ord}}
\newcommand{\inn}{\operatorname{inn}}
\newcommand{\sgn}{\operatorname{sgn}}
\newcommand{\kchar}{\operatorname{char}}
\newcommand{\kgV}{\operatorname{kgV}}
\newcommand{\ggT}{\operatorname{ggT}}
\newcommand{\nil}{\operatorname{nil}}
\newcommand{\Id}{\operatorname{Id}}
\newcommand{\GL}[2]{\operatorname{GL}(#1,#2)}
\newcommand{\Img}{\operatorname{Im}}
\newcommand{\Ker}{\operatorname{Ker}}
\newcommand{\Hom}{\operatorname{Hom}}
\newcommand{\End}{\operatorname{End}}
\newcommand{\Aut}{\operatorname{Aut}}
\newcommand{\Inn}{\operatorname{Inn}}
\newcommand{\vect}[1]{\begin{pmatrix}#1\end{pmatrix}}
\newcommand{\gen}[1]{\left\langle#1\right\rangle}
\newcommand{\lb}{[\![}
\newcommand{\rb}{]\!]}
\newcommand{\Deg}{\operatorname{Deg}}

\newenvironment{lgs}[1][c]{\left\{\setlength{\arraycolsep}{1pt}\begin{array}{#1}}{\end{array}\right.}

\renewcommand*{\arraystretch}{1.5}

\makeatletter
\renewcommand*\env@matrix[1][*\c@MaxMatrixCols c]{%
  \hskip -\arraycolsep
  \let\@ifnextchar\new@ifnextchar
  \array{#1}}
\makeatother

\setromanfont[Mapping=tex-text]{Linux Libertine O}
% \setsansfont[Mapping=tex-text]{DejaVu Sans}
% \setmonofont[Mapping=tex-text]{DejaVu Sans Mono}
\parindent0pt

\title{\textsc{Einführung in die Algebra \\ \Large Blatt 8}}
\author{Jendrik Stelzner}
\date{\today}

\begin{document}
\maketitle




\section{}


\subsection{}
Da $rs \cdot 1 = rs \cdot 1$ für alle $(r,s) \in R \times S$ mit $1 \in S$ ist $\sim$ reflexiv. Die Symmetrie von $\sim$ ergibt sich direkt aus der Symmetrie der Gleichheit. Für $(r,s), (r',s'), (r'',s'') \in R \times S$ mit $(r,s) \sim (r',s') \sim (r'', s'')$ gibt es $t, \tilde{t} \in S$ mit
\begin{align}
 rs't &= r'st \text{ und }         \label{eq: reflexiv sr't = s'rt} \\
 r's''\tilde{t} &= r''s'\tilde{t}. \label{eq: reflexiv s'r''t = s''r't}
\end{align}
Wegen der Abgeschlossenheit von $S$ unter Multiplikation ist auch $s't\tilde{t} \in S$, und wegen der Kommutativität von $R$ daher
\[
 r s'' s' t \tilde{t}
 \underset{\eqref{eq: reflexiv sr't = s'rt}}= r' s'' s t \tilde{t}
 \underset{\eqref{eq: reflexiv s'r''t = s''r't}}= r'' s' s t \tilde{t}
 = r'' s s' t \tilde{t}.
\]
Also ist $(r,s'') \sim (r'', s)$ und $\sim$ daher transitiv.


\subsection{}
Aus der Notation der Restklassen und der Definition von $\sim$ folgt direkt, dass für alle $(r,s), (r', s') \in R \times S$
\begin{equation} \label{eq: Klassengleichheit}
 \frac{r}{s} = \frac{r'}{s'} \Leftrightarrow \text{es gibt $t \in S$ mit } rs' t = r's t.
\end{equation}
Zunächst die Wohldefiniertheit: Seien $(r,s), (\tilde{r},\tilde{s}) \in R \times S$ mit $(r,s) \sim (\tilde{r},\tilde{s})$. Dann gibt es $t \in S$ mit $r \tilde{s} t = \tilde{r} s t$. Wegen der Kommutativität von $R$ ist daher für alle $(r', s') \in R \times S$
\[
 (rs'+r's) \tilde{s} s' t
 = r s' \tilde{s} s' t + r' s \tilde{s} s' t
 = \tilde{r} s' s s' t + r' s \tilde{s} s' t
 = (\tilde{r} s', r' \tilde{s}) s s ' t,
\]
und
\[
 r r' \tilde{s} s' t = \tilde{r} r' s s' t.
\]
Da die Ausdrücke
\[
 \frac{rs'+r's}{ss'} \text{ und } \frac{r r'}{s s'}
\]
wegen der Kommutativität von $R$ symmetrisch in $(r,s)$ und $(r', s')$ sind folgt damit wegen \eqref{eq: Klassengleichheit} die Wohldefiniertheit. 

Es ist klar, dass $R[S^{-1}]$ unter Addition und Multiplikation abgeschlossen ist. Die Addition ist assoziativ und kommutativ, da wegen der Kommutatvität von $R$ für alle $\frac{r}{s}, \frac{r'}{s'}, \frac{r''}{s''} \in R[S^{-1}]$
\begin{align*}
 \frac{r}{s} + \left( \frac{r'}{s'} + \frac{r''}{s''} \right)
 &= \frac{r}{s} + \frac{r's''+r''s'}{s's''}
 = \frac{rs's''+r'ss''+r''ss'}{ss's''} \\
 &= \frac{rs'+r's}{ss'} + \frac{r''}{s''}
 = \left(\frac{r}{s} + \frac{r'}{s'}\right) + \frac{r''}{s''},
\end{align*}
sowie
\[
 \frac{r}{s} + \frac{r'}{s'}
 = \frac{rs'+r's}{ss'}
 = \frac{r's+rs'}{s's}
 = \frac{r'}{s'} + \frac{r}{s}.
\]
Das Element $\frac{0}{1} \in R[S^{-1}]$ ist bezüglich der Addition neutral, da für alle $\frac{r}{s} \in R[S^{-1}]$
\[
 \frac{r}{s} + \frac{0}{1} = \frac{r \cdot 1 + s \cdot 0}{s \cdot 1} = \frac{r}{s},
\]
und $\frac{r}{s} \in R[S^{-1}]$ hat als additives Inverses $\frac{-r}{s}$, da
\[
 \frac{r}{s} + \frac{-r}{s} = \frac{rs-rs}{s^2} = \frac{0}{s^2} = \frac{0}{1},
\]
denn aus der Definition von $\sim$ folgt offenbar direkt, dass $\frac{0}{s} = \frac{0}{1}$ für alle $s \in S$, und wegen der Abgeschlossenheit von $S$ bezüglich der Multiplikation ist $s^2 \in S$. Also ist $R[S^{-1}]$ bezüglich der Addition eine abelsche Gruppe.

Da Multiplikation ist assoziativ und kommutativ, da für alle $\frac{r}{s}, \frac{r'}{s'}, \frac{r''}{s''} \in R[S^{-1}]$
\[
 \frac{r}{s} \left( \frac{r'}{s'} \frac{r''}{s''} \right)
 = \frac{r}{s} \frac{r' r''}{s' s''}
 = \frac{r r' r''}{s s' s''}
 = \frac{r r'}{s s'} \frac{r''}{s''}
 = \left( \frac{r}{s} \frac{r'}{s'} \right) \frac{r''}{s''},
\]
und wegen der Kommutativität von $R$
\[
 \frac{r}{s} \frac{r'}{s'}
 = \frac{r r'}{s s'}
 = \frac{r' r}{s' s}
 = \frac{r'}{s'} \frac{r}{s}.
\]
Das Element $\frac{1}{1} \in R[S^{-1}]$ ist das multiplikativ Neutrale in $R[S^{-1}]$, da für alle $\frac{r}{s} \in R[S^{-1}]$
\[
 \frac{1}{1} \frac{r}{s} = \frac{r}{s} \frac{1}{1} = \frac{r \cdot 1}{s \cdot 1} = \frac{r}{s}.
\]
Dies zeigt, dass $R[S^{-1}]$ bezüglich der Multiplikation ein abelsches Monoid ist.

Zum Nachweis des Distributivgesetzes bemerken wir zunächst:

\begin{bem}\label{bem: s als einheiten}
 Für alle $\frac{r}{s} \in R[S^{-1}]$ und $t \in S$ gilt nach \eqref{eq: Klassengleichheit} die Kürzungsregel
 \[
  \frac{rt}{st} = \frac{r}{s},
 \]
 denn wegen der Kommutativität von $R$ ist $rts \cdot 1 = rst \cdot 1$ mit $1 \in S$. Insbesondere gilt für alle $s \in S$
 \[
  \frac{s}{1} \cdot \frac{1}{s} = \frac{s}{s} = \frac{1}{1}.
 \]
\end{bem}

Mit der obigen Bemerkung erhalten wir, dass für alle $\frac{r}{s}, \frac{r'}{s'}, \frac{r''}{s''} \in R[S^{-1}]$
\begin{align*}
 \frac{r}{s} \left( \frac{r'}{s'} + \frac{r''}{s''} \right)
 &= \frac{r}{s} \frac{r's'' + r''s'}{s's''}
 = \frac{rr's''+rr''s'}{ss's''} \\
 &= \frac{rr'ss'+rr''ss'}{s^2s's''}
 = \frac{rr'}{ss'} + \frac{rr''}{ss''}
 = \frac{r}{s} \frac{r'}{s'} + \frac{r}{s} \frac{r''}{s''}.
\end{align*}
Dies zeigt, dass $R[S^{-1}]$ ein kommutativer Ring (mit Einselement) ist.


\subsection{}
Da für alle $r,r' \in R$
\[
 \varphi(r+r')
 = \frac{r'+r}{1}
 = \frac{r \cdot 1 + r' \cdot 1}{1^2}
 = \frac{r}{1} + \frac{r'}{1}
 = \varphi(r) + \varphi(r'),
\]
und
\[
 \varphi(rr')
 = \frac{rr'}{1}
 = \frac{rr'}{1^2}
 = \frac{r}{1} \frac{r'}{1}
 = \varphi(r)\varphi(r')
\]
sowie
\[
 \varphi(1_R)
 = \frac{1}{1}
 = 1_{R[S^{-1}]}
\]
ist $\varphi$ ein Ringhomomorphismus. Aus Bemerkung \ref{bem: s als einheiten} folgt, dass $\varphi(S) \subseteq (R[S^{-1}])^*$.

Wir bemerken auch direkt, dass $\varphi$ nicht zwangsweise injektiv ist: Ist $R \neq 0$ und $0 \in S$, etwa $S = R$ oder $S = \{0,1\}$, so ist offenbar $R[S^{-1}] \cong 0$, also $\varphi = 0$ und wegen $R \neq 0$ damit nicht injektiv.

Für einen Homomorphismus $\psi_S : R[S^{-1}] \rightarrow R'$ mit $\psi = \psi_S \circ \varphi$ muss für alle $r \in R$ und $s \in S$
\[
 \psi_S\left(\frac{r}{1}\right)
 = \psi_S( \varphi(r) )
 = \psi(r)
\]
und daher
\[
 \psi_S\left(\frac{1}{s}\right)
 = \psi_S\left( \left (\frac{s}{1} \right)^{-1} \right)
 = \psi_S\left( \frac{s}{1} \right)^{-1}
 = \psi_S(s)^{-1},
\]
da $\psi_S$ durch Einschränkung einen Gruppenhomomorphismus von $(R[S^{-1}])^*$ nach $(R')^*$ induziert. Also ist $\psi_S$ durch
\[
 \psi_S\left(\frac{r}{s}\right)
 = \psi_S\left(\frac{r}{1} \frac{1}{s}\right)
 = \psi_S\left(\frac{r}{1}\right) \psi_S\left(\frac{1}{s}\right)
 = \psi(r)\psi(s)^{-1}
\]
für alle $\frac{r}{s} \in R[S^{-1}]$ eindeutig bestimmt. Definiert man $\psi_S$ auf diese Art, so handelt es sich bei $\psi_S$ um einen Ringhomomorphismus, denn für alle $\frac{r}{s}, \frac{r'}{s'} \in R[S^-1]$ ist
\begin{align*}
 \psi_S\left( \frac{r}{s} + \frac{r'}{s'} \right)
 &= \psi_S\left( \frac{rs' + r's}{ss'} \right)
 = \psi(rs' + r's)\psi(ss)^{-1} \\
 &= \left( \psi(r)\psi(s') + \psi(r')\psi(s) \right) \psi(s)^{-1}\psi(s')^{-1} \psi \\
 &= \psi(r)\psi(s)^{-1} + \psi(r')\psi(s')^{-1}
 = \psi_S\left(\frac{r}{s}\right) + \psi_S\left(\frac{r'}{s'}\right),
\end{align*}
sowie
\begin{align*}
 \psi_S\left(\frac{r}{s} \frac{r'}{s'}\right)
 &= \psi_S\left(\frac{rr'}{ss'}\right)
 = \psi(rr')\psi(ss')^{-1} \\
 &= \psi(r)\psi(r')\psi(s)^{-1}\psi(s')^{-1} \\
 &= \psi(r)\psi(s)^{-1} \psi(r')\psi(s')^{-1}
 = \psi_S\left(\frac{r}{s}\right) \psi_S\left(\frac{r'}{s'}\right),
\end{align*}
und inbesondere
\[
 \psi_S\left(1_{R[S^{-1}]}\right)
 = \psi_S\left(\frac{1}{1}\right)
 = \psi(1) \psi(1)^{-1}
 = 1_{R'}.
\]





\section{}





\section{}


\begin{bem}\label{bem: def noethersch}
 Sei $R$ ein kommutativer Ring. Dann ist $R$ genau dann noethersch, wenn jedes Ideal von $R$ endlich erzeugt ist. 
\end{bem}
\begin{proof}
 Angenommen $R$ ist noethersch. Sei $I \subseteq R$ ein Ideal. Wir konstruieren eine wachsende Folge $I_0 \subseteq I_1 \subseteq \ldots$ von Idealen von $R$, mit $I_n \subseteq I$ für alle $n \in \N$, rekursiv wie folgt: Wir setzen $I_0 := 0$. Für $n \geq 1$ setzen wir $I_n := I_{n-1} + (a_n)$, falls es ein $a_n \in I \setminus I_{n-1}$ gibt, und sonst $I_n := I_{n-1}$. Da $R$ noethersch ist stabilisiert sich die Folge $(I_n)_{n \in \N}$, d.h. es gibt ein $N \in \N$ mit $I_{n+1} = I_n$ für alle $n \geq N$. Insbesondere ist $I_{N+1} = I_N$, nach Definition und von $I_{N+1}$ und $I_N \subseteq I$ also $I = I_N$. Daher ist
 \[
  I = I_N = (a_1) + \ldots + (a_N) = (a_1, \ldots, a_N)
 \]
 endlich erzeugt.
 
 Angenommen, jedes Ideal in von $R$. Für eine wachsende Folge $I_0 \subseteq I_1 \subseteq \ldots$ von Idealen von $R$ setzen wir $I = \bigcup_{n \in \N} I_n = \sum_{n \in \N} I_n$. $I$ ist als Ideal von $R$ endlich erzeugt, es gibt also $a_1, \ldots, a_m \in \R$ mit $I = (a_1, \ldots, a_m)$. Nach Definition von $I$ gibt es ein $N \in \N$ mit $a_1, \ldots, a_m \in I_N$. Also ist $I = I_N$, und damit $I_{n} = I_{n+1}$ für alle $n \geq N$.
\end{proof}


\begin{bem}\label{bem: faktorringe noethersch}
 Faktorringe kommutativer, noetherscher Ringe sind noethersch.
\end{bem}
\begin{proof}
 Sei $R$ ein kommutativer, noetherscher Ring und $I \subseteq R$ ein Ideal. Die kanonische Projektion $\pi : R \twoheadrightarrow R/I$ induziert eine Bijektion zwischen den Idealen von $R/I$ und den Idealen von $R$, die $I$ beinhalten. Jede wachsende Folge $J_0 \subseteq J_1 \subseteq \ldots$ von Idealen von $R/I$ entspricht daher einer wachsenden Folge $I_0 \subseteq I_1 \subseteq \ldots$ von Idealen von $R$ mit $I \subseteq I_i$ für alle $i \in \N$. Da $R$ noethersch ist stabilisiert sich die Folge $(I_n)_{n \in \N}$ in $R$, also auch die Folge $(J_n)_{n \in \N}$ in $R/I$. Also ist $R/I$ noethersch.
\end{proof}


Wir zeigen nun, dass auch Lokalisierungen kommutativer, noetherscher Ringe wieder noethersch sind: Es sei $R$ ein kommutativer, noetherscher Ring und $S \subseteq R$ ein Untermonoid bezüglich der Multiplikation. Es sei $I \subseteq R[S^{-1}]$ ein Ideal. Wir setzen
\[
 J := \left\{ \frac{r}{1} \in I : r \in R \right\}.
\]

Es ist $(J)_{R[S^{-1}]} = I$, wobei $(J)_{R[S^{-1}]}$ das von $J$ in $R[S^{-1}]$ erzeugte Ideal bezeichnet. Es ist klar, dass $(J)_{R[S^{-1}]} \subseteq I$. Andererseits ist für alle $\frac{r}{s} \in I$ auch $\frac{s}{1} \frac{r}{s} = \frac{rs}{s} = \frac{r}{1} \in I$, also $\frac{r}{1} \in J$, und daher auch $\frac{r}{s} = \frac{1}{s} \frac{r}{1} \in (J)_{R[S^{-1}]}$.

Es ist $J \subseteq \Img \varphi$, wobei $\varphi : R \rightarrow R[S^{-1}], r \mapsto \frac{r}{1}$. Da $R$ noethersch ist, ist es nach Bemerkung \ref{bem: faktorringe noethersch} auch $\Img \varphi \cong R / \Ker \varphi$. Es gibt also $a_1, \ldots, a_n \in \Img \varphi$ mit $(J)_{\Img \varphi} = (a_1, \ldots, a_n)_{\Img \varphi}$. Es ist daher
\begin{align*}
 I
 &= (J)_{R[S^{-1}]}
 = ((J)_{\Img \varphi})_{R[S^{-1}]} \\
 &= ((a_1, \ldots, a_n)_{\Img \varphi})_{R[S^{-1}]}
 = (a_1, \ldots, a_n)_{R[S^{-1}]},
\end{align*}
also $I$ in $R[S^{-1}]$ endlich erzeugt. Aus Bemerkung \ref{bem: def noethersch} folgt, dass $R[S^{-1}]$ noethersch ist.






















\end{document}
